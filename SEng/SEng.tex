\documentclass[a4paper, 10pt]{article}

%\usepackage{scalefnt}
%\usepackage{parcolumns}

\usepackage{newclude}
% Zum Einbinden in die Zusammenfassungs-files

\usepackage{amsmath,amsthm,amsfonts,amssymb} % Verbesserter Mathesatz
\usepackage{bigfoot} % komplexe Fußnotenapparate(Fußnoten in Fußnoten und andere Späße)%
\usepackage{colortbl}
\usepackage[T1]{fontenc} % normaler erweitere Zeichnesatz
\usepackage{framed, color}  % Ramenpaket für zum Einfügen von schönen Ramen
\usepackage{graphicx}
\usepackage{hyperref} %used for link to creative commons license
\usepackage{listings} %for code-listings (inkl. Tab-Styling)
\usepackage{marginnote}
\usepackage{microtype} % div. Verbesserungen des Schriftsatzes (Grauwert, opt. Randausgleich, Zeilenumbruch)
%\usepackage{multirow}
\usepackage[ngerman]{babel} % Neue Rechtschreibung
\usepackage[sans]{dsfont} %für alternative Mengensymbole
\usepackage{stmaryrd} %u.a. für \lightning
\usepackage{tikz,tikz-er2,tikz-uml} % für Diagramme(Dia!) und Bilder (z.B. *.eps)/für ER-Diagramme/für UML-Diagramme
\usepackage{units} % z.B. fuer \nicefrac{}{}
\usepackage[utf8]{inputenc} % utf8 für den Editor
\usepackage{wasysym} %u.a. für \lightning
\usepackage{xcolor}


\usetikzlibrary{shapes,decorations,arrows,fit,backgrounds} %Zum diversen zeichen%
%\usetikzlibrary{automata} % für den CFlipper, wenn es den soweit ist			%
\usetikzlibrary{positioning} % positionierung 
%\usetikzlibrary{shadows} % fuer schoene schlagschatten

%%%%%%%%%%%%%%%%%%%%%%%%%%%
%  Formatierung der Seite
%%%%%%%%%%%%%%%%%%%%%%%%%%%
\usepackage{fancyhdr}
\pagestyle{fancy}		% für den footer										%
\renewcommand{\headrulewidth}{0pt} % damit oben kein dummer Strich kommt		%
\fancyhead{}
\topmargin -2cm 		% Oberer Rand											%
\textheight 25cm		% Texthöhe												%
\textwidth 16.0 cm		% Textbreite											%
\oddsidemargin -0.1cm 	% Warum?												%
\newcommand{\Gruppe}[2]
{
	\lfoot{#1}
	\rfoot{#2}
}
\colorlet{shadecolor}{gray!25} % Farbe für graue Box definieren
%%%%%%%%%%%%%%%%%%%%%%%%%%%%%%%%%%%%%%%%%%%%%%%%%%%%%%%%%%%%%%%%%%%%%%%%%%%%%%%%%
%Farben die definiert werden zum schreiben und zeichnen							%
%%%%%%%%%%%%%%%%%%%%%%%%%%%%%%%%%%%%%%%%%%%%%%%%%%%%%%%%%%%%%%%%%%%%%%%%%%%%%%%%%
\xdefinecolor{schwarz}{HTML}{000000}
\xdefinecolor{dunkelGruen}{HTML}{007D00}
\xdefinecolor{dunkelBlau}{HTML}{0000A0}
\xdefinecolor{dunkelRot}{HTML}{A00000}
\xdefinecolor{dunkelGelb}{HTML}{FFAA00}
\xdefinecolor{hellesGelb}{HTML}{FFCC00}
\colorlet{dGreen}{dunkelGruen}
\colorlet{dBlue}{dunkelBlau}
\colorlet{dRed}{dunkelRot}
\colorlet{dYellow}{dunkelGelb}

%%%%%%%%%%%%%%%%%%%%%%%%%%%%%%%%%%%%%%%%%%%%%%%%%%
%Farbliche Ausgaben:
%Parameter #1: Text oder Mathematische formel...
%z.B. : \gruen{Hallo Welt Test!}
%%%%%%%%%%%%%%%%%%%%%%%%%%%%%%%%%%%%%%%%%%%%%%%%%%

\newcommand{\yellow}[1]{\textcolor{dYellow}{#1}}
\newcommand{\gray}[1]{\textcolor{gray}{#1}}
\newcommand{\red}[1]{\textcolor{red}{#1}}
\newcommand{\green}[1]{\textcolor{green}{#1}}
\newcommand{\blue}[1]{\textcolor{blue}{#1}}
\newcommand{\dGreen}[1]{\textcolor{dGreen}{#1}}
\newcommand{\dBlue}[1]{\textcolor{dBlue}{#1}}
\newcommand{\dRed}[1]{\textcolor{dRed}{#1}}

%%%%%%%%%%%%%%%%%%%%%%%%%%%%%%%%%%%%%%%%%%%%%%%%
%Konfiguration für das darstellen von Quelltext
%%%%%%%%%%%%%%%%%%%%%%%%%%%%%%%%%%%%%%%%%%%%%%%%
\lstset
{
	language=Java, % oder C++, Pascal, {[77]Fortran}, ...
	numbers=left, % Position der Zeilennummerierung
	firstnumber=auto, % Erste Zeilennummer
	basicstyle=\ttfamily, % Textgröße des Standardtexts
	keywordstyle=\ttfamily\color{dRot}, % Formattierung Schlüsselwörter
	commentstyle=\ttfamily\color{dGruen}, % Formattierung Kommentar
	stringstyle=\ttfamily\color{dBlau}, % Formattierung Strings
	numberstyle=\tiny, % Textgröße der Zeilennummern
	stepnumber=1, % Angezeigte Zeilennummern
	numbersep=5pt, % Abstand zw. Zeilennummern und Code
	aboveskip=15pt, % Abstand oberhalb des Codes
	belowskip=11pt, % Abstand unterhalb des Codes
	captionpos=b, % Position der Überschrift
	xleftmargin=10pt, % Linke Einrückung
	frame=single, % Rahmentyp
	breaklines=true, % Umbruch langer Zeilen
	showstringspaces=false, % Spezielles Zeichen für Leerzeichen
	tabsize=2,
	texcl=true
}

%%%%%%%%
% Kopf
%%%%%%%%
\newcommand{\Header}[3]
{
	{\footnotesize \parindent0em
		{\sc Universität Konstanz}                \hfill #1 \\
		{\sc Fachbereich Informatik \& Informationswissenschaft} \hfill #2 \\
		#3 \hfill \today
	}
}

%%%%%%%%%%%%%%%%%%%%%%%
% load some java code
% \loadJava{file}
%%%%%%%%%%%%%%%%%%%%%%%
\newcommand{\loadJava}[1]
{
	\lstinputlisting[language=Java]{#1.java}
}

%%%%%%%%%%%%%%%%%%%%%%%
% load some cpp code
% \loadCpp{file.cpp}
%%%%%%%%%%%%%%%%%%%%%%%
\newcommand{\loadCpp}[1]
{
	\lstinputlisting[language=C++]{#1}
}

%%%%%%%%%%%%%%%%%%%%%%%%%%%%%
% load some code
% \loadCode{Python}{file.py}
%%%%%%%%%%%%%%%%%%%%%%%%%%%%%
\newcommand{\loadCode}[2]
{
	\lstinputlisting[language=#1]{#2}
}

%%%%%%%%%%%%%%%%%%%%%%%%%%%%%%%%%%%%%%%%%%%%%%%%%%%%%%%%%%%%%%%%%%%%%%%
% some symbols
%%%%%%%%%%%%%%%%%%%%%%%%%%%%%%%%%%%%%%%%%%%%%%%%%%%%%%%%%%%%%%%%%%%%%%%
\newcommand{\correct}{\gruen{\text{\ding{52}}}} %for use in text and math
\newcommand{\wrong}{\rot{\text{\ding{56}}}} %for use in text
\newcommand{\tflash}{$\hellgelb{\lightning}$} %for use in text
\newcommand{\mflash}{\hellgelb{\lightning}} %for use in math
\newcommand{\follows}{$\Rightarrow$} %used so often...

%%%%%%%%%%%%%%%%%%%%%%%%%%%%%%%%%%%%%%
% languages for \loadCode
%ABAP		IDL				Plasm
%ACSL		inform			POV
%Ada		Java			Prolog
%Algol		JVMIS			Promela
%Ant		ksh				Python
%Assembler	Lisp			R
%Awk		Logo			Reduce
%bash		make			Rexx
%Basic		Mathematica1	RSL
%C			Matlab			Ruby
%C++		Mercury			S
%Caml		MetaPost		SAS
%Clean		Miranda			Scilab
%Cobol		Mizar			sh
%Comal		ML				SHELXL
%csh		Modula-2		Simula
%Delphi		MuPAD			SQL
%Eiffel		NASTRAN			tcl
%Elan		Oberon-2		TeX
%erlang		OCL				VBScript
%Euphoria	Octave			Verilog
%Fortran	Oz				VHDL
%GCL		Pascal			VRML
%Gnuplot	Perl			XML
%Haskell	PHP				XSLT
%HTML		PL/I
%%%%%%%%%%%%%%%%%%%%%%%%%%%%%%%%%%%%%%

\usetikzlibrary{trees}

\begin{document}

\Gruppe{Stephan Heidinger}{SEng - Zusammenfassung v0.1}
\Header{Software Engineering}{Wintersemester 2011/2012}{Stephan Heidinger}

\begin{shaded}
Dieses Dokument wurde unter der Creative Commons - Namensnennung-NichtKommerziell-Weitergabe unter gleichen Bedingungen (\textbf{CC by-nc-sa}) veröffentlicht. Die Bedingungen finden sich unter \href{http://creativecommons.org/licenses/by-nc-sa/3.0/de}{diesem Link}. \\
\centerline{\includegraphics[scale=1]{../cc-by-nc-sa.png} }
\end{shaded}

\textit{Find any errors? Please send them back, I want to keep them!}

\begin{shaded}
\textbf{Definitions for Software Engineering} \\
``Multi-person construction of multi-version software'' (D. Parnas, 1987)\\
``The application of a systematic, disciplined, quantifiable approach to the development, operation, and maintenance of software; that is, the application of engineering to software.''
\end{shaded}

\section*{Software Crisis}
Term from NATO study group in 1967.
\begin{description}
	\item[fault] mistake/error made by a human during a software activity (erroneous design, requirements, coding)
	\item[failure] observed departure of a system from the desired state
\end{description}
\subsection*{symptoms}
\begin{itemize}
	\item products are delivered late \follows high cost
	\item projects exceed budgeds \follows high cost \follows waste of resources
	\item product doesn't do, whats its supposed to do \follows inefficient \follows hight cost
	\item products are defective \follows high cost (failures, maintenance,\dots), ethical considerations
	\item projects get abandonded before delivery \follows waste of resources
\end{itemize}

\subsection*{Characteristics of software}
\begin{itemize}
	\item Software is \emph{engineered}, not manufactured (custom built, little to no (mechanical) assembly, human intensive process)
	\item Software does not wear out (but: change of requirements, changes in environment)
	\item software is \emph{complex}
	\item software is \emph{detemining system factor} (up to 80\% of development effort)
\end{itemize}

\subsection*{Why is software difficult to produce?}
\begin{itemize}
	\item no similiar system built so far (problems unknown, assumptions about environment may be wrong)
	\item \emph{Requirements} are not well understood/phrased
	\item Requirements \emph{change} during development
	\item complex \emph{interaction}
	\item nature of systems:
		\begin{itemize}
			\item concurrent systems (races, deadlocks, \dots)
			\item embedded systems (hardware interaction, timing, \dots)
			\item information systems (complexity, legacy, \dots)
		\end{itemize}
	\item software is \emph{easily changeble} \follows ``code and fix''
	\item software is \emph{discreet} (either fails, or doesn't)\marginnote{Stimmt auch nicht immer\dots}
\end{itemize}

\subsection*{Software Debvelopment Myths}
\begin{itemize}
	\item Management
		\begin{itemize}
			\item standard books, software, tools, \dots \follows software hard to standardize
			\item behind schedule? add programmers! \follows effort to integrate new people
		\end{itemize}
	\item Customer
		\begin{itemize}
			\item general statement of objectives sufficient
			\item change (as in requirements) can easily be implemented
		\end{itemize}
	\item Pracitioner
		\begin{itemize}
			\item once programm is running, job is done
			\item until programm is running, no way to check quality
			\item only deliverable is working programm
		\end{itemize}
\end{itemize}

\subsection*{Software Engineering}
Communication between a huge number of parties (customer, end user, sw designer, developer, \dots) must be maintained. \\
Highly complex systems: SE has to provide solutions.

\begin{description}
	\item[\follows] base sw production on engineering approach
	\item[\follows] design sw as a \emph{process} that supports
		\begin{itemize}
			\item correctness \& dependability
			\item cost-effectivenes
			\item complexity of project
			\item longevity of product, life cycle, changes
			\item communication amongst parties
		\end{itemize}
	\item[\follows] software engineering process model
\end{description}

\section*{waterfall model}
\begin{tikzpicture}[node distance=50]

\node[draw] (1) at (0,0) {System Design};
\node[draw] (2) [below right of = 1] {Requirements};
\node[draw] (3) [below right of = 2] {Design};
\node[draw] (4) [below right of = 3] {Implementation};
\node[draw] (5) [below right of = 4] {Integration};
\node[draw] (6) [below right of = 5] {Maintenance};

\node (c1) [right of = 1] {};
\node (c2) [right of = 2] {};
\node (c3) [right of = 3] {};
\node (c4) [right of = 4] {};
\node (c5) [right of = 5] {};
\node (c6) [right of = 6] {};

\node (1c) [left of = 1] {};
\node (2c) [left of = 2] {};
\node (3c) [left of = 3] {};
\node (4c) [left of = 4] {};
\node (5c) [left of = 5] {};
\node (6c) [left of = 6] {};

\path[draw,->] (1) .. controls (c1) .. (2);
\path[draw,->] (2) .. controls (c2) .. (3);
\path[draw,->] (3) .. controls (c3) .. (4);
\path[draw,->] (4) .. controls (c4) .. (5);
\path[draw,->] (5) .. controls (c5) .. (6);

\node[node distance=10] (a1) [right of = c1] {};
\node[node distance=10] (a2) [right of = c2] {\textsc{Srs}};
\node[node distance=10] (a3) [right of = c3] {Design};
\node[node distance=10] (a4) [right of = c4] {\textcolor{gray}{Test}};
\node[node distance=10] (a5) [right of = c5] {\textcolor{gray}{Test}};
\node[node distance=10] (a6) [right of = c6] {};

\path[draw,dashed,color=gray,->] (2) .. controls (2c) .. (1);
\path[draw,dashed,color=gray,->] (3) .. controls (3c) .. (2);
\path[draw,dashed,color=gray,->] (4) .. controls (4c) .. (3);
\path[draw,dashed,color=gray,->] (5) .. controls (5c) .. (4);
\path[draw,dashed,color=gray,->] (6) .. controls (6c) .. (5);

\node[node distance=20] (1a) [left of = 1c] {};
\node[node distance=20] (2a) [left of = 2c] {\textcolor{red}{``what''}};
\node[node distance=20] (3a) [left of = 3c] {\textcolor{red}{``how''}};
\node[node distance=20] (4a) [left of = 4c] {};
\node[node distance=20] (5a) [left of = 5c] {};
\node[node distance=20] (6a) [left of = 6c] {};

\end{tikzpicture}

\begin{description}
	\item[System Design] problem \& System requirements, fesability study, define main subsystems (allocate to hw/sw), system design document (\textsc{Sdd}, informal, with customer, somtimes: user manuals, user interfaces, test plans)
	\item[Requirements] requirement analysis (``what'' the system has to do (not how), software requirements describing observable external behaviour (functional, nonfunctional)), software requirement specification (\textsc{Srs})
	\item[Design] ``how'' the system is working, architectural (high level) design (decompose problem into components, global data structures, internal interfaces), detailed design (algorithmic design, internal data structures, programming language(s))
	\item[Implementation] (and Testing) \follows translate design modules into code, test modules in isolation
	\item[Integration] (and Testing) integrate tested modules to form system, \emph{integration testing}, validation, customer accptance tests
	\item[maintenance] deploy product, perform maintenance (corrective, adaptive, perfective), retire product
\end{description}
\subsubsection*{Benefits}
\begin{itemize}
	\item Definition of seperate Tasks (Seperation of concerns)\\dividing complex design problems into smaller units \follows facilitates team work/reproducibility
	\item specification and documentation  \follows enforces documentation, facilitates testing (against requirements specification)
	\item further concepts
		\begin{itemize}
			\item verification / validation (compare intermediate results with requirements and design)
			\item prototyping (mock-up available early, reduces risk)
			\item evolutionary process model (accommodate change)
		\end{itemize}
\end{itemize}

\subsubsection*{Defects}
\begin{itemize}
	\item no feedback loops
	\item document driven, inflexible
	\item large time gap between inception and completion
\end{itemize}

\subsubsection*{Alternatives}
\begin{itemize}
	\item modified waterfall model
	\item V-Model, V-Model XT
	\item Evolutionary Process models
	\item Spiral model
	\item Rational Unified Process
	\item Agile Processes
	\item \dots
\end{itemize}

\subsection*{Software Qualities}
\begin{description}
	\item[Correctness:] the software behaves according to the \emph{requirement specification}
	\item[Reliability:] the software guarantees a level of quality service ($<$Correctness)
	\item[Robustness:] the software behaves ``reasonably'' even in unexprected circumstances (e.g. input, power failure, \dots)
	\item[Maintainability:] software is easy to maintain/extend
	\item[Performance:] system is usable (e.g. input response)
	\item[Reusability:] re-use of previously used, tested and verified code
	\item[Interoperability:] standardisation, interfaces, \dots
\end{description}

\subsection*{Software Engineering Principles}
\begin{description}
	\item[Rigor:] use a method and apply it rigorously to every step
	\item[Formality:] use (mathematical) formalizable methods and notations
	\item[Speration of Concerns:] deal with different aspects in seperate steps
	\item[Abstraction:] seperate the concern of the important aspects from the concern of unimportant ones
	\item[Modularity:] decompose problem into independent modules \follows separation of concern
	\item[Generality:] focus on the discovery of more general problems
\end{description}

\section*{Requiremnt \& Analysis}

\subsection*{Requirement Elicitation}
\begin{shaded}
``A \textbf{requirement} is a condition or capability that must be met or professed by a system component to satisfy a contract, standard, specification or other formally inposed document.'' \\
(ANSI/IEEE Standard 729-1983)
\end{shaded}
requirements \follows description of the \emph{externally visible (interface) \marginnote{not environment behaviour}behaviour}, the ``what''\\
Interaction between the system and the system-relevant portion of the envrionment

\begin{description}
	\item[User (Client) requirements:] statements in natural language plus diagramms
	\item[System requirements:] structured document, detailed description of system's functions, services and operational constraints; may be part of contract between client and contractor
\end{description}
\subsubsection{Types of requirements}
\begin{description}
	\item[functional requirements:] high-level-``what'' the system should do
		\begin{itemize}
			\item statement which services the system should provide
			\item interaction between system and environment (system states, i/o)
			\item how the system should behave in particular situations
			\item describe the system services in detail
			\item often included with ``shall/should''
		\end{itemize}
	\item[nonfunctional requirements:] (if not met, system may be useless (not usable)) \\
	nonfunctional requirements often have quantitative measures (MB, training time, availability, \dots)\\
	\begin{tikzpicture}[
	dist0/.style={},
	dist1/.style={sibling distance=11em},
	dist2s/.style={sibling distance=5.2em},
	dist2b/.style={sibling distance=6.4em},
	down/.style={level distance=7ex},
	downB/.style={level distance=20ex},
	level0/.style={draw,rectangle,fill=blue!20,rounded corners=.8ex},
	level1/.style={draw,rectangle,fill=green!20,rounded corners=.8ex},
	level2/.style={draw,rectangle,fill=yellow!20,rounded corners=.8ex},
	level3/.style={draw,rectangle,fill=gray!20,rounded corners=.8ex},
]

	\coordinate
	child[grow=down,anchor=south, level distance=0ex] {node[level0] {non-functional}}
	[edge from parent fork down]
		child[dist1,down] {node[level1] {product}
			child[dist2s,down] {node[level2] {usability}
			}
			child[dist2s,down] {node[level2] {efficiency}
				child[dist2s,down] {node[level3] {performance}}
				child[dist2s,down] {node[level3] {space}}
			}
			child[dist2s,down] {node[level2] {reliability}
			}
			child[dist2s,down] {node[level2] {portability}
			}
		}
		child[dist1,down] {node[level1] {organisational}
			child[dist2b,downB] {node[level2] {delivery}
			}
			child[dist2b,downB] {node[level2] {implementation}
			}
			child[dist2b,downB] {node[level2] {standards}
			}
		}
		child[dist1,down] {node[level1] {external}
			child[dist2b,down] {node[level2] {interoperability}
			}
			child[dist2s,down] {node[level2] {ethical}
			}
			child[dist2s,down] {node[level2] {legislative}
				child[dist2s,down] {node[level3] {privacy}}
				child[dist2s,down] {node[level3] {safety}}
			}
		}
;
\end{tikzpicture}
		\begin{itemize}
			\item not primarly related to sytem's funcions/services, but to quality and additinal (quantitative) characteristics
			\item constraints on the services/functions:
			\begin{itemize}
				\item reliability (avalability, integrity, security, safety)
				\item accuracy of results
				\item performance / timing
				\item human-computer interface issues
				\item operating and physical constraints
				\item portability and interoperatbility
				\item reliability
				\item response time
				\item storage requirements
				\item standards \dots
				\item also: particular system, programming language or development method
			\end{itemize}
			\item product requirements (product must behave in particular way (execution time, speed, reliability, \dots))
			\item organisational requirements (standards used, impelementation requirements, \dots)
			\item external requirements (interoperatbility/legislative requirements, \dots)
		\end{itemize}
		\item[domain requirements:] (if not met, system may be unworkable in that domain)
			\begin{itemize}
				\item derived from application domain
				\item characteristics/features of domain
				\item constraints on existing requirements
				\item define specific computations
				\item may be expressed in domain specific language \follows often hard to understand
				\item often implicit \follows hard to aquire
			\end{itemize}
\end{description}


\subsubsection*{Requirement Specification Properties}
\begin{description}
	\item[correctness:] facts in requirement specification \follows required properties of the system
	\item[unambiguity:] all specifications have a single interpretation
	\item[completeness:] 
		\begin{enumerate}
			\item every property required of the system is expressed in the specification\\
			{\small does this include things that are \emph{not permitted?}}
			\item the responses of the software system on all types of possible input values are specified
			\item \dots {\small (more possible)}
		\end{enumerate}
	\item[verifiability:] there exists an effective (manual/automatic) process for checking wether a software product satisfies the required properties (formal verification (mathematical proof) or validation (model checking, testing, simulation)), often not possible for every requirement
	\item[consistency:] no two requirements contradict each other: never $A \wedge \neg A$
	\item[traced:] origin of every requirement is clear
	\item[traceable:] requirement specification is edited so it is easy to reference every single requirement (e.g. numbering)
	\item[design independent:] the requirement specification does not require specific software/architecture/algorithms
\end{description}

\subsubsection{Activities during the requirement Stage}
	\begin{description}
		\item[starting point] 
			\begin{itemize}
				\item customer requirements (abstract)
				\item system specification document (hw \& sw)
			\end{itemize}
		\item[Activities (Kotonya+Sommerville)]
			\begin{itemize}
				\item requirement elicitation\marginnote{Elicitation \follows Herausholen} (interviews, scenarios, market observation, \dots) \\
				determine, which of possibly contradictory requirements are important
				\item requirement documentation and specification (comprehensible requirements document)
				\item requirement validation (consistency, completeness, correspondence of documented requirements and abstract customer or user requirements)
			\end{itemize}
			\begin{tikzpicture}[%
	>=stealth,
	node distance=3.5cm,
	on grid,
	auto,
	io/.style={draw, text width=2.5cm, align=center},
	proc/.style={draw, rounded corners=1.2em, text width=2.5cm, align=center},
	dist/.style={node distance = 3cm},
	dick0/.style={draw,line width=1.5pt},
	dick/.style={dick0,->},
	duenn/.style={draw,line width=1.5pt,dashed,->}
]

\node[io] (root) at (0,0) {customer or user requirements (abstract)};

\node[proc] (2) [below left of  = root] {requirements analysis and negotiation};
\node[proc] (1) [left of       = 2   ] {requirements elicitation};
\node[proc] (3) [below right of = root] {requirements documentation and specification};
\node[proc] (4) [right of      = 3   ] {requirements validation};

\node[io,node distance=2cm] (leaf) [below of = 3] {negotiated and validated requirements};

\node[node distance=1.5cm] (phantom1) [below left of =root] {};
\node[node distance=1.5cm] (phantom2) [below right of =root] {};

\path[dick] (root) -| (1);
\path[dick] (1) -- (2);
\path[dick] (2) -- (3);
\path[dick] (3) -- (4);
\path[dick] (3) -- (leaf);
\path[duenn] (root) -| (4);

\path[dick] (phantom2) -| (1.north east);
\path[dick] (phantom2) -| (2);
\path[dick] (phantom1) -| (3);
\path[dick0] (phantom1) -| (4.north west);


%	\node[state] (A)              {A};
%	\node        (B) [right of=A,fill=blue!25,text width=3cm]{This is a demonstration text for showing how line breaking works.};;
%	\path[->] (A) edge (B);


\end{tikzpicture}
			\item[social activity:]
			Not a single person knows everything about the system \follows communication is needed \follows difficult (technical language, ambiguities, customers not aware of needs, personalities.)
	\end{description}

	% slide 2-43

\subsection*{Object Oriented Analysis}

\subsection*{Analysis with the UML}

\subsection*{Document Requirements}

\end{document}
