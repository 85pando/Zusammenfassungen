\documentclass[a4paper, 10pt]{article}

%\usepackage{scalefnt}
%\usepackage{parcolumns}

\usepackage{newclude}
% Zum Einbinden in die Zusammenfassungs-files

\usepackage{amsmath,amsthm,amsfonts,amssymb} % Verbesserter Mathesatz
\usepackage{bigfoot} % komplexe Fußnotenapparate(Fußnoten in Fußnoten und andere Späße)%
\usepackage{colortbl}
\usepackage[T1]{fontenc} % normaler erweitere Zeichnesatz
\usepackage{framed, color}  % Ramenpaket für zum Einfügen von schönen Ramen
\usepackage{graphicx}
\usepackage{hyperref} %used for link to creative commons license
\usepackage{listings} %for code-listings (inkl. Tab-Styling)
\usepackage{marginnote}
\usepackage{microtype} % div. Verbesserungen des Schriftsatzes (Grauwert, opt. Randausgleich, Zeilenumbruch)
%\usepackage{multirow}
\usepackage[ngerman]{babel} % Neue Rechtschreibung
\usepackage[sans]{dsfont} %für alternative Mengensymbole
\usepackage{stmaryrd} %u.a. für \lightning
\usepackage{tikz,tikz-er2,tikz-uml} % für Diagramme(Dia!) und Bilder (z.B. *.eps)/für ER-Diagramme/für UML-Diagramme
\usepackage{units} % z.B. fuer \nicefrac{}{}
\usepackage[utf8]{inputenc} % utf8 für den Editor
\usepackage{wasysym} %u.a. für \lightning
\usepackage{xcolor}


\usetikzlibrary{shapes,decorations,arrows,fit,backgrounds} %Zum diversen zeichen%
%\usetikzlibrary{automata} % für den CFlipper, wenn es den soweit ist			%
\usetikzlibrary{positioning} % positionierung 
%\usetikzlibrary{shadows} % fuer schoene schlagschatten

%%%%%%%%%%%%%%%%%%%%%%%%%%%
%  Formatierung der Seite
%%%%%%%%%%%%%%%%%%%%%%%%%%%
\usepackage{fancyhdr}
\pagestyle{fancy}		% für den footer										%
\renewcommand{\headrulewidth}{0pt} % damit oben kein dummer Strich kommt		%
\fancyhead{}
\topmargin -2cm 		% Oberer Rand											%
\textheight 25cm		% Texthöhe												%
\textwidth 16.0 cm		% Textbreite											%
\oddsidemargin -0.1cm 	% Warum?												%
\newcommand{\Gruppe}[2]
{
	\lfoot{#1}
	\rfoot{#2}
}
\colorlet{shadecolor}{gray!25} % Farbe für graue Box definieren
%%%%%%%%%%%%%%%%%%%%%%%%%%%%%%%%%%%%%%%%%%%%%%%%%%%%%%%%%%%%%%%%%%%%%%%%%%%%%%%%%
%Farben die definiert werden zum schreiben und zeichnen							%
%%%%%%%%%%%%%%%%%%%%%%%%%%%%%%%%%%%%%%%%%%%%%%%%%%%%%%%%%%%%%%%%%%%%%%%%%%%%%%%%%
\xdefinecolor{schwarz}{HTML}{000000}
\xdefinecolor{dunkelGruen}{HTML}{007D00}
\xdefinecolor{dunkelBlau}{HTML}{0000A0}
\xdefinecolor{dunkelRot}{HTML}{A00000}
\xdefinecolor{dunkelGelb}{HTML}{FFAA00}
\xdefinecolor{hellesGelb}{HTML}{FFCC00}
\colorlet{dGreen}{dunkelGruen}
\colorlet{dBlue}{dunkelBlau}
\colorlet{dRed}{dunkelRot}
\colorlet{dYellow}{dunkelGelb}

%%%%%%%%%%%%%%%%%%%%%%%%%%%%%%%%%%%%%%%%%%%%%%%%%%
%Farbliche Ausgaben:
%Parameter #1: Text oder Mathematische formel...
%z.B. : \gruen{Hallo Welt Test!}
%%%%%%%%%%%%%%%%%%%%%%%%%%%%%%%%%%%%%%%%%%%%%%%%%%

\newcommand{\yellow}[1]{\textcolor{dYellow}{#1}}
\newcommand{\gray}[1]{\textcolor{gray}{#1}}
\newcommand{\red}[1]{\textcolor{red}{#1}}
\newcommand{\green}[1]{\textcolor{green}{#1}}
\newcommand{\blue}[1]{\textcolor{blue}{#1}}
\newcommand{\dGreen}[1]{\textcolor{dGreen}{#1}}
\newcommand{\dBlue}[1]{\textcolor{dBlue}{#1}}
\newcommand{\dRed}[1]{\textcolor{dRed}{#1}}

%%%%%%%%%%%%%%%%%%%%%%%%%%%%%%%%%%%%%%%%%%%%%%%%
%Konfiguration für das darstellen von Quelltext
%%%%%%%%%%%%%%%%%%%%%%%%%%%%%%%%%%%%%%%%%%%%%%%%
\lstset
{
	language=Java, % oder C++, Pascal, {[77]Fortran}, ...
	numbers=left, % Position der Zeilennummerierung
	firstnumber=auto, % Erste Zeilennummer
	basicstyle=\ttfamily, % Textgröße des Standardtexts
	keywordstyle=\ttfamily\color{dRot}, % Formattierung Schlüsselwörter
	commentstyle=\ttfamily\color{dGruen}, % Formattierung Kommentar
	stringstyle=\ttfamily\color{dBlau}, % Formattierung Strings
	numberstyle=\tiny, % Textgröße der Zeilennummern
	stepnumber=1, % Angezeigte Zeilennummern
	numbersep=5pt, % Abstand zw. Zeilennummern und Code
	aboveskip=15pt, % Abstand oberhalb des Codes
	belowskip=11pt, % Abstand unterhalb des Codes
	captionpos=b, % Position der Überschrift
	xleftmargin=10pt, % Linke Einrückung
	frame=single, % Rahmentyp
	breaklines=true, % Umbruch langer Zeilen
	showstringspaces=false, % Spezielles Zeichen für Leerzeichen
	tabsize=2,
	texcl=true
}

%%%%%%%%
% Kopf
%%%%%%%%
\newcommand{\Header}[3]
{
	{\footnotesize \parindent0em
		{\sc Universität Konstanz}                \hfill #1 \\
		{\sc Fachbereich Informatik \& Informationswissenschaft} \hfill #2 \\
		#3 \hfill \today
	}
}

%%%%%%%%%%%%%%%%%%%%%%%
% load some java code
% \loadJava{file}
%%%%%%%%%%%%%%%%%%%%%%%
\newcommand{\loadJava}[1]
{
	\lstinputlisting[language=Java]{#1.java}
}

%%%%%%%%%%%%%%%%%%%%%%%
% load some cpp code
% \loadCpp{file.cpp}
%%%%%%%%%%%%%%%%%%%%%%%
\newcommand{\loadCpp}[1]
{
	\lstinputlisting[language=C++]{#1}
}

%%%%%%%%%%%%%%%%%%%%%%%%%%%%%
% load some code
% \loadCode{Python}{file.py}
%%%%%%%%%%%%%%%%%%%%%%%%%%%%%
\newcommand{\loadCode}[2]
{
	\lstinputlisting[language=#1]{#2}
}

%%%%%%%%%%%%%%%%%%%%%%%%%%%%%%%%%%%%%%%%%%%%%%%%%%%%%%%%%%%%%%%%%%%%%%%
% some symbols
%%%%%%%%%%%%%%%%%%%%%%%%%%%%%%%%%%%%%%%%%%%%%%%%%%%%%%%%%%%%%%%%%%%%%%%
\newcommand{\correct}{\gruen{\text{\ding{52}}}} %for use in text and math
\newcommand{\wrong}{\rot{\text{\ding{56}}}} %for use in text
\newcommand{\tflash}{$\hellgelb{\lightning}$} %for use in text
\newcommand{\mflash}{\hellgelb{\lightning}} %for use in math
\newcommand{\follows}{$\Rightarrow$} %used so often...

%%%%%%%%%%%%%%%%%%%%%%%%%%%%%%%%%%%%%%
% languages for \loadCode
%ABAP		IDL				Plasm
%ACSL		inform			POV
%Ada		Java			Prolog
%Algol		JVMIS			Promela
%Ant		ksh				Python
%Assembler	Lisp			R
%Awk		Logo			Reduce
%bash		make			Rexx
%Basic		Mathematica1	RSL
%C			Matlab			Ruby
%C++		Mercury			S
%Caml		MetaPost		SAS
%Clean		Miranda			Scilab
%Cobol		Mizar			sh
%Comal		ML				SHELXL
%csh		Modula-2		Simula
%Delphi		MuPAD			SQL
%Eiffel		NASTRAN			tcl
%Elan		Oberon-2		TeX
%erlang		OCL				VBScript
%Euphoria	Octave			Verilog
%Fortran	Oz				VHDL
%GCL		Pascal			VRML
%Gnuplot	Perl			XML
%Haskell	PHP				XSLT
%HTML		PL/I
%%%%%%%%%%%%%%%%%%%%%%%%%%%%%%%%%%%%%%


% \newmdenv[%
% roundcorner=5pt,
% subtitlebelowline=true,subtitleaboveline=true,
% subtitlebackgroundcolor=yellow!70!white,
% backgroundcolor=blue!20!white,
% frametitle={Theorem},frametitlerule=true,
% frametitlebackgroundcolor=yellow!70!white,
% ]{subtitleenv}

\newcommand{\until}{\,\mathcal{U}\,}
\newcommand{\wuntil}{\,\mathcal{W}\,}
\renewcommand{\O}{\mathcal{O}}

\begin{document}

\Gruppe{Stephan Heidinger}{ADMC - Zusammenfassung v0.1}
\Header{advanced Model Checking}{SS 14}{Stephan Heidinger}

% \begin{shaded}
% This document has been published under a Creative Commons - Attribution-NonCommercial-ShareAlike (\textbf{CC by-nc-sa}). The conditions of the licence can be found \href{http://creativecommons.org/licenses/by-nc-sa/3.0/}{here}. \\
% \centerline{\href{http://creativecommons.org/licenses/by-nc-sa/3.0/}{\includegraphics[scale=1]{../cc-by-nc-sa.png} } }
% \end{shaded}

\textit{\ensuremath{\overset{-\mkern-11mu-\mkern-3.5mu\rhook}{\smash{\odot}\rule{0ex}{.46ex}}\underline{\hspace{0.5em}}\overset{-\mkern-11mu-\mkern-3.5mu\rhook}{\smash{\odot}\rule{0ex}{.46ex}}}
Find any errors? Please send them back, I want to keep them!}

\section{Preliminaries}

\begin{definition}
Some things probably important:
\mdfsubtitle{Transformational Systems}
Transform set of input data into output data: $S_i\to S_k$. E.g. Compilers, database processing. \\
Correctness criteria: Termination, Correctness of $S_i\to S_k$
\mdfsubtitle{Reactive Systems}
Ongoing interaction with environment, driven by events/stimuli. E.g. Operating Systems, Control Systems.\\
Correctness: non-termination, correctness of stimuli-response pair.
\mdfsubtitle{Embedded Systems}
Usually reactive systems, tightly connected to the hardware they control.
\mdfsubtitle{Cyber-Physical systems}
Integration of computation and physical processes, often networked, e.g. sensor-/actuator systems, automotive control systems.
\mdfsubtitle{Real-Time Systems}
Correctness depends on time bounds:
\begin{description}
    \item[soft:] violating soft time bounds will decrease quality of system.
    \item[hard:] violating hard time bounds will make the system fail.
\end{description}
\mdfsubtitle{Hybrid systems}
Systems characterized by discrete and continuous variables. E.g. thermostat, \dots
\mdfsubtitle{Fault}
Mistake made by a human during software development/production.
\mdfsubtitle{Failure}
Behaviour of a system deviating from its specified behaviour. This is most often the result of a fault being executed.
\mdfsubtitle{Safety-Critical Systems}
When a safety-critical system fails, people, the environment or damage to property or assets may occur.
\end{definition}

\subsection{System Correctness}
When is a system correct?
\begin{enumerate}
    \item It does what we expect it to do. \follows functional model checking
    \item It does so in a timely manner. \follows real time or probabilistic model checking.
    \item It does so with a certain probability over a certain period of time. \follows probabilistic model checking
\end{enumerate}

\begin{minipage}[t]{3cm}
\begin{center}
\scalebox{1}{\begin{tikzpicture}[
]

\draw[fill=red] (0,0) circle (1.5);
\draw[fill=blue!70] (0,0) circle (1);

\node[] at (1.25,0) {$\mathcal{S}$};
\node[] at (0,0) {$\mathcal{M}$};


\end{tikzpicture}}
\end{center}
\end{minipage}
\begin{minipage}[t]{12cm}
\begin{itemize}
    \item Given a \blue{model} and a \red{specification}: \\ Does $\blue{M}\models\red{S}$?
    \item When every behaviour of \blue{M} is also behaviour of \red{M} this is the case. \follows Model does not reveal properties violating the specification.
    \item Model of course has to represent the behaviour of the system.
\end{itemize}
\end{minipage}

\section{CTL and CTL Model Checking}
\begin{mdframed}[roundcorner=5pt,
subtitlebelowline=false,subtitleaboveline=false,
subtitlebackgroundcolor=blue!50,
% backgroundcolor=blue!20,
frametitlerule=true,
frametitlebackgroundcolor=blue!50,
frametitle={State}
]
Characterizes the salient features of a system at a given point of observation. \follows A state can be observed as long as the features of interest don't change.
\end{mdframed}

\subsection{State-Based Modelling}
\begin{mdframed}[roundcorner=5pt,
subtitlebelowline=false,subtitleaboveline=false,
subtitlebackgroundcolor=blue!50,
% backgroundcolor=blue!20,
frametitlerule=true,
frametitlebackgroundcolor=blue!50,
frametitle={State Transition in Discrete Systems}
]
Instantaneous change of observed features of systems. Represents computation step.
\mdfsubtitle{real-time models}
time passes in a state \& state must be left when time-bound is reached
\mdfsubtitle{stochastic systems}
state transitions are labeled with probabilities
\mdfsubtitle{hybrid systems}
\begin{itemize}
    \item continuous state variables change in a state
    \item discrete state variables change during state transition
\end{itemize}
\end{mdframed}

State transitions:
\begin{itemize}
    \item In a given state a certain number of events are possible (Leading to several different successor states).
    \item Represent valid sequence of computations.
    \item They encode history information (a state can only be reached trough a series of transitions).
\end{itemize}

Guidelines:
\begin{description}
    \item[Abstraction:] Focus only on important facts, disregard the rest.
    \item[Simplicity:] Find simplest abstraction that still reveals phenomena of interest.
\end{description}

Characterization of reactive systems:
\begin{itemize}
    \item State of the system.
    \item State transitions (caused by events/stimuli).
    \item Reactions triggered by transitions
\end{itemize}

\subsection{Transition Systems}

\begin{mdframed}[roundcorner=5pt,
subtitlebelowline=false,subtitleaboveline=false,
subtitlebackgroundcolor=blue!50,
% backgroundcolor=blue!20,
frametitlerule=true,
frametitlebackgroundcolor=blue!50,
frametitle={Transition System}
]
A \emph{Transition System} TS is a tuple $(S,Act,\rightarrow, I, AP, L)$ where:\\
\begin{tabular}{ll}
$S$ & set of states\\
$Act$ & set of actions\\
$\rightarrow\subseteq S\times Act\times S$ & transition relation\\
$I\subseteq S$ & set of initial states\\
$AP$ & atomic propositions\\
$L:S\rightarrow 2^{AP}$ & labeling function\\
\end{tabular}
\begin{itemize}
    \item $(s,\alpha.s')$ can be written as $s\rightarrow^\alpha s'$ or $s\overset{\alpha}{\longrightarrow}s'$.
\end{itemize}
\end{mdframed}
\begin{center}
\scalebox{1.5}{\begin{tikzpicture}[
state/.style={draw,rounded corners,node distance=60},
arrow/.style={draw,->,>=latex},
label/.style={black!50},
]
\node[state] (idle) at (0,0) {$idle$};
\node[state,above right of = idle] (orig) {$orig$};
\node[state,below right of = idle] (term) {$term$};
\node[state,below right of = orig] (con) {$con$};

\path[arrow]
(idle) edge node[above left,label] {off-hook} (orig)
(idle) edge node[below left,label] {ringing}(term)
(orig) edge node[above right,label] {digits}(con)
(term) edge node[below right,label] {off-hook} (con)
(con)  edge node[above,label] {on-hook} node [below,label] {end-of-call} (idle)
;

\end{tikzpicture}}
\end{center}

Atomic Propositions:
\begin{itemize}
    \item logical representation of facts that may hold in a given state.
    \item AP set of all atomic propositions used in the system model.
\end{itemize}

Labeling functions:
\begin{itemize}
    \item which atomic propositions actually hold in a given state.
\end{itemize}

\begin{mdframed}[roundcorner=5pt,
subtitlebelowline=false,subtitleaboveline=false,
subtitlebackgroundcolor=blue!50,
% backgroundcolor=blue!20,
frametitlerule=true,
frametitlebackgroundcolor=blue!50,
frametitle={Predecessors and Successors}
]
\begin{align*}
Post(s,\alpha)&= \left\{ s'\in S | s\overset{\alpha}{\longrightarrow}s' \right\} &
Post(s) &= \bigcup_{\alpha\in Act} Post(s\alpha)
\\
Pre(s,\alpha)&= \left\{ s'\in S | s'\overset{\alpha}{\longrightarrow}s \right\} &
Pre(s) &= \bigcup_{\alpha\in Act} Pre(s\alpha)
\\
Post(C,\alpha) &= \bigcup_{s\in C} Post(s,\alpha) &
Post(C) &= \bigcup_{s\in C} Post(s) \text{ for } C\subseteq S
\\
Pre(C,\alpha) &= \bigcup_{s\in C} Pre(s,\alpha) &
Pre(C) &= \bigcup_{s\in C} Pre(s) \text{ for } C\subseteq S
\end{align*}
\mdfsubtitle{Terminal/Final State}
a state for which $Post(s)=\varnothing$
\mdfsubtitle{Action-Determinism}
A TS is \emph{action-deterministic}, iff for all $s,\alpha$
\begin{itemize}
    \item $|I|\leq 1$
    \item $|Post(s,\alpha)|\leq 1$
\end{itemize}
otherwise it is \emph{action-nondeterministic}. In other words: For every state $s$ and every action $\alpha$ there is at most one outgoing transition labeled with $\alpha$.
\mdfsubtitle{$AP-Determinism$}
A TS is \emph{$AP$-deterministic}, iff for all $s,A\in 2^{AP}$
\begin{itemize}
    \item $|I|\leq 1$
    \item $|Post(s)\cap\{ s'\in S | L(s')=A \} | \leq 1$
\end{itemize}
where $|Post(s)\cap\{ s'\in S | L(s')=A \}$ denotes the set of all equally labeled successors of $s$. In other words: For every state $s$, every successor state has a unique $AP$ labeling.
\end{mdframed}

Nondeterminism can be used to implement abstraction and concurrency.

\subsection{System Executions}

\begin{mdframed}[roundcorner=5pt,
subtitlebelowline=false,subtitleaboveline=false,
subtitlebackgroundcolor=blue!30,
% backgroundcolor=blue!20,
frametitlerule=true,
frametitlebackgroundcolor=blue!30,
frametitle={Finite Execution Fragment}
]
A \emph{finite execution fragment} $\varrho$ of TS is an alternating sequence of states and executions ending with a state:
\[ \varrho=s_0\alpha_1 s_2 \alpha_2\dots\alpha_n s_n \text{ such that } s_i\overset{\alpha_{i+1}}{\longrightarrow} s_{i+1} \text{ for all } 0\leq i<n \]
\mdfsubtitle{infinite execution fragment}
An \emph{infinite execution fragment} $\varrho$ of TS is an alternating sequence of states and executions ending with a state:
\[ \varrho=s_0\alpha_1 s_2 \alpha_2\dots \text{ such that } s_i\overset{\alpha_{i+1}}{\longrightarrow} s_{i+1} \text{ for all } 0\leq i \]
\mdfsubtitle{maximal execution fragment}
An execution fragment, that is
\begin{description}
    \item[either] finite and ending in terminal state
    \item[or] infinite.
\end{description}
\mdfsubtitle{initial execution fragment}
An execution fragment is initial, iff $s_0\in I$.
\mdfsubtitle{Execution}
A initial, maximal execution fragment.
\mdfsubtitle{Reachability}
State $s\in S$ is called \emph{reachable} in a TS, if there exists an initial, finite execution fragment $s_0\alpha_1 s_1\alpha_2\dots\alpha_n s_n$ such that $s_n=s$.\\
$Reach(TS)$ denotes the set of all reachable states in TS.
\mdfsubtitle{State Graph}
The \emph{State Graph} of TS, $G(TS)$, is the directed Graph $(V,E)$ with vertices $V=S$ and edges $E=\{s,s'\}\in S\times S | s'\in Post(s)\}$.
\mdfsubtitle{Transitive Post Hull}
\begin{align*}
Post^*(s) & \text{ is the set of states reachable from } s &
Post^*(C) &= \bigcup_{s\in C} Post^*(s), \text{ for } C\subseteq S \\
Pre^*(s) & \text{ is the set of states from which } s \text{ is reachable} &
Pre^*(C) &= \bigcup_{s\in C} Pre^*(s), \text{ for } C\subseteq S \\
Reach(TS) &= Post^*(I)
\end{align*}
\mdfsubtitle{Path fragments}
A \emph{path fragment} is an exeuction fragment without actions.
\mdfsubtitle{Finite Path fragments}
A \emph{Finite path fragment} $\hat{\pi}$ of $TS$ is a state sequence:
\[ \hat{\pi} = s_0s_1\dots s_n \text{ such that } s_{i+1}\in Post(s_i) \text{ for all } 0\leq i\leq n \text{ where } n\geq0\]
\mdfsubtitle{Infinite Path fragments}
An \emph{Infinite path fragment} $\hat{\pi}$ of $TS$ is a infite state sequence:
\[ \hat{\pi} = s_0s_1\dots \text{ such that } s_{i+1}\in Post(s_i) \text{ for all } i\geq 0 \]
\mdfsubtitle{Path}
A \emph{Path} is a maximal, initial path fragment
\mdfsubtitle{Trace}
When only registering the atomic propositions along execution, this is called a \emph{trace}.
\begin{align*}
\hat{\pi}&=s_0s_1\dots,s_n &
trace(\hat{\pi})&=L(s_0)L(s_1)\dots L(s_n)
\\
Traces(s) &= trace(Paths(s)) &
Traces(TS) &= \bigcup_{s\in I} Traces(s)
\\
Traces_{fin} &= trace(Paths_{fin}(s)) &
Traces_{fin}(s)(TS) &= \bigcup_{s\in I} Traces_{fin}(s)
\end{align*}
\end{mdframed}

\subsection{Structural Operational Semsantics}

Semantics of a program in terms of computation steps defined by transition system: \[ \frac{premise}{conclusion} \]
If the premise holds, the conclusion holds (and can be used to trigger a new inference rule).
Can be recursively applied \follows structural inductive creation.

\begin{mdframed}[roundcorner=5pt,
subtitlebelowline=false,subtitleaboveline=false,
subtitlebackgroundcolor=blue!30,
% backgroundcolor=blue!20,
frametitlerule=true,
frametitlebackgroundcolor=blue!30,
frametitle={Interleaving}
]
\[ TS_1 ||| TS_2 = (S_1\times S_2, Act_1 \uplus Act_2,\longrightarrow, I_1\times I_2, AP_1\uplus AP_2, L) \]
    where $L(\langle s_1,s_2\rangle) = L_1(s_1) \cup L_2(s_2)$ and the transition relation $\longrightarrow$ is defined by:
    \begin{align*}
    \frac{s_1\overset{\alpha}{\longrightarrow}_1 \red{s'_1}}
    {\langle s_1,s_2\rangle\overset{\alpha}{\longrightarrow}\langle \red{s'_1},s_2\rangle}
    \text{ and }
    \frac{s_2\overset{\alpha}{\longrightarrow}_2 \red{s'_2}}
    {\langle s_1,s_2\rangle\overset{\alpha}{\longrightarrow}\langle s_1,\red{s'_2}\rangle}
    \end{align*}
\end{mdframed}

A Computation tree is obtained from transition system by unfolding operation:
\begin{itemize}
    \item $s_k$ is a successor node of $s_i$ in the computation tree, iff there is a transition from $s_i$ to $s_k$ in the transition system.
\end{itemize}

\subsection{Property Specification}

based on modal logic:
\begin{description}
    \item[$Lp$] it is \emph{necessary} that $p$
    \item[$Mp$] it is \emph{possible} that $p$
    \item[$\neg Lp$] it is \emph{not necessary} that $p$
    \item[$\neg Mp$] it is \emph{not possible} that $p$
\end{description}

\begin{mdframed}[roundcorner=5pt,
subtitlebelowline=false,subtitleaboveline=false,
subtitlebackgroundcolor=blue!30,
% backgroundcolor=blue!20,
frametitlerule=true,
frametitlebackgroundcolor=blue!30,
frametitle={Kripke-Structure}
]
Let
\begin{itemize}
    \item $M=W,V,A$ be a Kripke-Structure:
    \item $\Pi$ a set of atomic propositions and $p\in\Pi$
    \item $w,v\in W$
    \item $\Phi,\rho$ formulae
\end{itemize}
then we define the relation $\models$ (satisfaction relation) for $M$:
\begin{align*}
(M,w)&\models p & \iff&& A(w,p)&=true \\
(M,w)&\models\neg p& \iff&& A(w,p)&=false \\
(M,w)&\models\Phi\wedge\rho & \iff&& (M,w)&\models\Phi \text{ and } (M,w)\models \rho \\
(M,w)&\models L\Phi& \iff& & (\forall v:(w,v)&\in V)((M,v)\models\Phi) \\
(M,w)&\models M\Phi& \iff& & (\exists v:(w,v)&\in V)((M,v)\models\Phi)
\end{align*}

Further syntactic definitions:

\begin{align*}
\Phi&\vee\rho & \cong && &\neg(\neg\Phi\wedge\neg\rho) \\
\Phi&\supset\rho & \cong && &\neg\Phi\vee\rho & \text{\tiny implies}\\
\Phi&\equiv\rho & \cong && &(\Phi\supset\rho)\wedge(\rho\supset\Phi)\\
M\Phi& &\cong&& &\neg L\neg\Phi
\end{align*}
\end{mdframed}

In temporal logic:
\begin{itemize}
    \item $Mp$ corresponds to $\square p$
    \item $Lp$ corresponds to $\diamond p$
\end{itemize}

\begin{mdframed}[roundcorner=5pt,
subtitlebelowline=false,subtitleaboveline=false,
subtitlebackgroundcolor=blue!30,
% backgroundcolor=blue!20,
frametitlerule=true,
frametitlebackgroundcolor=blue!30,
frametitle={Computation Tree Logic Syntax (CTL Syntax)}
]
$a\in AP$
\\
CTL state formula $\Phi$:
\begin{align*}
true && \neg\Phi \\
a && \exists\upvarphi \\
\Phi_1\wedge\Phi_2 && \forall\upvarphi
\end{align*}
CTL path formula
\begin{align*}
\circ\Phi && \Phi_1\until\Phi_2
\end{align*}
To be syntactically correct, temporal operators and path quantifiers alternate.

Derived Operators:
\begin{align*}
potentially\ \Phi: && \exists\diamond\Phi&=\exists(true\until\Phi) \\
inevitably\ \Phi: && \forall\diamond\Phi&=\forall(true\until\Phi) \\
potentially\ always \Phi: && \exists\square\Phi&=\neg\forall\diamond\neg\Phi \\
invariantly\ \Phi: && \forall\square\Phi&=\neg\exists\diamond\neg\Phi \\
weak\ until\ \Phi: && \exists(\Phi\wuntil\Psi) &= \neg\forall((\Phi\wedge\neg\Psi)\until(\neg\Phi\wedge\neg\Psi)) \\
&& \forall(\Phi\wuntil\Psi)&=\neg\exists((\Phi\wedge\neg\Psi)\until(\neg\Phi\wedge\neg\Psi))
\end{align*}
\mdfsubtitle{Computation Tree Logic Semantic (CTL Semantic)}
CTL state formulae:
\begin{align*}
s&\models a & \iff& & a&\in L(s) \\
s&\models\neg\Phi &\iff& & \neg(s&\models\Phi) \\
s&\models\Phi\wedge\Psi &\iff&& (s&\models\Phi)\wedge(s\models\Psi) \\
s&\models\red\exists\varphi &\iff&& \pi&\models\varphi \text{ for \red{some} path $\pi$ that starts in $s$} \\
s&\models\red\forall\varphi &\iff&& \pi&\models\varphi \text{ for \red{all} path $\pi$ that starts in $s$} \\
\end{align*}
\mdfsubtitle{Satisfaction Set}
The satisfaction set $Sat(\Phi)$ for a CTL formula is defined by: \[ Sat(\Phi)=\{ s\in S| s\models\Phi \} \]
A $TS$ satisfies a CTL formula $\Phi$ if $\Phi$ holds in all initial states: \[ TS\models\Phi \; \iff\; \forall s_0\in I: s_0\models\Phi \]
\mdfsubtitle{CTL Equivalence}
CTL formulas $\Phi$ and $\Psi$ are \emph{equivalent}, $\Phi\equiv\Psi$ iff $Sat(\Phi)=Sat(\Psi)$.
\[ \Phi\equiv\Psi\;\iff\;(TS\models\Phi\,\iff\,TS\models\Psi) \]
\mdfsubtitle{Equivalence-based Rewrite Rules}
Duality Laws:
\begin{align*}
\forall\circ\Phi &\equiv \neg\exists\circ\neg\Phi \\
\exists\circ\Phi &\equiv \neg\forall\circ\neg\Phi \\
\forall\diamond\Phi &\equiv \neg\exists\square\neg\Phi \\
\exists\diamond\Phi &\equiv \neg\forall\square\neg\Phi \\
\forall(\Phi\until\Psi) &\equiv \neg\exists((\Phi\wedge\neg\Psi)\wuntil(\neg\Phi\wedge\neg\Psi))
\end{align*}
Expansion Laws:
\begin{align*}
\forall(\Phi\until\Psi) &\equiv \Psi\vee(\Phi\wedge\forall\circ\forall(\Phi\until\Psi)) \\
\forall\diamond\Phi &\equiv \Phi\vee\forall\circ\forall\diamond\Phi \\
\forall\square\Phi &\equiv \Phi\wedge\forall\circ\forall\square\Phi \\
\exists(\Phi\until\Psi) &\equiv \Psi\vee(\Phi\wedge\exists\circ\exists(\Phi\until\Psi)) \\
\exists\diamond\Phi &\equiv \Phi\vee\exists\circ\exists\diamond\Phi \\
\exists\square\Phi &\equiv \Phi\wedge\exists\circ\exists\square\Phi
\end{align*}
Distributive Laws:
\begin{align*}
\forall\square(\Phi\wedge\Psi) &\equiv \forall\square\Phi\wedge\forall\square\Psi \\
\exists\diamond(\Phi\wedge\Psi) &\equiv \exists\diamond\Phi\wedge\exists\diamond\Psi
\end{align*}
But:
\begin{align*}
\exists\square(\Phi\wedge\Psi) &\not\equiv \exists\square\Phi\wedge\exists\square\Psi \\
\forall\diamond(\Phi\wedge\Psi) &\not\equiv \forall\diamond\Phi\wedge\forall\diamond\Psi
\end{align*}
\end{mdframed}

\subsubsection{LTL}
\begin{missing}
    Definition and Explanation of LTL, see Model Checking aggregation.
\end{missing}

\begin{mdframed}[roundcorner=5pt,
subtitlebelowline=false,subtitleaboveline=false,
subtitlebackgroundcolor=blue!30,
% backgroundcolor=blue!20,
frametitlerule=true,
frametitlebackgroundcolor=blue!30,
frametitle={CTL and LTL equivalence}
]
CTL formula $\Phi$ and LTL formula $\psi$ are \emph{equivalent}, $\Phi\equiv\psi$, iff for any transition system $TS$ over $AP$
\[ TS\models\Phi \iff TS\models\psi \]
\\
There can only be equivalent $\Phi$ and $\psi$ if by omitting all path quantifiers from $\Phi$ yields \[ \Phi\equiv\psi \] otherwise there does not exist an equivalent LTL formula.
\\
\follows LTL and CTL have incomparable expressiveness.
\end{mdframed}

\subsubsection{CTL*}

\begin{mdframed}[roundcorner=5pt,
subtitlebelowline=false,subtitleaboveline=false,
subtitlebackgroundcolor=blue!30,
% backgroundcolor=blue!20,
frametitlerule=true,
frametitlebackgroundcolor=blue!30,
frametitle={CTL$^*$}
]
State formula: $\Phi\;:=\;true\;|\;a\;|\;\Phi_1\wedge\Phi_2\;|\;\neg\Phi\;|\;\exists\varphi$
\\
path formula: $\varphi\;:=\;\Phi\;|\varphi_1\wedge\varphi_2\;|\;\neg\varphi\;\circ\varphi\;|\varphi_1\until\varphi_2$
\mdfsubtitle{CTL$^*$ Semantics}
\begin{align*}
s&\models a &\iff&& a&\in L(s) \\
s&\models \neg\Phi &\iff&& not\,s&\models\Phi \\
s&\models \Phi\wedge\Psi &\iff&& (s&\models\Phi) \text{ and }(s\models\Psi) \\
s&\models \exists\varphi &\iff&& \pi&\models\varphi \text{ for some } \pi\in Paths(s)\\
\pi&\models\Phi &\iff&& \pi[0]&\models\Phi\\
\pi&\models \varphi_1\wedge\varphi_2 &\iff&& \pi&\models\varphi_1 \text{ and } \pi\models\varphi_2\\
\pi&\models \neg\varphi &\iff&& \pi&\not\models\varphi\\
\pi&\models \circ\varphi &\iff&& \pi[1..]&\models\varphi\\
\pi&\models \varphi_1\until\varphi_2&\iff&& \exists j&\geq0.(\pi[j..]\models\varphi_2\wedge(\forall 0\leq k<j.\pi[k..]\models\varphi_1))
\end{align*}
Satisfaction set and TS satisfaction is same as for CTL.
\begin{center}
\scalebox{1}{\begin{tikzpicture}[
]
\draw[green,thick] (0,0) ellipse (3 and 2);
\draw[blue,thick] (.5,0) circle (1.5);
\draw[orange,thick] (-.5,0) circle (1.5);
\node[green] (ctl*) at (2.6,0) {CTL$^*$};
\node[blue] (ctl) at (1.6,0) {CTL};
\node[orange] (ltl) at (-1.6,0) {LTL};
\end{tikzpicture}}
\end{center}
\end{mdframed}

\subsubsection{CTL Model Checking Procedure}

\begin{enumerate}
    \item convert CTL formula $\Phi'$ into an equivalent CTL formula $\Phi$ in \emph{Existential Normal Form (ENF)}
    \item recursively compute the set $Sat(\Phi)=\{s\in S|s\models\Phi\}$
    \item $TS\models\Phi$ iff \emph{each initial sate} of TS belongs to $Sat(\Phi)$
\end{enumerate}

\begin{mdframed}[roundcorner=5pt,
subtitlebelowline=false,subtitleaboveline=false,
subtitlebackgroundcolor=blue!30,
% backgroundcolor=blue!20,
frametitlerule=true,
frametitlebackgroundcolor=blue!30,
frametitle={ENF Conversion}
]
ENF Subset of CTL:
\[ \Phi\;:=\;true\;|\;a\;|\;\Phi_1\wedge\Phi_2\;|\;\neg\Phi\;|\;\exists\circ\Phi\;|\;\exists(\Phi_1\until\Phi_2\;|\;\exists\square\Phi) \]
Conversion Rules:
\begin{align*}
\forall&\circ\Phi &\equiv&& \neg\exists&\circ\neg\Phi \\
\forall&(\Phi\until\Psi) &\equiv&& \neg\exists&(\neg\Psi\until(\neg\Phi\wedge\neg\Psi))\wedge\neg\exists\square\neg\Psi
\end{align*}
\mdfsubtitle{Computation of $Sat(\Phi)$}
\begin{multicols}{2}
\begin{enumerate}
    \item create \emph{parse tree} from formula
    \item compute $Sat(a_i)$ for leaf nodes
    \item move up in the parse tree by level, computing $Sat(.)$ from child nodes
    \item when all root tree is computed, check if $I\in Sat(\Phi)$
\end{enumerate}
\begin{center}
\scalebox{1}{\begin{tikzpicture}[
state/.style={draw,rounded corners,text width=.5cm,align=center},
arrow/.style={draw,->,>=latex},
]
\node[state] (root) {$\wedge$};
\node[state,below left of=root] (l) {$\exists\circ$};
\node[state,below right of=root] (r) {$\exists\mathcal{U}$};
\node[state,below left of=l] (ll) {$a$};
\node[state,below left of=r] (rl) {$b$};
\node[state,below right of=r] (rr) {$c$};

\draw[arrow]
(root) edge (l)
(root) edge (r)
(l) edge (ll)
(r) edge (rl)
(r) edge (rr)
;
\end{tikzpicture}}
\end{center}
\end{multicols}
Computation of $Sat(a_i)$
\begin{align*}
Sat(true) &= S \\
Sat(a) &= \{s\in S | a\in L(s)\} \text{ for any } a\in AP \\
Sat(\blue\Phi\wedge\red\Psi) &= Sat(\blue\Phi) \cap Sat(\red\Psi) \\
Sat(\neg\Phi) &= s\setminus Sat(\Phi) \\
Sat(\exists\circ\Phi) &= \{s\in S | Post(s)\cap Sat(\Phi)\not=\varnothing\} \\
%
Sat(\exists(\blue\Phi\until\red\Psi)) &\text{ is the smallest subset $T$ of $S$, such that } \\
&Sat(\red\Psi)\subseteq T \text{ and} \\
&(s\in Sat(\blue\Phi) \text{ and } Post(s)\cap T\not=\varnothing)\Rightarrow s\in T \\
%
Sat(\exists\square\Phi) &\text{ is the largest subset $T$ of $S$, such that} \\
&T\subseteq Sat(\Phi) \text{ and} \\
&s\in T \longrightarrow Post(s)\cap T\not=\varnothing
\end{align*}
smallest Fixpoint calculation: $\exists(\Phi_1\until\Phi_2)$\\
\begin{algorithm}[H]
$T:=Sat(\Phi_2);$ \\
\While{$\{ s\in Sat(\Phi_1)\setminus T | Post(s)\cap T\not=\varnothing\}\not=\varnothing$}{
    $\{ s\in Sat(\Phi_1)\setminus T | Post(s)\cap T\not=\varnothing\}\not=\varnothing$ \\
    $T:=T\cup\{ s \}$
}
\end{algorithm}
\ \\
greatest Fixpoint calculation: $\exists\square\Phi$\\
\begin{algorithm}[H]
\While{$\{s\in T | Post(s)\cap T=\varnothing\}\not=\varnothing$}{
    let $\{s\in T | Post(s)\cap T=\varnothing\}$ \\
    $T:=T\setminus\{s\}$;
}
\end{algorithm}
\ \\ Compute $Sat(\exists(\blue\Phi\until\red\Psi))$ by \emph{Enumerative Backward Search} \\
\begin{algorithm}[H]
$T:=Sat(\red\Psi);$ \\
$E:=T$\\
\While{$E\not=\varnothing$}{
    $s'\in E;$ \\
    $E:=E\setminus\{s'\};$ \\
    \For{ all $s\in Pre(s')$ }{
    \If{$s\in Sat(\blue\Phi)\setminus T$}{
        $E:=E\cup \{s\};$ \\
        $T:=T\cup\{s\};$
        }
    }
}
\Return T
\end{algorithm}
\ \\ Compute $Sat(\exists\square\Phi)$ by \emph{Enumerative Backwards Search} \\
\begin{algorithm}[H]
$E:=S\setminus Sat(\Phi)$ \tcc*{$E$ contains any unvisited $s'$ with
$s'\not\models\exists\square\Phi$}
$T:=Sat(\Phi)$ \tcc*{$T$ contains any $s$ for which $s\models\exists\square\Phi$ is not disproven}
\For{all $s\in Sat(\Phi)$}{
    $c[s]:= |Post(s)|;$
}
\While{$E\not=\varnothing$}{
    $s'\in E;$ \\
    $E:=E\setminus\{s'\};$ \tcc*{$s'$ has been considered}
    \For{all $s\in Pre(s')$}{
        \If{$s\in T$}{
            $c[s]:=c[s]-1;$ \tcc*{update counter $c[s]$ for predecessor $s$ of $s'$}
            \If{$c[s]=0$}{
                $T:=T\setminus\{s\};$ \\
                $E:=E\cup\{s\}$
            }
        }
    }
}
\Return T
\end{algorithm}
\ \\ Alternative Algorithm for $Sat(\exists\square\Phi)$: \\
\begin{enumerate}
    \item Consider state $s$ only if $s\models\Phi$, otherwise \emph{eliminate} $s$
    \begin{itemize}
        \item change $TS$ into $TS[\Phi]=(S',Act,\to',I',AP,L')$ with $S'=Sat(\Phi)$
        \item $\to'=\to\cap(S'\times Act\times S'), I'=I\cap S', L'(s)=L(s)$ for $s\in S'$
        \item[\follows] all removed states do not satisfy $\exists\square\Phi$ and can therefore be removed
    \end{itemize}
    \item Dtermine all \emph{non-trivial strongly connected components} in $TS[\Phi]$
    \begin{itemize}
        \item non-trivial SCC \follows maximal, connected subgraph with at least one transition
        \item[\follows] any state in such SCC satisfies $\exists\square\Phi$
    \end{itemize}
    \item $s\models\exists\square\Phi$ is equivalent to ``some SCC is reachable from $s$''
    \begin{itemize}
        \item simple reachability search (backward manner)
    \end{itemize}
\end{enumerate}
\end{mdframed}

\subsubsection{Time Complexity}

The CTL Model Checking Problem $TS\models\Phi$ can be determined in $\mathcal{O}(|\Phi|\cdot(N+M)$, where $N$ is the number of states and $M$ the number of transitions, $N+M$ is the size of the transition system, which can be exponentially large.
\\
LTL Model Checking can be done in $\mathcal{O}((N+M)\cdot 2^{|\Phi|}$. But LTL formulae can be exponentially shorter.

\subsubsection{Counterexamples}

\begin{missing}
    Counterexamples in LTL, see Model Checking aggregation.
\end{missing}

\begin{mdframed}[roundcorner=5pt,
subtitlebelowline=false,subtitleaboveline=false,
subtitlebackgroundcolor=blue!30,
% backgroundcolor=blue!20,
frametitlerule=true,
frametitlebackgroundcolor=blue!30,
frametitle={Counterexample and Witnesses}
]
\begin{itemize}
    \item counterexample: path fragment $s\to s'$ such that
    \begin{itemize}
        \item $s\in I$ and $s'\in Post(s)$ with $s'\not\models\Phi$
    \end{itemize}
    \item witness: a path fragment $s\to s'$ such that
    \begin{itemize}
        \item $s\in I$ and $s'\in Post(s)$ with $s'\models\Phi$
    \end{itemize}
    \item algorithmic computation: Inspection of direct successors of initial states.
\end{itemize}
\mdfsubtitle{Witness for $\Phi\until\Psi$}
backwards search starting in $Sat(\Psi)$
\mdfsubtitle{Counterexample for $\Phi\until\Psi$}
has one of the forms:
\begin{itemize}
    \item $\underset{\text{satisfy } \Phi\wedge\neg\Psi}{\underbrace{s_0\dots s_{n-1}\underset{cycle}{\underbrace{s_ns_1'\dots s'_r}}}}$ with $s_n=s'_r$ (would work for $\wuntil$, but not $\until$)
    \item $s_0\dots s_{n-1}s_n$ where $s_n\models\neg\Phi\wedge\neg\Psi$
\end{itemize}
Computing Counterexample:
\begin{itemize}
    \item let $G=(S,E)$ a directed graph, where $S$ is the set of states of the $TS$ and $E=\{(s,s')\in S\times S | s'\in Post(s)\wedge s\models\Phi\wedge\neg\Psi\}$
    \item Each path in $G$ starting in an $s_0\in I$ leading to an trivial or non-trivial SCC yields a counterexample.
    \item[\follows] counterexample generation requires SCC computation (e.g. Tarjans Algorithm)
\end{itemize}
\end{mdframed}

\section{CTL* Model Checking}
\begin{mdframed}[roundcorner=5pt,
subtitlebelowline=false,subtitleaboveline=false,
subtitlebackgroundcolor=blue!30,
% backgroundcolor=blue!20,
frametitlerule=true,
frametitlebackgroundcolor=blue!30,
frametitle={CTL$^*$ Model Checking}
]
Follow same recursive pattern using parse tree as for CTL model checking.
\\
\begin{itemize}
    \item replace maximal proper state formula by new proposition $a_\Psi$
    \item $\Psi$ is a \emph{maximal proper state subformula} of $\Phi$ whenever $\Psi$ is a subformula of $\Phi$ that differs from $\Phi$ and that is not contained in any other proper state formula of $\Phi$.
    \item adjust labeling of $TS$ such that $a_\Psi\in L(s)$ iff $s\in Sat(\Psi)$
    \item[\follows] LTL formula \[ s\models\exists\varphi \;\iff\; s\not\models_{CTL^*}\forall\neg\varphi \;\iff\; s\not\models_{LTL}\neg\varphi \]
\end{itemize}
\mdfsubtitle{Algorithm}
\begin{algorithm}[H]
\For{all $i\leq|\Phi|$}{
    \For{all $\Psi\in Sub(\Phi)$ with $|\Psi|=i$}{
        \Switch{$\Psi$}{
            \begin{align*}
            true &:& Sat(\Psi)&:=S \\
            a &:& Sat(\Psi)&:=\{s\in S|a\in L(s)\}; \\
            a_1\wedge a_2 &:& Sat(\Psi)&:=Sat(a_1)\cap Sat(a_2); \\
            \neg a &:& Sat(\Psi)&:=S\setminus Sat(a); \\
            \exists\varphi &:& determine &Sat_{LTL}(\neg\varphi); \\
             &:& Sat(\Psi)&:=S\setminus Sat_{LTL}(\neg\varphi)
            \end{align*}
        }
        $AP:=AP\cup\{a_\Psi\};$ \tcc*{introduce fresh atomic proposition}
        replace $\Psi$ with $a_\Psi;$
        \For{all $s\in Sat(\Psi)$}{
            $L(s)\cup\{a_\Psi\};$
        }
    }
}
\Return $I\subseteq Sat(\Phi)$
\end{algorithm}
\end{mdframed}

\subsection{Time Complexity}

For transition systems with $N$ states and $M$ transitions the CTL$^*$ model checking problem $TS\models\Phi$ can be determined in $\mathcal{O}(N+M)\cdot2^{|\Phi|}$

\subsection{Fairness}
\begin{mdframed}[roundcorner=5pt,
subtitlebelowline=false,subtitleaboveline=false,
subtitlebackgroundcolor=blue!30,
% backgroundcolor=blue!20,
frametitlerule=true,
frametitlebackgroundcolor=blue!30,
frametitle={Fairness}
]
\emph{Fairness Constraints} \follows rule out unrealistic executions by putting constraints on actions that occur along infinite executions
\begin{align*}
unconditional && \Rightarrow && strong && \Rightarrow && weak
\end{align*}
weak rules out the least executions\\
\emph{Fairness Assumptions \follows distinct constraints on distinct action sets}
\mdfsubtitle{Fairness Constraints}
\begin{description}
    \item[unconditional LTL fairness constraint:] $u_{fair} = \square\diamond\Psi$
    \item[strong LTL fairness constraint:] $s_{fair}=\square\diamond\Phi\longrightarrow\square\diamond\Psi$
    \item[weak LTL fairness constraint:] $w_{fair}=\diamond\square\Phi\longrightarrow\square\diamond\Psi$
\end{description}
\[ fair = u_{fair} \wedge s_{fair} \wedge w_{fair} \]
\begin{itemize}
    \item strong and unconditional fairness \follows solve contentions
    \item weak fairness \follows resolve nondeterminism
\end{itemize}
\begin{align*}
FairPaths_{fair}(s) &= \{\pi\in Paths(s) | \pi\models fair\} \\
FairTraces_{fair}(s) &= \{trace(\pi)|\pi\in FairPaths_{fair}(s)\} \\
s\models_{fair}\varphi &\text{ iff } \forall\pi\in FairPaths_{fair}(s).\pi\models \varphi \\
TS\models_{fair}\varphi &\text{ iff }\forall s_0\in I.s_0\models_{fair}\varphi
\end{align*}
For TS and LTL formula $\varphi$ and LTL fairness assumption $fair$:
\[ TS\models_{fair}\varphi \text{ iff } TS\models(fair\to\varphi) \]
\mdfsubtitle{Fairness in CTL}
\follows ignore unfair paths
\begin{description}
    \item[unconditional] $u_{fair} = \bigwedge_{0<i\leq k}\square\diamond\Psi$
    \item[strong:] $s_{fair} = \bigwedge_{0<i\leq k}(\square\diamond\Phi_i\to\square\diamond\Psi_i)$
    \item[weak] $w_{fair} = \bigwedge_{0<i\leq k}(\diamond\square\Phi_i\to\square\diamond\Psi_i)$
\end{description}
A \emph{CTL fairness} constraint is an \emph{LTL} formula over \emph{CTL formulas}
\[ Sat_{fair}(\Phi) = \{s\in S| s\models_{fair} \Phi\} \]
For transition system $TS$ without terminal states, a CTL formula $\Phi$ in ENF and CTL fairness assumption $fair$:
\begin{enumerate}
    \item establish whether $TS\models_{air}\Phi$
    \item use bottom-up CTL procedure to determine $Sat_{fair}(\Phi)$
    \begin{enumerate}
        \item replace CTL-state formulas in $s_{fair}$ by atomic propositions \[ s_{fair}:=\bigwedge_{0<i\leq k}(\square\diamond a_i \to \square\diamond b_i) \]
    \end{enumerate}
\end{enumerate}
\mdfsubtitle{Fair CTL Model Checking}
$s\models_{fair}\exists\circ a$ iff $\exists s'\in Post(s)$ with $s'\models a$ and $\underset{s'\models_{fair}\exists\square true}{\underbrace{FairPaths(s')\not=\varnothing}}$
\\
$s\models{fair}\exists(a\until a')$ iff there exists a finite path fragment $s_0s_1\dots s_{n-1}s_n\in Paths_{fin}(s)$ with $n\geq 0$ such that $s_i\models a$ for $0\leq i<n,s_n\models a'$ and $\underset{s'\models_{fair}\exists\square true}{\underbrace{FairPaths(s')\not=\varnothing}}$
\\
Model Checking with fairness can be reduced to:
\begin{itemize}
    \item Model Checking CTL
    \item computing $Sat_{fair}(\exists\square a)$ for $a\in AP$
\end{itemize}

Algorithm: \\
\begin{algorithm}[H]
compute $Sat_{fair}(\exists\square true) = \{s\in S | FairPaths(s)\not=\varnothing\}$
\For{all $s\in Sat_{fair}(\exists\square true)$}{
    $L(s) := L(s) \cup \{a_{fair}\}$
    \tcc*{compute $Sat_{fair}(\Phi)$}
}
\For{all $0<i\leq|\Phi|$}{
    \For{all $\Psi\in Sub(\Phi)$ with $|\Psi|=i$}{
    \Switch{$\Psi$}{
    \begin{align*}
        true   &&: & &Sat_{fair}(\Psi) &:= S;\\
        a      &&: & &Sat_{fair}(\Psi) &:= \{s\in S | a\in L(s)\}; \\
        \neg a &&: & &Sat_{fair}(\Psi) &:= S\setminus Sat_{fair}(a); \\
        a \wedge a' &&: & &Sat_{fair}(\Psi) &:= Sat_{fair}(a) \cap Sat_{fair}(a'); \\
        \exists\circ a &&: & &Sat_{fair}(\Psi) &:= Sat(\exists\circ(a\wedge\red{a_{fair}})); \\
        \exists(a\until a') &&: & &Sat_{fair}(\Psi) &:= Sat(\exists(a\until(a'\wedge\red{a_{fair}}))); \\
        \exists\square a &&: & &Sat_{fair}(\Psi) &:= compute \, Sat_{fair}(\exists\square a);
    \end{align*}
    }
    replace all occurrences of $\Psi$ (in $\Phi$) b< the fresh atomic proposition $a_\Psi$
    \For{all $s\in Sat_{fair}(\Psi)$}{
        $L(s):=L(s)\cup\{a_\Psi\}$
    }
    }
}
\Return $I\subseteq Sat_{fair}(\Phi)$
\end{algorithm}

Computation of $Sat_{fair}(\exists\square a)$:
\begin{itemize}
    \item Consider state $s$ only if $s\models a$, otherwise eliminate $s$
    \item $s\models_{fair}\exists\square a$ iff there is a non-trivial SCC $D$ in $TS[a]$ reachable from $s$: \[ D\cap Sat(a_i)=\varnothing \text{ or } D\cap Sat(b_i)\not=\varnothing \text{ for } 0<i\leq k \]
    \item $Sat_{sfair}(\exists\square a)=\{s\in S| Reach_{TS[a]}(s)\cap T\not=\varnothing\}$ where $T$ is the union of all non-trivial SSCs $C$ that contain $D$ satisfying above equation.
\end{itemize}
\mdfsubtitle{Time Complexity}
$TS$ with $N$ states and $M$ transitions, CTL formula $\Phi$, CTL fairness constraint $fair$ with $k$ conjuncts: \\
CTL model checking in $\O(|\Phi|\cdot(N+M)\cdot k)$
\end{mdframed}

\section{Real-Time Model Checking}

\begin{mdframed}[roundcorner=5pt,
subtitlebelowline=false,subtitleaboveline=false,
subtitlebackgroundcolor=blue!30,
% backgroundcolor=blue!20,
frametitlerule=true,
frametitlebackgroundcolor=blue!30,
frametitle={soft real-time systems}
]
Violating soft real-time bounds does \emph{not} lead to invalidation of system. \follows ``quality of service'' requirements. Usually with probabilities attached (reach state X with probility of Y in Z time).
\mdfsubtitle{hard real-time systems}
Correctness of system depends on satisfying real-time constraints.
\mdfsubtitle{discrete time domain}
\begin{itemize}
    \item time advances in discrete steps
    \item actions only happen at natural time values \follows time domain $\mathds{N}$
\end{itemize}
\begin{description}
    \item[advantages]
    \begin{itemize}
        \item conceptually simple
        \item no need to change $TS$
        \item take LTL or CTL
        \item use traditional model checking algorithms
    \end{itemize}
    \item[disadvantages]
    \begin{itemize}
        \item fixed minimal delay granularity, between two points not observable
        \item not invariant to changes in time scale
        \item for asynchronous systems determination of mimimal delay hard
    \end{itemize}
\end{description}
\begin{itemize}
    \item time domain is dense
    \item infinite branching of computation tree
\end{itemize}
\mdfsubtitle{Clocks}
\begin{itemize}
    \item value increases while in a state
    \item may only be reset to zero
    \item can be referenced in constraints
    \item clocks increase at same pace (with rate $1$)
    \item[\follows] guards on edges
    \item[\follows] invariants on locations
\end{itemize}
\mdfsubtitle{Clock Constraints ($CC$)}
$c\in\mathds{N},\, x\in C,\, C \text{ set of clocks}$
\[ g:= x<c | x\leq c | x>c | x\geq c | g\wedge g \]
Clock constraints without any conjunctions are atomic: $ACC(C)$ \\
\follows \emph{rational} valued constraints can be translated into naturals by proper scaling.
\mdfsubtitle{Timed Automata}
$TA=(Loc,Act,C,\hookrightarrow,Loc_0,Inv,AP,L)$, where:
\begin{align*}
Loc & \text{ is a finite set of locations} \\
Loc_0 & \text{ is a set of initial locations} \\
C & \text{ is a finite set of clocks} \\
\hookrightarrow\subseteq Loc\times CC(C)\times Act\times2^C\times Loc & \text{ is a transition relation} \\
Inv: Loc\to CC(C) & \text{ is an invariant-assignement function} \\
L:Loc\to2^{AP} & \text{ is a labeling function} \\
\end{align*}
Edge $\ell \overset{g:\alpha, C}{\scalebox{5}[1]{$\hookrightarrow$}} \ell'$ means intuitively:
\begin{itemize}
    \item action $\alpha$ is enabled once guard $g$ holds
    \item when moving from $\ell$ to $\ell'$
    \begin{itemize}
        \item perform action $\alpha$
        \item reset any clock in $C$ to zero
        \item all clocks not in $C$ keep their value
    \end{itemize}
    \item Nondeterminism if multiple transitions are enabled
    \item $Inv(\ell)$ constraints amount of time that may be spent in location $\ell$
    \begin{itemize}
        \item once it becomes invalid, $\ell$ must be left
        \item if leaving is not possible, deadlock
    \end{itemize}
\end{itemize}
\mdfsubtitle{Composition}
$TA_i = (Loc_i,Act_i,C_i,\hookrightarrow_i,Loc_{0,i},Inv_i,AP_i,L_i)$ and handshake action set $H$:
\[ TA_1 \Vert_H TA_2= (Loc,Act_1\cup Act_2,C,\hookrightarrow,Loc_0,Inv,AP,L) \]
where
\begin{align*}
Loc   &= Loc_1\times Loc_2 \\
Loc_0 &= Loc_{0,1} \times Loc_{0,2} \\
C     &= C_1 \cup C_2 \\
Inv(\langle\ell_1,\ell_2\rangle) &= Inv_1(\ell_1) \wedge Inv_2(\ell_2) \\
L(\langle\ell_1,\ell_2)\rangle) &= L_1(\ell_1) \cup L_2(\ell_2)
\end{align*}
$\hookrightarrow$ is defined by
\begin{description}
    \item[$\alpha\in H$:] $\frac{
    \ell_1 \overset{g_1:\alpha, D_1}{\scalebox{5}[1]{$\hookrightarrow$}}_1 \ell'_1 \wedge \ell_2 \overset{g_2:\alpha, D_2}{\scalebox{5}[1]{$\hookrightarrow$}}_2 \ell'_2
    }{
    \langle\ell_1,\ell_2\rangle\overset{g_1\wedge g_2: \alpha, D_1\cup D_2}{\scalebox{5}[1]{$\hookrightarrow$}} \langle\ell_1,\ell_2\rangle
    }$
    \item[$\alpha\not\in H$:] $\frac{
    \ell_1 \overset{g_1:\alpha, D_1}{\scalebox{5}[1]{$\hookrightarrow$}}_1 \ell'_1
    }{
    \langle\ell_1,\ell_2\rangle\overset{g_1: \alpha, D_1}{\scalebox{5}[1]{$\hookrightarrow$}} \langle\ell_1,\ell_2\rangle
    }$
    and
    $\frac{
    \ell_2 \overset{g_2:\alpha, D_2}{\scalebox{5}[1]{$\hookrightarrow$}}_2 \ell'_2
    }{
    \langle\ell_1,\ell_2\rangle\overset{g_2: \alpha, D_2}{\scalebox{5}[1]{$\hookrightarrow$}} \langle\ell_1,\ell_2\rangle
    }$
\end{description}
\mdfsubtitle{Clock Valuations}
\begin{itemize}
    \item \emph{clock valuation} $\eta$ for set $C$ of clocks is a function $\eta:C\to\mathds{R}_{\geq0}$ assigning each clock $x\in C$ its current value $\eta(x)$
    \item $\eta+d$ for $d\in\mathds{R}_{\geq0}$ is defined by: $(\eta+d(x)=\eta(x)+d$ for all clocks $x\in C$
    \item $reset\,x$ in $\eta$ for clock $x$ is defined by:
    \[ (reset\, x\,in\, \eta)(x)=\begin{cases} \eta(y) & if\,y\not=x \\ 0 & if\,y=x \end{cases} \]
\end{itemize}
\mdfsubtitle{Satisfaction of Clock Constraints}
$\models\subseteq Eval(C)\times CC(C)$ is defined by:
\begin{align*}
\eta\models true \\
\eta\models x<c         & iff\,\eta(x) <c \\
\eta\models x\leq c     & iff\,\eta(x) \leq c \\
\eta\models x>c         & iff\,\eta(x) >c \\
\eta\models x\geq c     & iff\,\eta(x) \geq c \\
\eta\models g\wedge g'  & iff\,\eta\models g \wedge\eta\models g'
\end{align*}
\mdfsubtitle{Transition System from Timed Automata}
For the timed automaton $TA=(Loc,Act,C,\hookrightarrow,Loc_0,Inv,AP,L)$ the transition system is: $TS(TA) = (S,Act',\to,I,AP',L')$
\begin{align*}
S &= Loc\times Eval(C), \text{ so states are of the form } s=\langle\ell,\eta\rangle \\
Act'&=Act\cup\mathds{R}_{\geq0},\text{ (discrete) actions and time passage actions} \\
I &= \{\langle\ell_0,\eta_0\rangle | \ell_0\in Loc_0 \wedge\eta_0(x)=0 \text{ for all } x\in C\} \\
AP' &= AP\cup ACC(C) \\
L' (\langle\ell,\eta\rangle) &= L(\ell)\cup\{g\in ACC(C)|\eta g\} \\
\to& \text{is the transition relation defined below}
\end{align*}
\begin{description}
    \item[Discrete Transition:] $\langle\ell,\eta\rangle\overset{\alpha}{\longrightarrow}\langle\ell',\eta'\rangle$ if there is a transition labeled $(g:\alpha,D)$ from location $\ell$ to $\ell'$ such that:
    \begin{itemize}
        \item $g$ is satisfied by $\eta$, i.e. $\eta\models g$
        \item $\eta'=\eta$ with all clocks in $D$ resets to $0$, i.e. $\eta'=reset\, D\,in\,\eta$
        \item $\eta'$ fulfills the invariant of location $\ell'$, i.e. $\eta'\models Inv(\ell')$
    \end{itemize}
    \item[Delay Transition] $\langle\ell,\eta\rangle\overset{d}{\longrightarrow}\langle\ell,\eta+d\rangle$ for $d\in\mathds{R}_{\geq0}$ if $\eta+d\models Inv(\ell)$
\end{description}
\follows uncountably many states of the form $\langle\ell,\eta+t\rangle$ possible
\mdfsubtitle{Timed Paths through $TS(TA)$}
Model possible behaviour of $TA$. Not every Path is realistic:
\begin{description}
    \item[time convergence:] time converges to a specific value \\
    Time convergence is unrealistic and needs to be ignored (similar to unfair paths) \follows only use time-divergent paths.
    \item[timelock:] passage of time stops \\
    $TA$ is \emph{timelock-free} if no state in $Reach(TS(TA))$ contains a timelock \\
    timelocks are modelling flaws \follows need mechanisms to check for them
    \item[zenoness:] infinitely many actions take place in finite time. \\
\end{description}
\mdfsubtitle{Zenoness}
\begin{itemize}
    \item A $TA$ that performs infinitely many actions in finite time is \emph{zeno}.
    \item Path $\pi$ in $TS(TA)$ is \emph{zeno}, if it is time-convergent and infinitely many actions $\alpha\in Act$ are executed along $\pi$
    \item $TA$ is \emph{non-zeno} if there does not exist a zeno path in $TS(TA)$
    \begin{itemize}
        \item any $\pi$ in $TS(TA)$ is time-divergent
        \item any time-convergent path has at least one delay transition.
    \end{itemize}
    \item Sufficient Condition for \emph{Non-Zenoness} (static analysis):\\
    Let $TA$ with set $C$ of clocks such that for every (control) cycle:
    \[ \ell_0\overset{g_1:\alpha_1,C_1}{\scalebox{5}[1]{$\hookrightarrow$}}\ell_1\overset{g_2:\alpha_2,C_2}{\scalebox{5}[1]{$\hookrightarrow$}}\dots\overset{g_n:\alpha_n,C_n}{\scalebox{5}[1]{$\hookrightarrow$}}\ell_n=\ell_0 \] there exists a clock $x\in C$ such that:
    \begin{enumerate}
        \item $x\in C$, for some $0<i\leq n$, and
        \item for all clock evaluations $\eta$ there exists $c\in\mathds{N}_{>0}$ such that \[ \eta(x)<c \text{ implies } (\exists0<j\leq n.n\not\models g_j \text{ or } \eta\not\models Inv(\ell_j)) \]
    \end{enumerate}
\end{itemize}
\mdfsubtitle{Adequate Modelling}
A timed automaton is adequately modeling a time-critical system whenever it is \emph{non-zeno} and \emph{timelock-free}.
\end{mdframed}

\subsection{Timed CTL}

\begin{mdframed}[roundcorner=5pt,
subtitlebelowline=false,subtitleaboveline=false,
subtitlebackgroundcolor=blue!30,
% backgroundcolor=blue!20,
frametitlerule=true,
frametitlebackgroundcolor=blue!30,
frametitle={Syntax of Timed CTL (TCTL)}
]
TCTL formula over $AP$ and set $C$:
\[ \Phi:= true | a | g | \Phi\wedge\Phi | \neg\Phi | \exists\varphi | \forall\varphi \]
where $a\in AP, g\in ACC(C)$ and $\varphi$ is a path formula defined by $\varphi:=\diamond^J\Phi$ where
$J\subseteq\mathds{R}_{\geq0}$ is an interval whose bounds are natural: $J:[n,m], (n,m], [n,m), (n,m)$ for $n,m\in\mathds{N}$ and $n\leq m$, $m=\infty$ allowed for right-open intervalls.
\\
\begin{align*}
\exists\square^J\Phi&=\neg\forall\diamond^J\neg\Phi \\
\forall\square^J\Phi&= \neg\exists\diamond^J\neg\Phi \\
\diamond\Phi&=\diamond^{[0,\infty)}\Phi \\
\square\Phi&=\square^{[0,\infty)}\Phi
\end{align*}
\mdfsubtitle{Semantics of TCTL}
\begin{align*}
s&\models true \\
s&\models a &iff&& a&\in L(\ell) \\
s&\models g &iff&& & \eta\models g\\
s&\models \neg\Phi &iff&& & \neg s\models\Phi\\
s&\models \Phi\wedge\Psi &iff&& & (s\models\Phi) \text{ and }(s\models\Psi) \\
s&\models \exists\varphi &iff&& & \pi\models\varphi \text{ for some } \pi\in Pahts_{div}(s) \\
s&\models \forall\varphi &iff&& & \pi\models\varphi \text{ for all } \pi\in Pahts_{div}(s) \\\\
\end{align*}
\mdfsubtitle{Delay Equivalence Relation $\Longrightarrow$}
For infinite path fragments in $TS(TA)$ performing $\infty$ many actions, let
\[s_0\overset{d_0}{\Longrightarrow}s_1\overset{d_1}{\Longrightarrow}s_2\overset{d_2}{\Longrightarrow}\dots \text{ with } d_0,d_1,d_2\dots \geq0\]
denote the equivalence class such that the same time passes during the path. $ExecTime(\pi)=\sum_{i\geq0}d_i$
\mdfsubtitle{Satisfaction Set}
\[Sat(\Phi) = \{s\in Loc\times Eval(C) | s\models_{TCTL}\Phi\}\]
\[TA\models\Phi \text{ iff } \forall\ell_0\in Loc_0.\langle\ell_0,\eta_0\rangle\models\Phi\]
where $\eta_0(x)=0$ for all $x\in C$
\mdfsubtitle{TCTL and CTL}
TCTL only uses time-divergent paths, therefore:
\[ \underset{\text{TCTL semantics}}{\underbrace{TS(TA)\models_{TCTL} \forall\varphi}} \text{ but } \underset{\text{CTL semantics}}{\underbrace{TS(TA)\not\models_{CTL} \forall\varphi}} \]
\mdfsubtitle{Timelock}
A state is \emph{timelock-free} iff $\exists\square true$. I.e., there is a time-divergent path starting in this state.\\
$TA$ is timelock-free, iff $\forall s \in Reach(TS(TA)):\,s\models\exists\square true$\\
\follows Timelock-checking with a timed CTL formula.
\mdfsubtitle{TCTL Model Checking}
\[ \underset{\text{timed automaton}}{\underbrace{TA}}\models\Phi \iff \underset{\text{infinite transition system}}{\underbrace{TS(TA)}}\models\Phi \]
consider finite quotient of $TS(TA)$ \follows Region Transition System $RTS(TA)$ \\
transform TCTL formula $\Phi$ into ``equivalent'' CTL formula $\hat{\Phi}$
\[ TA\models_{TCTL}\Phi \iff \underset{finite transition system}{\underbrace{RTS(TA)}}\models_{CTL}\hat{\Phi} \]
\begin{enumerate}
    \item elimintation of timing parameters
    \begin{itemize}
        \item eliminate all intervalls $J\not=[0\infty)$ from TCTL formulas
        \item introduce fresh clock $z$ not formerly in $TA$
        \item $s\models\exists\diamond^J\Phi$ iff $reset\,z\in\,s\models z\in J\wedge\Phi$
        \item process $\exists\square^J\Phi, \forall\diamond^J\Phi, \forall\square^J\Phi$ similarly
    \end{itemize}
    Formally: for any state $s$ of $TS(TA)$ it holds:
    \[ s\models\exists\diamond^J\Phi\,\iff\,\underset{\text{state in } TS(TA\oplus z)}{\underbrace{s\{z:=0\}}}\models\exists\diamond((z\in J)\wedge\Phi) \] where $TA\oplus z$ it $TA$ over $C$ extended with $z\not\in C$
    \\
    For any state $s$ of $TS(TA) it holds. that:$
    \begin{enumerate}
        \item $s\models\exists(\Phi\until^J\Psi)$ iff $\underset{\text{state in } TS(TA\oplus z)}{\underbrace{s\{z:=0\}}}\models\exists((\Phi\vee\Psi)\until((z\in J)\wedge\Psi))$
        \item $s\models\forall(\Phi\until^J\Psi)$ iff $\underset{\text{state in } TS(TA\oplus z)}{\underbrace{s\{z:=0\}}}\models\forall((\Phi\vee\Psi)\until((z\in J)\wedge\Psi))$
    \end{enumerate}
    \item Clock Equivalence $\cong$ is an equivalence realtion on clock valuations:
    \begin{itemize}
        \item Equivalent clock valuations satisfy the same clock constraint $g$:
        \[ \eta\cong\eta'\Rightarrow(\eta\models g \iff \eta'\models g) \]
        \item Time-divergent paths of equivalent paths are ``equivalent'' \follows equivalent paths satisfy the same path formulas.
        \item The number of equivalence classes under $\cong$ is finite.
    \end{itemize}
    (a) and (b) are ensured, if equivalent states:
    \begin{itemize}
        \item agree on the integer part of all clock values
        \item agree on the ordering of the fractional parts of all clocks.
    \end{itemize}
    if clocks exceed the \emph{maximal constant} with which they are compared, their precise value is not of interest.
    \item Construct Region Transition System $TS=RTA(TA)$
    \item apply CTL model-checking algorithm to check $TS\models\Phi$
\end{enumerate}
\mdfsubtitle{Clock Equivalence}
$\eta$ and $\eta'$ are equivalent, $\eta\cong\eta'$ if: \\
{\tiny $c_x$ is the largest constant which $x$ is compared to
 }
\begin{itemize}
    \item for any $x\in C:\,\eta(x)>c_x \iff \eta'(x)>c_x$
    % \\ $\lfloor\eta(x)\rfloor=\lfloor\eta'(x)\rfloor$
    \item for any $x\in C:$ if $\,\eta(x),\eta'\leq c_x$ then: \\
    % $frac(\eta(x))=0 \iff frac(\eta'(x))=0$
    $\lfloor\eta(x)\rfloor = \lfloor\eta'(x)\rfloor$ and $frac(\eta(x))=0 \iff frac(\eta'(x))=0$
    \item for any $x,y\in C:$ if $\eta(x),\eta'(x)\leq c_x$ and $\eta(y),\eta'(y)\leq c_y$, then: \\
    $frac(\eta(x))\leq frac(\eta(y)) \iff frac(\eta'(x))\leq frac(\eta'(y))$
\end{itemize}
\begin{center}
\scalebox{1}{\begin{tikzpicture}[
arrow/.style={draw,->,>=stealth},
dot/.style={draw,fill=black,shape=circle,scale=.75},
label/.style={draw,rounded corners},
labelarrow/.style={draw,dotted,thick,->,>=latex},
]
\draw (0,0) grid[step=1] (3.5,3.5);
\draw[arrow] (0,0) -- (3.6,0) node[below] {$x$};
\path (0,-.1) -- (3,-.1) node[pos=0,below]{0} node[pos=0.33,below]{1} node[pos=0.66,below]{2} node[pos=1,below]{3};
\path (-.1,0) -- (-.1,3) node[pos=0,left]{0} node[pos=0.33,left]{1} node[pos=0.66,left]{2} node[pos=1,left]{3};
\draw[arrow] (0,0) -- (0,3.6) node[left] {$y$};

\draw  (0,3) -- (0.5,3.5);
\draw  (0,2) -- (1.5,3.5);
\draw  (0,1) -- (2.5,3.5);
\draw  (0,0) -- (3.5,3.5);
\draw  (1,0) -- (3.5,2.5);
\draw  (2,0) -- (3.5,1.5);
\draw  (3,0) -- (3.5,0.5);

% this is the vertical line
\draw[green,ultra thick] (3,0) edge (3,1);
% diagonal line
\draw[green,ultra thick] (2,1) edge (3,2);
% upper triangle
\draw[fill=green] (2,2) -- (3,3) -- (2,3) --cycle;
% lower triangle
\draw[fill=green] (0,2) -- (1,2) -- (1,3) --cycle;
% nodes
\node[dot] (00) at (0,0) {};
\node[dot] (10) at (1,0) {};
\node[dot] (20) at (2,0) {};
\node[dot] (30) at (3,0) {};
\node[dot] (01) at (0,1) {};
\node[dot] (11) at (1,1) {};
\node[dot] (21) at (2,1) {};
\node[dot] (31) at (3,1) {};
\node[dot] (02) at (0,2) {};
\node[dot] (12) at (1,2) {};
\node[dot] (22) at (2,2) {};
\node[dot,fill=green] (32) at (3,2) {};
\node[dot] (03) at (0,3) {};
\node[dot] (13) at (1,3) {};
\node[dot] (23) at (2,3) {};
\node[dot] (33) at (3,3) {};

% labels
% node
\node[label] (32l) at (5,3) {$x=3,y=2$};
\draw[labelarrow] (32l) edge[bend left] (32);

%line
\node[label] (30l) at (5.5,.5) {$x=3,0<y<1$};
\draw[labelarrow] (30l) edge (3,.5);

% vertical line
\node[label] (21d) at (6.5,1.5) {$2<x<3,\,1<y<2,\,x-y=1$};
\draw[labelarrow] (21d) edge (2.5,1.5);

% upper triangle
\node[label] (25u) at (4,4) {$2<x<3,\,2<y<3,\,x-y<1$};
\draw[labelarrow] (25u) edge (2.3,2.7);

% lower triangle
\node[label] (01l) at (2,5) {$0<x<1,\,2<y<3,\,x-y>-2$};
\draw[labelarrow] (01l) edge[bend right] (.7,2.25);
\end{tikzpicture}}
\end{center}
furthermore: $s\cong s'$ iff $\ell=\ell'$ and $\eta\cong\eta'$
\mdfsubtitle{Regions}
\begin{description}
    \item[clock region:] $[\eta]=\{\eta'\in Eval(C) | \eta\cong\eta'\}$
    \item[state region:] $[s]=\langle\ell,[\eta]\rangle = \{\langle s,\eta'\rangle | \eta'\in[\eta]\}$ e
\end{description}
\mdfsubtitle{Bounds on number of regions}
$|C|!\cdot\prod_{x\in C} c_x\leq | \underset{\text{number of regions}}{\underbrace{Eval(C)\setminus\cong}} | \leq |C|!\cdot 2^{|C|-1} \cdot \prod_{x\in C} (2c_x+2)$ \\
The number of state regions is $|Loc|$ times larger. \\
\emph{Exponential} in number of Clocks.
\mdfsubtitle{Preservation of Atomic Properties}
\begin{enumerate}
    \item For $\eta,\eta'\in Eval(C)$ such that $\eta\cong\eta'$:
    \[ (\eta\models g \text{ if and only if } \eta'\models g) \text{ for any } g \in ACC(TA\cup\Phi) \]
    \item For $s,s'\in TS(TA)$ such that $s\cong s'$:
    \[ s\models a \text{ if and only iff } s'\models a \text{ for any } a\in AP' \]
\end{enumerate}
\end{mdframed}

\subsection{Region Automaton}
\begin{mdframed}[roundcorner=5pt,
subtitlebelowline=false,subtitleaboveline=false,
subtitlebackgroundcolor=blue!30,
% backgroundcolor=blue!20,
frametitlerule=true,
frametitlebackgroundcolor=blue!30,
frametitle={Unbounded Regions}
]
Clock reagion is \emph{unbounded}: $r_\infty=\{\eta\in Eval(C) | \forall x\in C.\eta(x)>c_x\}$
\mdfsubtitle{Successor Regions}
$r'$ is the \emph{successor} (clock) region of $r$, $r'=succ(r)$ if either:
\begin{enumerate}
    \item $r=r_\infty$ and $r=r'$
    \item $r\not=r_\infty,r\not=r'$ and $\forall\eta\in r$:
    \[ \exists d\in\mathds{R}_{>0}.(\eta+d\in r' \text{ and } \forall0\leq d'\leq d.\eta+d'\in r\cup r) \]
\end{enumerate}
The \emph{successor region}: $succ(\langle\ell,r\rangle) = \langle\ell,succ(r)\rangle$
\mdfsubtitle{Time Convergence}
Time convergent paths that only perform \emph{delay transitions}. \\
For non-zeno $TA$ and $\pi=s_0s_1s_2\dots$ a path in $TS(TA)$:
\begin{enumerate}
    \item $\pi$ is \emph{time convergent} \follows $\exists$ state region $\langle\ell,r\rangle$ such that for some $j$:
    \[ s_i\in\langle\ell,r\rangle \text{ for all } i\geq j \]
    \item If $\exists$ state region with $r\not=r_\infty$ and an index $j$ such that:
    \[ s_i\in\langle\ell,r\rangle \text{ for all } i\geq j \]
    then $\pi$ is \emph{time-convergent}.
\end{enumerate}
\mdfsubtitle{Region Automaton}
For non-zeno $TS$ with $TS(TA)=(S,Act,\to,I,AP,L)$ let:
\[ RTS(TA,\Phi)=(S',Act\cup\{\tau\},\to',I',AP',L') \text{ with} \]
\begin{align*}
S' = S\setminus\cong &= \{[s]|s\in S\} &\text{the state regions} \\
I'&=\{[s]|s\in I\} &\text{the initial states} \\
L'(\langle\ell,r\rangle) &= L(\ell)\cup \{g\in AP'\setminus AP |r\models g\} \\
\to' :& \frac{\ell \overset{g:\alpha,D}{\scalebox{5}[1]{$\hookrightarrow$}} \ell'\;r\models g\; reset\;D\;in\;r\models Inv(\ell')}{\langle\ell,r\rangle\overset{\alpha}{\longrightarrow}'\langle\ell',reset\;D\;in\;r\rangle} &\text{and} \\
&\frac{r\models Inv(\ell)\quad succ(r)\models Inv(\ell)}{\langle\ell,r\rangle\overset{\tau}{\longrightarrow'}\langle\ell,succ(r)\rangle}
\end{align*}
\mdfsubtitle{Correctness}
For non-Zeno timed automaton $TA$ and $TCTL_\diamond$ formula $\Phi$:
\[ \underset{\text{TCTL semantics}}{\underbrace{TA\models\Phi}} \text{ iff } \underset{\text{CTL semantics}}{\underbrace{RTS(TA,\Phi)\models\Phi}} \]
\mdfsubtitle{Timelock Freedom}
Non-zeno $TA$ is \emph{timelock-free} iff no reachable state in $RTS(TA)$ is terminal. \\
\follows timelock freedom checking can be reduced to reachability analysis on $RTS(TA)$
\end{mdframed}

\subsection{TCTL Model Checking Algorithm}
\begin{mdframed}[roundcorner=5pt,
subtitlebelowline=false,subtitleaboveline=false,
subtitlebackgroundcolor=blue!30,
% backgroundcolor=blue!20,
frametitlerule=true,
frametitlebackgroundcolor=blue!30,
frametitle={TCTL Model Checking Algorithm}
]
\begin{algorithm}[H]
R:=$RTS(TA\oplus z,\Phi)$; \tcc*{with state space $S_{rts}$ and labelling $L{rts}$}
\For{all $i\leq|\Phi|$}{
    \For{all $\Psi\in Sub(\Phi)$ with $|\Psi|=i$}{
    \Switch{$\Psi$}{
        \begin{align*}
        true &&:&& Sat_R(\Psi)&:= S_{rts};\\
        a &&:&& Sat_R(\Psi)&:= \{s\in S_{rts}| a\in L_{rts}(s)\};\\
        \Psi_1\wedge\Psi_2 &&:&& Sat_R(\Psi)&:= \{s\in S_{rts}| \{a_{\Psi_1},a_{\Psi_2}\}\subseteq L_{rts}(s)\};\\
        \neg\Psi' &&:&& Sat_R(\Psi)&:= \{s\in S_{rts}| a_{\Psi'}\not\in L_{rts}(s)\};\\
        \exists(\Psi_1\until^J\Psi_2) &&:&& Sat_R(\Psi)&:= Sat_{CTL}\left(\exists((a_{\Psi_1}\vee a_{\Psi_2})\until(z\in J)\wedge a_{\Psi_2})\right);\\
        \forall(\Psi_1\until^J\Psi_2) &&:&& Sat_R(\Psi)&:= Sat_{CTL}\left(\forall((a_{\Psi_1}\vee a_{\Psi_2})\until(z\in J)\wedge a_{\Psi_2})\right);
        \end{align*}
    }
    \For{all $s\in S_{rts}$ with $s\{z:=0\}\in Sat_R(\Psi)$}{
    $L_{rts}(s):=L_{rts}(s)\cup\{a_\Psi\}$
    \tcc*{add $a_\Psi$ to labelling of state regions where $\Psi$ holds}
    }
    }
}
\Return $I_{rts}\subseteq Sat_R(\Phi)$
\end{algorithm}
\mdfsubtitle{Time Complexity}
timed automaton $TA$, TCTL $\Phi$, $N$ is number of states, $K$ is number of transitions in $RTS(TA,\Phi)$:
\[ TA\models\Phi: \quad \o((N+K)\cdot|\Phi|) \]
\end{mdframed}

% part 4

\end{document}