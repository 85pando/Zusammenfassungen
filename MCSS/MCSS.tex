\documentclass[a4paper, 10pt]{article}

%\usepackage{scalefnt}
%\usepackage{parcolumns}

\usepackage{newclude}
% Zum Einbinden in die Zusammenfassungs-files

\usepackage{amsmath,amsthm,amsfonts,amssymb} % Verbesserter Mathesatz
\usepackage{algorithm2e}
\usepackage{bigfoot} % komplexe Fußnotenapparate(Fußnoten in Fußnoten und andere Späße)%
\usepackage{colortbl}
\usepackage[T1]{fontenc} % normaler erweitere Zeichnesatz
\usepackage{framed, color}  % Ramenpaket für zum Einfügen von schönen Ramen
\usepackage{graphicx}
\usepackage{hyperref} %used for link to creative commons license
\usepackage{listings} %for code-listings (inkl. Tab-Styling)
\usepackage{marvosym}
\usepackage{marginnote}
\usepackage{microtype} % div. Verbesserungen des Schriftsatzes (Grauwert, opt. Randausgleich, Zeilenumbruch)
%\usepackage{multirow}
\usepackage{multicol} %use this with next line for vertical divided environment
%\setlength\columnseprule{.4pt}:465
\usepackage[ngerman]{babel} % Neue Rechtschreibung
\usepackage{pifont}
\usepackage[sans]{dsfont} %für alternative Mengensymbole
\usepackage{stmaryrd} %u.a. für \lightning
\usepackage{tikz} % für Diagramme(Dia!) und Bilder (z.B. *.eps)/für ER-Diagramme/für UML-Diagramme
%\usepackage{tikz-er2} %  er-diagramme
\usepackage{tikz-uml} % uml-Diagramme
\usepackage{units} % z.B. fuer \nicefrac{}{}
\usepackage[utf8]{inputenc} % utf8 für den Editor
\usepackage{wasysym} %u.a. für \lightning
\usepackage{xcolor}


\usetikzlibrary{shapes,decorations,arrows,fit,backgrounds} %Zum diversen zeichen%
%\usetikzlibrary{automata} % für den CFlipper, wenn es den soweit ist			%
\usetikzlibrary{positioning} % positionierung
%\usetikzlibrary{shadows} % fuer schoene schlagschatten
\usetikzlibrary{automata} % für Automaten

%%%%%%%%%%%%%%%%%%%%%%%%%%%
%  Formatierung der Seite
%%%%%%%%%%%%%%%%%%%%%%%%%%%
\usepackage{fancyhdr}
\pagestyle{fancy}		% für den footer										%
\renewcommand{\headrulewidth}{0pt} % damit oben kein dummer Strich kommt		%
\fancyhead{}
\topmargin -2cm 		% Oberer Rand											%
\textheight 25cm		% Texthöhe												%
\textwidth 16.0 cm		% Textbreite											%
\oddsidemargin -0.1cm 	% Warum?												%
\newcommand{\Gruppe}[2]
{
	\lfoot{#1}
	\rfoot{#2}
}
\colorlet{shadecolor}{gray!25} % Farbe für graue Box definieren
%%%%%%%%%%%%%%%%%%%%%%%%%%%%%%%%%%%%%%%%%%%%%%%%%%%%%%%%%%%%%%%%%%%%%%%%%%%%%%%%%
%Farben die definiert werden zum schreiben und zeichnen							%
%%%%%%%%%%%%%%%%%%%%%%%%%%%%%%%%%%%%%%%%%%%%%%%%%%%%%%%%%%%%%%%%%%%%%%%%%%%%%%%%%
\xdefinecolor{schwarz}{HTML}{000000}
\xdefinecolor{dunkelGruen}{HTML}{007D00}
\xdefinecolor{dunkelBlau}{HTML}{0000A0}
\xdefinecolor{dunkelRot}{HTML}{A00000}
\xdefinecolor{dunkelGelb}{HTML}{FFAA00}
\xdefinecolor{hellesGelb}{HTML}{FFCC00}
\colorlet{dGreen}{dunkelGruen}
\colorlet{dBlue}{dunkelBlau}
\colorlet{dRed}{dunkelRot}
\colorlet{dYellow}{dunkelGelb}

%%%%%%%%%%%%%%%%%%%%%%%%%%%%%%%%%%%%%%%%%%%%%%%%%%
%Farbliche Ausgaben:
%Parameter #1: Text oder Mathematische formel...
%z.B. : \gruen{Hallo Welt Test!}
%%%%%%%%%%%%%%%%%%%%%%%%%%%%%%%%%%%%%%%%%%%%%%%%%%

\newcommand{\yellow}[1]{\textcolor{dYellow}{#1}}
\newcommand{\gray}[1]{\textcolor{gray}{#1}}
\newcommand{\red}[1]{\textcolor{red}{#1}}
\newcommand{\green}[1]{\textcolor{green}{#1}}
\newcommand{\blue}[1]{\textcolor{blue}{#1}}
\newcommand{\dGreen}[1]{\textcolor{dGreen}{#1}}
\newcommand{\dBlue}[1]{\textcolor{dBlue}{#1}}
\newcommand{\dRed}[1]{\textcolor{dRed}{#1}}

%%%%%%%%%%%%%%%%%%%%%%%%%%%%%%%%%%%%%%%%%%%%%%%%
%Konfiguration für das darstellen von Quelltext
%%%%%%%%%%%%%%%%%%%%%%%%%%%%%%%%%%%%%%%%%%%%%%%%
\lstset
{
	language=Java, % oder C++, Pascal, {[77]Fortran}, ...
	numbers=left, % Position der Zeilennummerierung
	firstnumber=auto, % Erste Zeilennummer
	basicstyle=\ttfamily, % Textgröße des Standardtexts
	keywordstyle=\ttfamily\color{dRot}, % Formattierung Schlüsselwörter
	commentstyle=\ttfamily\color{dGruen}, % Formattierung Kommentar
	stringstyle=\ttfamily\color{dBlau}, % Formattierung Strings
	numberstyle=\tiny, % Textgröße der Zeilennummern
	stepnumber=1, % Angezeigte Zeilennummern
	numbersep=5pt, % Abstand zw. Zeilennummern und Code
	aboveskip=15pt, % Abstand oberhalb des Codes
	belowskip=11pt, % Abstand unterhalb des Codes
	captionpos=b, % Position der Überschrift
	xleftmargin=10pt, % Linke Einrückung
	frame=single, % Rahmentyp
	breaklines=true, % Umbruch langer Zeilen
	showstringspaces=false, % Spezielles Zeichen für Leerzeichen
	tabsize=2,
	texcl=true
}

%%%%%%%%
% Kopf
%%%%%%%%
\newcommand{\Header}[3]
{
	{\footnotesize \parindent0em
		{\sc Universität Konstanz}                \hfill #1 \\
		{\sc Fachbereich Informatik \& Informationswissenschaft} \hfill #2 \\
		#3 \hfill \today
	}
}

%%%%%%%%%%%%%%%%%%%%%%%
% load some java code
% \loadJava{file}
%%%%%%%%%%%%%%%%%%%%%%%
\newcommand{\loadJava}[1]
{
	\lstinputlisting[language=Java]{#1.java}
}

%%%%%%%%%%%%%%%%%%%%%%%
% load some cpp code
% \loadCpp{file.cpp}
%%%%%%%%%%%%%%%%%%%%%%%
\newcommand{\loadCpp}[1]
{
	\lstinputlisting[language=C++]{#1}
}

%%%%%%%%%%%%%%%%%%%%%%%%%%%%%
% load some code
% \loadCode{Python}{file.py}
%%%%%%%%%%%%%%%%%%%%%%%%%%%%%
\newcommand{\loadCode}[2]
{
	\lstinputlisting[language=#1]{#2}
}

%%%%%%%%%%%%%%%%%%%%%%%%%%%%%%%%%%%%%%%%%%%%%%%%%%%%%%%%%%%%%%%%%%%%%%%
% some symbols
%%%%%%%%%%%%%%%%%%%%%%%%%%%%%%%%%%%%%%%%%%%%%%%%%%%%%%%%%%%%%%%%%%%%%%%
\newcommand{\correct}{\green{\text{\ding{52}}}} %for use in text and math
\newcommand{\wrong}{\red{\text{\ding{56}}}} %for use in text
\newcommand{\tflash}{$\yellow{\lightning}$} %for use in text
\newcommand{\mflash}{\yellow{\lightning}} %for use in math
\newcommand{\follows}{$\Rightarrow$} %used so often...
\newcommand{\good}{\item[\dGreen{\ding{58}}]} %an item with a green plus as bullet point
\newcommand{\bad}{\item[\red{\Emailct}]} %better icon for bad items
\newcommand{\note}[1]{\red{\marginnote{#1}}} %add a red margin note
\newcommand{\fitem}{\item[\follows]} %items with a follows arrow
\newcommand{\hm}{\ensuremath{\overset{-\mkern-11mu-\mkern-3.5mu\rhook}{\smash{\odot}\rule{0ex}{.46ex}}\underline{\hspace{0.5em}}\overset{-\mkern-11mu-\mkern-3.5mu\rhook}{\smash{\odot}\rule{0ex}{.46ex}}}}

%%%%%% make emph bold instead of italic %%%%%
\makeatletter
\DeclareRobustCommand{\em}{%
  \@nomath\em \if b\expandafter\@car\f@series\@nil
  \normalfont \else \bfseries \fi}
\makeatother

%%%%%%%%%%%%%%%%%%%%%%%%%%%%%%%%%%%%%%
% languages for \loadCode
%ABAP		IDL				Plasm
%ACSL		inform			POV
%Ada		Java			Prolog
%Algol		JVMIS			Promela
%Ant		ksh				Python
%Assembler	Lisp			R
%Awk		Logo			Reduce
%bash		make			Rexx
%Basic		Mathematica1	RSL
%C			Matlab			Ruby
%C++		Mercury			S
%Caml		MetaPost		SAS
%Clean		Miranda			Scilab
%Cobol		Mizar			sh
%Comal		ML				SHELXL
%csh		Modula-2		Simula
%Delphi		MuPAD			SQL
%Eiffel		NASTRAN			tcl
%Elan		Oberon-2		TeX
%erlang		OCL				VBScript
%Euphoria	Octave			Verilog
%Fortran	Oz				VHDL
%GCL		Pascal			VRML
%Gnuplot	Perl			XML
%Haskell	PHP				XSLT
%HTML		PL/I
%%%%%%%%%%%%%%%%%%%%%%%%%%%%%%%%%%%%%%

\begin{document}

\Gruppe{Stephan Heidinger}{MCSS - Zusammenfassung v0.1}
\Header{Modell Checking in Software Systems}{WS 12/13}{Stephan Heidinger}

\begin{shaded}
Dieses Dokument wurde unter der Creative Commons - Namensnennung-NichtKommerziell-Weitergabe unter gleichen Bedingungen (\textbf{CC by-nc-sa}) veröffentlicht. Die Bedingungen finden sich unter \href{http://creativecommons.org/licenses/by-nc-sa/3.0/de}{diesem Link}. \\
\centerline{./../cc-by-nc-sa.png}
\end{shaded}

\textit{\hm \; Find any errors? Please send them back, I want to keep them!}

\section*{Basic Principle}
\begin{multicols}{2}
given:
\begin{itemize}
    \item Software Model $M$
    \item Property Specification $S$
\end{itemize}
does $M\models S$
\columnbreak
\begin{center}
\scalebox{0.75}{%!TEX root = ../MCSS.tex
\begin{tikzpicture}[
]

\tikzstyle{nome}=[draw, anchor=west,text width=3.5cm]

\draw[thick,fill=red!70]    (0,0) circle (2.5);
\draw[thick,fill=violet!70] (0,0) circle (2);

\node[nome,text=violet!70] at (3,.26) {Software Model};
\node[nome,text=red!70]    at (3,-.25) {Property Specification};

\end{tikzpicture}}
\end{center}
\end{multicols}

\section*{Classification of Software Systems}
\begin{description}
    \item[Transformational Systems] transforms set of (empty) input data into output data, function from state $S_i$ to $S_k$ \\ correctness:
    \begin{itemize}
        \item termination
        \item correctness of function $S_i\to S_k$
        \item correctness of input - output transformation
    \end{itemize}
    \item[Reactive Systems] ongoing interaction with environment, driven by environment \\ correctness:
    \begin{itemize}
        \item non-termination (normally)
        \item correctness of stimuli-response pairs
    \end{itemize}
    \item[Embedded Systems] usually reactive, directly connected to hardware
    \item[Cyber-Physical Systems] integration of computation and physical processes
    \item[Real-Time Systems] systems, where correctness depends on time a result is delivered
    \begin{description}
        \item[Soft Real-Time Systems] missed deadlines will decrease result, not lead to failure, propabilities
        \item[Hard Real-Time Systems] missed deadlines will lead to failure
    \end{description}
    \item[Hybrid Systems] state is characterized by discrete and continuous variables
    \item[Safety-Critical Systems] systems failure may entail, death, serious injury, environmental harm, damage to property/assets
\end{description}

\section*{Requirement Specification}
\begin{description}
    \item[Natural Language] \ \\
    \begin{itemize}
        \good very expressive
        \good understood by all parties
        \bad ambiguous
    \end{itemize}
    \item[Formal Language] \ \\
    \begin{itemize}
        \good unambigous
        \good machine-analyzeable
        \bad limited expressiveness
        \bad hard to understand
    \end{itemize}
\end{description}

\section*{Software Verification Method}
\subsection*{Requirements}
\begin{itemize}
    \item formal foundation (automatic procedures)
    \item should be capable of relating artifacts from different stages in design cycle (formal vs informal req, design vs req, \dots)
    \item should be easy to integrate in design cycle (high degree of automation, low degree of interaction)
    \item scalable
\end{itemize}

\subsection*{Model Checking Process}
\begin{enumerate}
    \item provide model (e.g. Promela), involves abstraction
    \item simulate (check if model does, what you want)
    \item elicit and formalize requirements $\Rightarrow$ property specification
    \item model execution: run model checker with model and specification\\ outcome:
    \begin{itemize}
        \item property is valid (check next property)
        \item property is invalid (counterexample, check model for errors, check properties for errors, rethink design)
        \item exhaust of memory (use more abstraction, state space reduction, incomplete methods)
        \item exhaust of time (smaller model, faster computer)
    \end{itemize}
\end{enumerate}

\section*{\textsc{Spin}}
\begin{itemize}
    \item designed for communication protocols
    \item Open Source
    \item still in development
    \item Promela (\textsc{PROtocol/PROcess MEta LAnguage})
    \begin{itemize}
        \item concurrent modeling language
        \item guarded commands
        \item modeling of reactive systems
    \end{itemize}
\end{itemize}

\subsection*{State-Based Modeling}
\subsubsection*{State}
\begin{itemize}
    \item salient features of a system at given point of observation
    \item states can be observed, as long as features of interest unchanged
    \item features of interest
    \begin{description}
        \item[point of control] ``program counter'' (of all processes)
        \item[values] of local and global variables
        \item[communication channels] (messages sent but not received)
    \end{description}
    \item state vector (byte-wise representation of features of interest) \\
    % \begin{center}
    \scalebox{1}{\begin{tikzpicture}[
]

\draw[thick,fill=gray!50] (1,0) rectangle (2,1);
\node (pc) at (1.5,.5) {PC};

\draw[thick,fill=blue!50] (2,0) rectangle (6,1);
\node (var) at (4,.5) {variables};

\draw[thick,fill=orange!50] (6,0) rectangle (10,1);
\node (chan) at (8,.5) {communication channels};

\end{tikzpicture}}
    % \end{center}
\end{itemize}

\subsubsection*{State Transition}
\begin{itemize}
    \item instantaneous change of observed features of the system (``something happens'')
    \item represents computation step
    \item sequence of state transitions characterize system computation \\
    % \begin{center}
    \scalebox{1}{\begin{tikzpicture}[
state/.style={draw,rounded corners},
every node/.style={node distance = 70},
every path/.style={draw,->, >=stealth, thick, minimum width=1cm},
]

\node[state] (zero) at (0,0) {$S_0$};

\node[state] (one) [right of = zero] {$S_1$};

\node (ff1) [above right of = one] {$\dots$};
\node (ff2) [below right of = one] {$\dots$};

\node[state] (fin) [below right of = ff1] {$S_{fin}$};

\path
	(zero) edge (one)
	(one) edge (ff1)
	(ff1) edge (fin)
	(one) edge (ff2)
	(ff2) edge (fin)
;

\end{tikzpicture}}
    % \end{center}
    \item in real-time systems, time passes in states
    \item in stochastic systems, transitions are labeled with probability distributions
    \item in hybrid systems, continuous and discrete state variables change during state transition
    \item transitions show the possible events
    \item transition sequences are valid computations sequences
    \item transitions encode history ($S_0$ has been visited before $S_1$)
\end{itemize}

\subsubsection*{Devising State-Machines}
\begin{itemize}
    \item ``programm a model''
    \item abstraction: focus on relevant, can lead to non-determinism
    \item simplicity: find most simple abstraction that still reveals phenomena
\end{itemize}

\subsubsection*{Deadlock}
\begin{itemize}
    \item concurrent processes wait for each other in a circular wait with no pre-emption
    \item highly undesired
\end{itemize}

\subsubsection*{Closed System Modeling}
\begin{itemize}
    \item model checker can only validate all possible system executions under assumed environment
    \item model includes also environment
\end{itemize}

\subsection*{Transition Systems}
\begin{shaded}
Transition System $TS$ is a tuple $(S,Act,\to,I,AP,L$ where \\
\begin{minipage}{0.48\textwidth}
\begin{tabular}{ll}
$S$ & is a set of states \\
$Act$ & is a set of actions \\
$\to\subseteq S\times Act\times S$ & is a transition relation \\
$I\subseteq S$ & is a set of initial states \\
$AP$ & is a set of atomic propositions \\
$L:S\to2^AP$ & is a labeling function \\
\end{tabular}
\end{minipage}
\hfill \vrule \hfill
\begin{minipage}{0.48\textwidth}
\begin{itemize}
    \item where $S$ and $Act$ are finite or countably infinte
    \item we write $s\to^\alpha s'$ for $\left(s,\alpha,s'\right)\in\to$
    \item Atomic Propositions: Facts that we want to observer/that are observable in the system in any given state.
    \item Labeling Function: shows, which atomic propositions hold in given state.
\end{itemize}
\end{minipage}
\end{shaded}

\begin{shaded}
    \begin{align*}
        Post(s,\alpha) &= \left\{ s'\in S | s\overset{\alpha}{\longrightarrow} s' \right\}
        & Post(s)&=\bigcup_{\alpha\in Act} Post(s,\alpha)
        \\
        Pre(s,\alpha) &= \left\{ s'\in S | s'\overset{\alpha}{\longrightarrow} s \right\}
        & Pre(s)&=\bigcup_{\alpha\in Act} Pre(s,\alpha)
        \\
        Post(C,\alpha) &= \bigcup_{s\in C} Post(s,\alpha),
        & Post(C) &= \bigcup_{s\in C} Post(s)\textrm{ for } C\subseteq S
        \\
        Pre(C,\alpha) &= \bigcup_{s\in C} Pre(s,\alpha),
        & Pre(C) &= \bigcup_{s\in C} Pre(s)\textrm{ for } C\subseteq S
        \\
    \end{align*}
    \begin{center}
    a state is \emph{terminal} or \emph{final} iff $Post(s)=\varnothing$
    \end{center}
\end{shaded}

\subsubsection{(Non)Determinism}
\begin{shaded}
    A transition system $TS=\left(S,Act,\to,I,AP,L\right)$ is \emph{action-deterministic}, iff for all $s,\alpha$
    \begin{itemize}
        \item $|I|\leq1$ and
        \item $|Post(s),\alpha)|\leq1$
    \end{itemize}
    else it is \emph{action-nondeterministic} \\
    {\tiny there is at most one outgoing transition from each state labeled $\alpha$}
\end{shaded}

\begin{shaded}
    A transition system $TS=\left(S,Act,\to,I,AP,L\right)$ is \emph{AP-deterministic}, iff for all $s,A\in2^{AP}$
    \begin{itemize}
        \item $|I|\leq1$ and
        \item $|Post(s)\cap \left\{s'\in S | L(s')=A\right\}|\leq1$
    \end{itemize}
    else it is \emph{AP-nondeterministic} \\
    {\tiny every successor of a state $s$ has a unique $AP$ labeling}
\end{shaded}

\begin{itemize}
    \item Nondeterminism can lead to potentially smaller representation.
    \item in Software Engineering/Modeling: Abstraction
    \begin{itemize}
        \item avoid overspecification
        \item what does the system do, not how is it done
    \end{itemize}
    \begin{itemize}
        \item concurrency
        \begin{itemize}
            \item simulate concurrency by nondeterminsm
            \item either nondeterministic action can be executed first
        \end{itemize}
    \end{itemize}
\end{itemize}

\subsection*{System Execution}
Given a transition system $TS=\left(S,Act,\to,I,AP,L\right)$
\begin{description}
    \item[finite execution fragment] $\varrho$ of $TS$ is an alternating sequence of states and actions ending with a state:
    \[ \varrho = s_0\alpha_1 s_1\alpha_2\dots\alpha_n s_n \textrm{ s.t. } s_i\to^{\alpha_{i+1}} s_{i+1} \textrm{ for all } 0\leq i<n \]
    \item[infinite execution fragment] $\varrho$ of $TS$ is an infinite, alternating sequence of states and actions:
    \[ \varrho = s_0\alpha_1 s_1\alpha_2\dots\alpha_n s_n \textrm{ s.t. } s_i\to^{\alpha_{i+1}} s_{i+1} \textrm{ for all } 0\leq i \]
    \item[execution] of TS is an inital, maximal execution fragment
    \begin{itemize}
        \item an execution fragment is maximal, iff it is either
        \begin{itemize}
            \item finite and ending in a terminal state
            \item[or]
            \item infinite
        \end{itemize}
        \item an execution fragment is initial, iff $s_0\in I$
    \end{itemize}
    \item[reachable state] is a state $s\in S$ if there exists an initial, finite execution fragment $s_0\alpha_1s_1\dots\alpha_ns_n$ so that $s_n=s$
    \item[$Reach(TS)$] denotes the set of all reachable states in $TS$ \\
    % \begin{center}
    \scalebox{1}{\begin{tikzpicture}[
state/.style={draw,rounded corners},
every node/.style={node distance = 70},
every path/.style={draw,->, >=stealth, thick, minimum width=1cm},
]

\draw[fill=blue!40] (-1.5,2.5) rectangle (3,-2.5);
\node at (.75,-3) {$Reach(TS)$};
\node[state] (zero) at (0,0) {$s_0$};
\path node[node distance=40] [left of = zero] {} edge (zero);
\node[state] (one) [above right of = zero] {$s_1$};
\node[state] (two) [below right of = zero] {$s_2$};
\path 
(zero) edge (one)
(one) edge (two)
(two) edge (zero)
;

\draw[fill=orange!40] (3,2.5) rectangle (5.5,-2.5);
\node at (4.25,-3) {not reachable};
\node[state] (three) [right of = one] {$s_3$};
\node[state] (four) [right of = two] {$s_4$};
\path
(three) edge[bend left] (four)
(four) edge[bend left] (three)
;

\end{tikzpicture}}
    % \end{center}
\end{description}

\section*{Program Graphs \textsc{Pg}}
\begin{itemize}
    \item want to include things like variables, assignements, etc $\Rightarrow$ Program Graphs
    \item introduce conditional transitions (transition can only be exectured, if condition is true) \[ s \overset{\red g:\blue\alpha}{\longrightarrow}s' \]
    \begin{itemize}
        \item $\red g$: a boolean condition on data variables (``guard'')
        \item $\blue\alpha$: an action, that is possible, if g is satisfied
    \end{itemize}
    \item assume domain of variables as infinite
    \begin{itemize}
        \item practical implementations use variables over finite domains
    \end{itemize}
    \item Stepwise unfolding of $PG$s lead to $TS$
\end{itemize}

\begin{description}
    \item[valuation:] assign values to variables (e.g. $\eta(x)=17$)
    \item[propositional logic formulae]
    \item[effect] of actions: \[ Effect: Act\times Eval(Var)\to Eval(Var) \]
    Effects define \emph{operational semantics}
\end{description}

\begin{shaded}
    A \emph{program graph} $PG$ over set $Var$ of typed variables is a tuple $\left( Loc, Act, Effect, \to, Loc_0, g_0 \right)$ where
    \begin{itemize}
        \item $Loc$ is a set of locations with initial locations $Loc_0\subseteq Loc$
        \item $Act$ is a set of actions
        \item $Effect: Act\times Eval(Var)\to Eval(Var)$ is the effect function
        \item $\to \subseteq Loc\times \left(Cond(Var) \times Act\right) \times Loc $, transition relation {\tiny cond = boolean condition}
        \item $g_0\in Cond(Var)$ is the inital condition
    \end{itemize}
    Notation $\ell\overset{\red g:\blue\alpha}{\longrightarrow} \ell'$ denotes $\left(\ell,\red g, \blue\alpha,\ell'\right)\in \longrightarrow$
\end{shaded}
\subsection*{Semantics for Program Graphs}
\begin{itemize}
    \item Unfolding: Transformation of $PG$ to equivalent $TS$
    \begin{itemize}
        \item states:
        \begin{itemize}
            \item state $<l,\eta\rangle$: current control location $l$ + data valuation $\eta$
            \item initial state: initial location satisfying condition $g_0$
        \end{itemize}
        \item propositions:
        \begin{itemize}
            \item $at\_l$: control is at location $l$
            \item $x\in D$ iff $D\subseteq dom(x)$
        \end{itemize}
        \item labeling
        \begin{itemize}
            \item $<l,\eta\rangle$ is labeled with $at\_l$ and all conditions that hold in $\eta$
        \end{itemize}
    \end{itemize}
    \item from transitions in $PG$ to transitions in $TS$
    \begin{itemize}
        \item if $\ell\overset{\red g:\blue\alpha}{\longrightarrow} \ell'$ and $g$ holds in $\eta$, then $<l,\eta\rangle\to^\alpha<l',Effect(\alpha,\eta)\rangle$
    \end{itemize}
\end{itemize}

\subsection*{Structured Operational Semantics \textsc{Sos}}
\begin{itemize}
    \item definition of \emph{operational semantics} of a program in terms f computations steps defined by a transition system
    \item whether a step happens is determined by inference rules: \[ \frac{premise}{conclusion} \]
    \begin{itemize}
        \item if the premise holds, the conclusion holds (and can trigger further inference rules)
        \item if the premise is a \emph{tautology}, it may be omited\\
        the rule is then called an \emph{axiom}
    \end{itemize}
    \item semantics is structural, because it applies inference rules recursively to syntactic structure
\end{itemize}

\subsection*{Transition Systems for Program Graphs}
\begin{shaded}
    The transition system $TS(PG)$ of program graph \[PG=(Loc, \blue{Act}, Effect,\longrightarrow, Loc_0,g_0)\] over set $Var$ of variables is the tuple \[ (S,\blue{Act},\longrightarrow,I,AP,L) \] where
    \begin{itemize}
        \item $S=Loc\times Eval(Var)$
        \item $\longrightarrow\subseteq S\times Act\times S$ is defined by $\displaystyle\frac{\ell\overset{\red g:\blue\alpha}{\longrightarrow} \ell'\; \wedge \; \eta\models \red g}{<\ell,\eta\rangle\overset{\blue\alpha}{\longrightarrow}<\ell',Effect(\blue\alpha,\eta)\rangle}$
        \item $I=\left\{<\ell,\eta\rangle | \ell \in Loc_0,\eta \models g_0\right\}$
        \item $AP=Loc \cup Cond(Var)$ and $L\left(<\ell,\eta\rangle\right)=\{\ell\} \cup \left\{ g\in Cond(Var) | \eta \models g \right\} $
    \end{itemize}
\end{shaded}

\subsubsection*{Data in Transition Systems}
\begin{itemize}
    \item $TS$ do not possess data variables, data values and their changes need to be encoded in states
    \item assume $i=1,\dots,n$ variables of domain $s_i$ \follows $\Pi^n_{i=1}s_i$ \\
    \emph{combinatorial, exponential state space explosion}
    \item variables over infinite domains \follows infinite number of states
\end{itemize}

\section*{Modelling Concurrency}
\subsection{Concurrent Event}
\begin{itemize}
    \item logically independent events in separate autonomous system
\end{itemize}

\subsection*{Lamport's \emph{happened-Before} Relation $\to$}
\begin{shaded}
    \begin{description}
        \item[HB1:] for any pair of events $e$ and $e'$, if there is a process $p_i$ such that $e \to_i e'$, then $e \to e'$
        \item[HB2:] for any pair of events $e$ and $e'$ and for any message $m$, if $e = send(m)$ and $e' = receive(m)$, then $e \to e'$
        \item[HB3:] if $e$, $e'$ and $e''$ are events and if $e \to e'$ and $e' \to e''$, then $e \to e''$ (HB is identical to its transitive closure)
    \end{description}
\end{shaded}

\begin{center}
\scalebox{1}{%!TEX root = ../MCSS.tex
\begin{tikzpicture}[
arrow/.style={draw,->,>=stealth},
point/.style={circle,fill=black},
every node/.style={node distance = 10},
]

\node (p1) at (0,2) {$p_1$};
\node (p2) at (0,1) {$p_2$};
\node (p3) at (0,0) {$p_3$};

\node (p1l) at (0.2,2) {};
\node (p2l) at (0.2,1) {};
\node (p3l) at (0.2,0) {};

\node (p1r) at (8,2) {};
\node (p2r) at (8,1) {};
\node (p3r) at (8,0) {};

\path[draw] (p1l.center) edge (p3l.center);

\draw[arrow] (p1l.center) to (p1r);
\path[arrow] (p2l.center) to (p2r);
\path[arrow] (p3l.center) to (p3r);

\node[point] (a) at (1,2) {};
\node [below of = a] {a};
\node[point] (b) at (2.5,2) {};
\node [below of = b] {b};
\node[point] (c) at (4,1) {};
\node [below of = c] {c};
\node[point] (d) at (5.5,1) {};
\node [below of = d] {d};
\node[point] (e) at (1.75,0) {};
\node [below of = e] {e};
\node[point] (f) at (7,0) {};
\node [below of = f] {f};

\path[arrow] (b) edge node [right] {$m_1$} (c);
\path[arrow] (d) edge node [right] {$m_2$} (f);

\end{tikzpicture}}
\end{center}

\subsection*{Concurrency}
\begin{itemize}
    \item two events are $E_1, E_2$ are concurrent, if each ordering of them is possible
\end{itemize}

\subsection*{Interleaving}
\begin{itemize}
    \item concurrent actions $\alpha,\beta$ are executed in arbitrary order
    \item $Effect(\alpha \interleave \beta,\nu) = Effect((\alpha;\beta)+(\beta;\alpha),\nu$
    \begin{itemize}
        \item $\interleave$  is binary interleaving operator
        \item $;$ is sequential execution
        \item $+$ is nondeterministic choice
        \item $\underset{=\alpha}{\underbrace{x:=x+1}} \interleave \underset{=\alpha}{\underbrace{y:=y-2}}$ \\
        \begin{center}
        \scalebox{1}{%!TEX root = ../MCSS.tex
\begin{tikzpicture}[
state/.style={draw,rounded corners,text width=1cm,align=center},
every path/.style={draw,->,>=stealth,thick},
every label/.style={draw},
long/.style={node distance = 70},
short/.style={node distance = 50},
init/.style={node distance=70},
]

\node[state] (alpha1) at (0,0) {$x=0$};
\node[state,long] (alpha2) [below of = alpha1] {$x=1$};

\node[state,long] (beta1) [right of = alpha1] {$x=0$};
\node[state,long] (beta2) [below of = beta1] {$x=1$};

\path (alpha1) to node [right] {$\alpha$} (alpha2);
\path (beta1) to node [right] {$\beta$} (beta2);

\node (int) at (1.25,-1.2) {$\interleave$};
\node (eq)  at (3.5,-1.2)    {$=$};

\node[state] (int1) at (8,0) {$x=0$, $y=7$};
\node[state,short] (int12) [below left of = int1] {$x=1$, $y=7$};
\node[state,short] (int21) [below right of = int1] {$x=0$, $y=5$};
\node[state,short] (int3) [below right of = int12] {$x=1$, $y=5$};

\path (int1.west)  to[bend right] node [above left] {$\alpha$}  (int12.120);
\path (int1.east)  to[bend left]  node [above right] {$\beta$}  (int21.60);
\path (int12.240) to[bend right]  node [below left] {$\beta$}  (int3.west);
\path (int21.300) to[bend left]   node [below right] {$\alpha$}  (int3.east);

\end{tikzpicture}}
        \end{center}
    \end{itemize}
    \begin{shaded}
        Let $TS_i=(S_i,Act_i,\to_i,I_i,AP_i,L_i), i=1,2$ be two transition systems.
        \[ TS_1\interleave TS_2 = (S_1\times S_2,Act_1\uplus Act_2),\to,I_1\times I_2,AP_1\uplus AP_2, L) \]
        where $L(\langle s_1,s_2\rangle)=L_1(s_1) \cup L_2(s_2)$ and the transition relation $\to$ is defined by the rules:
        \[
        \frac{\red{s_1}\overset{\alpha}{\longrightarrow}_1 \red{s'_1}}
        {\langle\red{s_1},s_2\rangle \overset{\alpha}{\longrightarrow} \langle\red{s'_1},s_2\rangle}
        \textrm{ and }
        \frac{\red{s_2}\overset{\alpha}{\longrightarrow}_2 \red{s'_2}}
        {\langle s_1,\red{s_2}\rangle \overset{\alpha}{\longrightarrow} \langle s_1,\red{s'_2}\rangle}
        \]
    \end{shaded}
    \begin{itemize}
        \item Interleaving of Program Graphs\\
        let $PG_1$ and $PG_2$ be program graphs
        \begin{description}
            \item[without shared variables:] \ \\
            i.e. $Var_1\cap Var_2=\varnothing$\\
            concurrent behaviour: $TS(PG_1)\interleave TS(PG_2)$
            \item[with shared variables:] \ \\
            concurrent behaviour: $TS(PG_1 \interleave PG_2)$
            \item[in general:] \ \\
            $TS(PG_1) \interleave TS(PG_2) \not= TS(PG_1\interleave PG_2)$
        \end{description}
        \begin{shaded}
            For program graph $PG_i=(Loc_i,Act_i,Effect_i,\longrightarrow_i,Loc_{0,i},g_{0,i})$ over variables $Var_i$, the Program graph $PG_1\interleave PG_2$ over $Var_1\cup Var_2$ is defined by:
            \[ (Loc_1\times Loc_2,Act_1\uplus Act_2,Effect,\longrightarrow,Loc_{0,1}\times Loc_{0,2},g_{0,1}\wedge g_{0,2}) \]
            where $\longrightarrow$ is defined by the inference rules:
            \[
            \frac{\red{l_1} \overset{g:\alpha}{\longrightarrow} \red{l'_1}}
            {\langle \red{l_1},l_2 \rangle \overset{g:\alpha}{\longrightarrow} \langle \red{l'_1},l_2 \rangle}
            \textrm{ and }
            \frac{\red{l_2} \overset{g:\alpha}{\longrightarrow} \red{l'_2}}
            {\langle l_1,\red{l_2} \rangle \overset{g:\alpha}{\longrightarrow} \langle l_1,\red{l'_2} \rangle}
            \]
            and $Effect(\alpha,\eta)=Effect_i(\alpha,\eta)$ if $\alpha\in Act_i$
        \end{shaded}
    \end{itemize}
\end{itemize}

%next: 5-9

\end{document}
