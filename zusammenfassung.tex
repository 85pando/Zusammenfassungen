% Zum Einbinden in die Zusammenfassungs-files

\usepackage{amsmath,amsthm,amsfonts,amssymb} % Verbesserter Mathesatz
\usepackage{algorithm2e}
\usepackage{bigfoot} % komplexe Fußnotenapparate(Fußnoten in Fußnoten und andere Späße)%
\usepackage{colortbl}
\usepackage[T1]{fontenc} % normaler erweitere Zeichnesatz
\usepackage{framed, color}  % Ramenpaket für zum Einfügen von schönen Ramen
\usepackage{graphicx}
\usepackage{hyperref} %used for link to creative commons license
\usepackage{listings} %for code-listings (inkl. Tab-Styling)
\usepackage{marginnote}
\usepackage{microtype} % div. Verbesserungen des Schriftsatzes (Grauwert, opt. Randausgleich, Zeilenumbruch)
%\usepackage{multirow}
\usepackage{multicol} %use this with next line for vertical divided environment
%\setlength\columnseprule{.4pt}:465
\usepackage[ngerman]{babel} % Neue Rechtschreibung
\usepackage{pifont}
\usepackage[sans]{dsfont} %für alternative Mengensymbole
\usepackage{stmaryrd} %u.a. für \lightning
\usepackage{tikz} % für Diagramme(Dia!) und Bilder (z.B. *.eps)/für ER-Diagramme/für UML-Diagramme
%\usepackage{tikz-er2} %  er-diagramme
\usepackage{tikz-uml} % uml-Diagramme
\usepackage{units} % z.B. fuer \nicefrac{}{}
\usepackage[utf8]{inputenc} % utf8 für den Editor
\usepackage{wasysym} %u.a. für \lightning
\usepackage{xcolor}


\usetikzlibrary{shapes,decorations,arrows,fit,backgrounds} %Zum diversen zeichen%
%\usetikzlibrary{automata} % für den CFlipper, wenn es den soweit ist			%
\usetikzlibrary{positioning} % positionierung
%\usetikzlibrary{shadows} % fuer schoene schlagschatten
\usetikzlibrary{automata} % für Automaten

%%%%%%%%%%%%%%%%%%%%%%%%%%%
%  Formatierung der Seite
%%%%%%%%%%%%%%%%%%%%%%%%%%%
\usepackage{fancyhdr}
\pagestyle{fancy}		% für den footer										%
\renewcommand{\headrulewidth}{0pt} % damit oben kein dummer Strich kommt		%
\fancyhead{}
\topmargin -2cm 		% Oberer Rand											%
\textheight 25cm		% Texthöhe												%
\textwidth 16.0 cm		% Textbreite											%
\oddsidemargin -0.1cm 	% Warum?												%
\newcommand{\Gruppe}[2]
{
	\lfoot{#1}
	\rfoot{#2}
}
\colorlet{shadecolor}{gray!25} % Farbe für graue Box definieren
%%%%%%%%%%%%%%%%%%%%%%%%%%%%%%%%%%%%%%%%%%%%%%%%%%%%%%%%%%%%%%%%%%%%%%%%%%%%%%%%%
%Farben die definiert werden zum schreiben und zeichnen							%
%%%%%%%%%%%%%%%%%%%%%%%%%%%%%%%%%%%%%%%%%%%%%%%%%%%%%%%%%%%%%%%%%%%%%%%%%%%%%%%%%
\xdefinecolor{schwarz}{HTML}{000000}
\xdefinecolor{dunkelGruen}{HTML}{007D00}
\xdefinecolor{dunkelBlau}{HTML}{0000A0}
\xdefinecolor{dunkelRot}{HTML}{A00000}
\xdefinecolor{dunkelGelb}{HTML}{FFAA00}
\xdefinecolor{hellesGelb}{HTML}{FFCC00}
\colorlet{dGreen}{dunkelGruen}
\colorlet{dBlue}{dunkelBlau}
\colorlet{dRed}{dunkelRot}
\colorlet{dYellow}{dunkelGelb}

%%%%%%%%%%%%%%%%%%%%%%%%%%%%%%%%%%%%%%%%%%%%%%%%%%
%Farbliche Ausgaben:
%Parameter #1: Text oder Mathematische formel...
%z.B. : \gruen{Hallo Welt Test!}
%%%%%%%%%%%%%%%%%%%%%%%%%%%%%%%%%%%%%%%%%%%%%%%%%%

\newcommand{\yellow}[1]{\textcolor{dYellow}{#1}}
\newcommand{\gray}[1]{\textcolor{gray}{#1}}
\newcommand{\red}[1]{\textcolor{red}{#1}}
\newcommand{\green}[1]{\textcolor{green}{#1}}
\newcommand{\blue}[1]{\textcolor{blue}{#1}}
\newcommand{\dGreen}[1]{\textcolor{dGreen}{#1}}
\newcommand{\dBlue}[1]{\textcolor{dBlue}{#1}}
\newcommand{\dRed}[1]{\textcolor{dRed}{#1}}

%%%%%%%%%%%%%%%%%%%%%%%%%%%%%%%%%%%%%%%%%%%%%%%%
%Konfiguration für das darstellen von Quelltext
%%%%%%%%%%%%%%%%%%%%%%%%%%%%%%%%%%%%%%%%%%%%%%%%
\lstset
{
	language=Java, % oder C++, Pascal, {[77]Fortran}, ...
	numbers=left, % Position der Zeilennummerierung
	firstnumber=auto, % Erste Zeilennummer
	basicstyle=\ttfamily, % Textgröße des Standardtexts
	keywordstyle=\ttfamily\color{dRot}, % Formattierung Schlüsselwörter
	commentstyle=\ttfamily\color{dGruen}, % Formattierung Kommentar
	stringstyle=\ttfamily\color{dBlau}, % Formattierung Strings
	numberstyle=\tiny, % Textgröße der Zeilennummern
	stepnumber=1, % Angezeigte Zeilennummern
	numbersep=5pt, % Abstand zw. Zeilennummern und Code
	aboveskip=15pt, % Abstand oberhalb des Codes
	belowskip=11pt, % Abstand unterhalb des Codes
	captionpos=b, % Position der Überschrift
	xleftmargin=10pt, % Linke Einrückung
	frame=single, % Rahmentyp
	breaklines=true, % Umbruch langer Zeilen
	showstringspaces=false, % Spezielles Zeichen für Leerzeichen
	tabsize=2,
	texcl=true
}

%%%%%%%%
% Kopf
%%%%%%%%
\newcommand{\Header}[3]
{
	{\footnotesize \parindent0em
		{\sc Universität Konstanz}                \hfill #1 \\
		{\sc Fachbereich Informatik \& Informationswissenschaft} \hfill #2 \\
		#3 \hfill \today
	}
}

%%%%%%%%%%%%%%%%%%%%%%%
% load some java code
% \loadJava{file}
%%%%%%%%%%%%%%%%%%%%%%%
\newcommand{\loadJava}[1]
{
	\lstinputlisting[language=Java]{#1.java}
}

%%%%%%%%%%%%%%%%%%%%%%%
% load some cpp code
% \loadCpp{file.cpp}
%%%%%%%%%%%%%%%%%%%%%%%
\newcommand{\loadCpp}[1]
{
	\lstinputlisting[language=C++]{#1}
}

%%%%%%%%%%%%%%%%%%%%%%%%%%%%%
% load some code
% \loadCode{Python}{file.py}
%%%%%%%%%%%%%%%%%%%%%%%%%%%%%
\newcommand{\loadCode}[2]
{
	\lstinputlisting[language=#1]{#2}
}

%%%%%%%%%%%%%%%%%%%%%%%%%%%%%%%%%%%%%%%%%%%%%%%%%%%%%%%%%%%%%%%%%%%%%%%
% some symbols
%%%%%%%%%%%%%%%%%%%%%%%%%%%%%%%%%%%%%%%%%%%%%%%%%%%%%%%%%%%%%%%%%%%%%%%
\newcommand{\correct}{\green{\text{\ding{52}}}} %for use in text and math
\newcommand{\wrong}{\red{\text{\ding{56}}}} %for use in text
\newcommand{\tflash}{$\yellow{\lightning}$} %for use in text
\newcommand{\mflash}{\yellow{\lightning}} %for use in math
\newcommand{\follows}{$\Rightarrow$} %used so often...
\newcommand{\good}{\item[\dGreen{\ding{58}}]} %an item with a green plus as bullet point
\newcommand{\bad}{\item[\red{\ding{216}}]} %an item with a red downwards sloping arrow, maybe find something better
\newcommand{\hm}{\ensuremath{\overset{-\mkern-11mu-\mkern-3.5mu\rhook}{\smash{\odot}\rule{0ex}{.46ex}}\underline{\hspace{0.5em}}\overset{-\mkern-11mu-\mkern-3.5mu\rhook}{\smash{\odot}\rule{0ex}{.46ex}}}}

%%%%%% make emph bold instead of italic %%%%%
\makeatletter
\DeclareRobustCommand{\em}{%
  \@nomath\em \if b\expandafter\@car\f@series\@nil
  \normalfont \else \bfseries \fi}
\makeatother

%%%%%%%%%%%%%%%%%%%%%%%%%%%%%%%%%%%%%%
% languages for \loadCode
%ABAP		IDL				Plasm
%ACSL		inform			POV
%Ada		Java			Prolog
%Algol		JVMIS			Promela
%Ant		ksh				Python
%Assembler	Lisp			R
%Awk		Logo			Reduce
%bash		make			Rexx
%Basic		Mathematica1	RSL
%C			Matlab			Ruby
%C++		Mercury			S
%Caml		MetaPost		SAS
%Clean		Miranda			Scilab
%Cobol		Mizar			sh
%Comal		ML				SHELXL
%csh		Modula-2		Simula
%Delphi		MuPAD			SQL
%Eiffel		NASTRAN			tcl
%Elan		Oberon-2		TeX
%erlang		OCL				VBScript
%Euphoria	Octave			Verilog
%Fortran	Oz				VHDL
%GCL		Pascal			VRML
%Gnuplot	Perl			XML
%Haskell	PHP				XSLT
%HTML		PL/I
%%%%%%%%%%%%%%%%%%%%%%%%%%%%%%%%%%%%%%