\documentclass[a4paper, 10pt]{article}

%\usepackage{scalefnt}
%\usepackage{parcolumns}

\usepackage{newclude}
% Zum Einbinden in die Zusammenfassungs-files

\usepackage{amsmath,amsthm,amsfonts,amssymb} % Verbesserter Mathesatz
\usepackage{algorithm2e}
\usepackage{bigfoot} % komplexe Fußnotenapparate(Fußnoten in Fußnoten und andere Späße)%
\usepackage{colortbl}
\usepackage[T1]{fontenc} % normaler erweitere Zeichnesatz
\usepackage{framed, color}  % Ramenpaket für zum Einfügen von schönen Ramen
\usepackage{graphicx}
\usepackage{hyperref} %used for link to creative commons license
\usepackage{listings} %for code-listings (inkl. Tab-Styling)
\usepackage{marvosym}
\usepackage{marginnote}
\usepackage{microtype} % div. Verbesserungen des Schriftsatzes (Grauwert, opt. Randausgleich, Zeilenumbruch)
%\usepackage{multirow}
\usepackage{multicol} %use this with next line for vertical divided environment
%\setlength\columnseprule{.4pt}:465
\usepackage[ngerman]{babel} % Neue Rechtschreibung
\usepackage{pifont}
\usepackage[sans]{dsfont} %für alternative Mengensymbole
\usepackage{stmaryrd} %u.a. für \lightning
\usepackage{tikz} % für Diagramme(Dia!) und Bilder (z.B. *.eps)/für ER-Diagramme/für UML-Diagramme
%\usepackage{tikz-er2} %  er-diagramme
\usepackage{tikz-uml} % uml-Diagramme
\usepackage{units} % z.B. fuer \nicefrac{}{}
\usepackage[utf8]{inputenc} % utf8 für den Editor
\usepackage{wasysym} %u.a. für \lightning
\usepackage{xcolor}


\usetikzlibrary{shapes,decorations,arrows,fit,backgrounds} %Zum diversen zeichen%
%\usetikzlibrary{automata} % für den CFlipper, wenn es den soweit ist			%
\usetikzlibrary{positioning} % positionierung
%\usetikzlibrary{shadows} % fuer schoene schlagschatten
\usetikzlibrary{automata} % für Automaten

%%%%%%%%%%%%%%%%%%%%%%%%%%%
%  Formatierung der Seite
%%%%%%%%%%%%%%%%%%%%%%%%%%%
\usepackage{fancyhdr}
\pagestyle{fancy}		% für den footer										%
\renewcommand{\headrulewidth}{0pt} % damit oben kein dummer Strich kommt		%
\fancyhead{}
\topmargin -2cm 		% Oberer Rand											%
\textheight 25cm		% Texthöhe												%
\textwidth 16.0 cm		% Textbreite											%
\oddsidemargin -0.1cm 	% Warum?												%
\newcommand{\Gruppe}[2]
{
	\lfoot{#1}
	\rfoot{#2}
}
\colorlet{shadecolor}{gray!25} % Farbe für graue Box definieren
%%%%%%%%%%%%%%%%%%%%%%%%%%%%%%%%%%%%%%%%%%%%%%%%%%%%%%%%%%%%%%%%%%%%%%%%%%%%%%%%%
%Farben die definiert werden zum schreiben und zeichnen							%
%%%%%%%%%%%%%%%%%%%%%%%%%%%%%%%%%%%%%%%%%%%%%%%%%%%%%%%%%%%%%%%%%%%%%%%%%%%%%%%%%
\xdefinecolor{schwarz}{HTML}{000000}
\xdefinecolor{dunkelGruen}{HTML}{007D00}
\xdefinecolor{dunkelBlau}{HTML}{0000A0}
\xdefinecolor{dunkelRot}{HTML}{A00000}
\xdefinecolor{dunkelGelb}{HTML}{FFAA00}
\xdefinecolor{hellesGelb}{HTML}{FFCC00}
\colorlet{dGreen}{dunkelGruen}
\colorlet{dBlue}{dunkelBlau}
\colorlet{dRed}{dunkelRot}
\colorlet{dYellow}{dunkelGelb}

%%%%%%%%%%%%%%%%%%%%%%%%%%%%%%%%%%%%%%%%%%%%%%%%%%
%Farbliche Ausgaben:
%Parameter #1: Text oder Mathematische formel...
%z.B. : \gruen{Hallo Welt Test!}
%%%%%%%%%%%%%%%%%%%%%%%%%%%%%%%%%%%%%%%%%%%%%%%%%%

\newcommand{\yellow}[1]{\textcolor{dYellow}{#1}}
\newcommand{\gray}[1]{\textcolor{gray}{#1}}
\newcommand{\red}[1]{\textcolor{red}{#1}}
\newcommand{\green}[1]{\textcolor{green}{#1}}
\newcommand{\blue}[1]{\textcolor{blue}{#1}}
\newcommand{\dGreen}[1]{\textcolor{dGreen}{#1}}
\newcommand{\dBlue}[1]{\textcolor{dBlue}{#1}}
\newcommand{\dRed}[1]{\textcolor{dRed}{#1}}

%%%%%%%%%%%%%%%%%%%%%%%%%%%%%%%%%%%%%%%%%%%%%%%%
%Konfiguration für das darstellen von Quelltext
%%%%%%%%%%%%%%%%%%%%%%%%%%%%%%%%%%%%%%%%%%%%%%%%
\lstset
{
	language=Java, % oder C++, Pascal, {[77]Fortran}, ...
	numbers=left, % Position der Zeilennummerierung
	firstnumber=auto, % Erste Zeilennummer
	basicstyle=\ttfamily, % Textgröße des Standardtexts
	keywordstyle=\ttfamily\color{dRot}, % Formattierung Schlüsselwörter
	commentstyle=\ttfamily\color{dGruen}, % Formattierung Kommentar
	stringstyle=\ttfamily\color{dBlau}, % Formattierung Strings
	numberstyle=\tiny, % Textgröße der Zeilennummern
	stepnumber=1, % Angezeigte Zeilennummern
	numbersep=5pt, % Abstand zw. Zeilennummern und Code
	aboveskip=15pt, % Abstand oberhalb des Codes
	belowskip=11pt, % Abstand unterhalb des Codes
	captionpos=b, % Position der Überschrift
	xleftmargin=10pt, % Linke Einrückung
	frame=single, % Rahmentyp
	breaklines=true, % Umbruch langer Zeilen
	showstringspaces=false, % Spezielles Zeichen für Leerzeichen
	tabsize=2,
	texcl=true
}

%%%%%%%%
% Kopf
%%%%%%%%
\newcommand{\Header}[3]
{
	{\footnotesize \parindent0em
		{\sc Universität Konstanz}                \hfill #1 \\
		{\sc Fachbereich Informatik \& Informationswissenschaft} \hfill #2 \\
		#3 \hfill \today
	}
}

%%%%%%%%%%%%%%%%%%%%%%%
% load some java code
% \loadJava{file}
%%%%%%%%%%%%%%%%%%%%%%%
\newcommand{\loadJava}[1]
{
	\lstinputlisting[language=Java]{#1.java}
}

%%%%%%%%%%%%%%%%%%%%%%%
% load some cpp code
% \loadCpp{file.cpp}
%%%%%%%%%%%%%%%%%%%%%%%
\newcommand{\loadCpp}[1]
{
	\lstinputlisting[language=C++]{#1}
}

%%%%%%%%%%%%%%%%%%%%%%%%%%%%%
% load some code
% \loadCode{Python}{file.py}
%%%%%%%%%%%%%%%%%%%%%%%%%%%%%
\newcommand{\loadCode}[2]
{
	\lstinputlisting[language=#1]{#2}
}

%%%%%%%%%%%%%%%%%%%%%%%%%%%%%%%%%%%%%%%%%%%%%%%%%%%%%%%%%%%%%%%%%%%%%%%
% some symbols
%%%%%%%%%%%%%%%%%%%%%%%%%%%%%%%%%%%%%%%%%%%%%%%%%%%%%%%%%%%%%%%%%%%%%%%
\newcommand{\correct}{\green{\text{\ding{52}}}} %for use in text and math
\newcommand{\wrong}{\red{\text{\ding{56}}}} %for use in text
\newcommand{\tflash}{$\yellow{\lightning}$} %for use in text
\newcommand{\mflash}{\yellow{\lightning}} %for use in math
\newcommand{\follows}{$\Rightarrow$} %used so often...
\newcommand{\good}{\item[\dGreen{\ding{58}}]} %an item with a green plus as bullet point
\newcommand{\bad}{\item[\red{\Emailct}]} %better icon for bad items
\newcommand{\note}[1]{\red{\marginnote{#1}}} %add a red margin note
\newcommand{\fitem}{\item[\follows]} %items with a follows arrow
\newcommand{\hm}{\ensuremath{\overset{-\mkern-11mu-\mkern-3.5mu\rhook}{\smash{\odot}\rule{0ex}{.46ex}}\underline{\hspace{0.5em}}\overset{-\mkern-11mu-\mkern-3.5mu\rhook}{\smash{\odot}\rule{0ex}{.46ex}}}}

%%%%%% make emph bold instead of italic %%%%%
\makeatletter
\DeclareRobustCommand{\em}{%
  \@nomath\em \if b\expandafter\@car\f@series\@nil
  \normalfont \else \bfseries \fi}
\makeatother

%%%%%%%%%%%%%%%%%%%%%%%%%%%%%%%%%%%%%%
% languages for \loadCode
%ABAP		IDL				Plasm
%ACSL		inform			POV
%Ada		Java			Prolog
%Algol		JVMIS			Promela
%Ant		ksh				Python
%Assembler	Lisp			R
%Awk		Logo			Reduce
%bash		make			Rexx
%Basic		Mathematica1	RSL
%C			Matlab			Ruby
%C++		Mercury			S
%Caml		MetaPost		SAS
%Clean		Miranda			Scilab
%Cobol		Mizar			sh
%Comal		ML				SHELXL
%csh		Modula-2		Simula
%Delphi		MuPAD			SQL
%Eiffel		NASTRAN			tcl
%Elan		Oberon-2		TeX
%erlang		OCL				VBScript
%Euphoria	Octave			Verilog
%Fortran	Oz				VHDL
%GCL		Pascal			VRML
%Gnuplot	Perl			XML
%Haskell	PHP				XSLT
%HTML		PL/I
%%%%%%%%%%%%%%%%%%%%%%%%%%%%%%%%%%%%%%

\begin{document}

\Gruppe{Stephan Heidinger}{fses 09}
\Header{Functional Safety in Embedded Systems}{Session 09 - Formal Methods for Functional Safety I}

\section*{Formal Methods for Safety Assessment}
\begin{itemize}
    \item formal method \follows use of formalisms and techniques (based on mathematics) for specification, design, analysis
    \item formalized methods: several notations, different sorts of diagrams, pseudocode, formality, but no formal foundation
\end{itemize}

\subsection*{Advantages of Formal Methods}
\begin{itemize}
    \item growing complexity of computer-based system
    \item remove ambiguities in specification of requirements and models
    \item \emph{verification} checking, whether a system correctly implements requirements
    \item \emph{validation} check, whether the started requirements are the intended ones
    \item traditional techniques: testing, simulation (not exhaustive)
    \item formal methods: exhaustive, automation possible
\end{itemize}

\subsection*{Formal Methods in the Development Process}
\begin{itemize}
    \item formal methods can have high level ob abstraction
    \item Rushby's four different levels:
    \begin{description}
        \item[Level 0] \emph{no use of formal methods} standard practice
        \item[Level 1] \emph{use of concepts and notation from discrete mathematics} concepts and notations from discrete mathematics and mathematical logic
        \item[Level 2] \emph{use of formalized specification languages with some mechanized support tools} formal notations with fixed syntax, at least limited tool support, proof performed manually
        \item[Level 3] \emph{Use of fully formal specification languages with comprehensive support environments, including mechanized theorem proving or proof checking} fully formal specification languages, supported by tools, automatic generation of proofs, proofs are fully formalized
    \end{description}
    \item lower levels of rigor: not so safety relevant
    \item high levels of rigor \follows fully safety critical systems
    \item formal methods need not be applied everywhere \follows overall correctness not guaranteed
    \item apply formal methods early or late?
    \begin{description}
        \item[early] specification bugs are the worst, work better at early stages than testing and simulation (which work good at code/ gate level)
        \item[late] completed system will be verified
    \end{description}
    \item limit formal methods to certain subsystems
    \item formal methods can be used for \emph{certification} of safety-critical systems
\end{itemize}

\subsection*{Problems and Limitations}
\begin{itemize}
    \bad formal methods are expensive
    \good formal methods can find bugs in very early stages \follows save money in long run
    \bad difficult to use, requires expertise
    \bad maximum benefits only, if applied in every stage of development process
    \good use lightweight formal methods (produce specifications of system, although no full coverage)
    \bad single formalism not suited for whole process, use different formalisms \follows may cause problems when verification has to be assembled
    \bad not fully automatic, problems often huge, computing power not enough
    \bad formal languages are often hard to understand
    \bad proof (by theorem prover/model checker may not be enough to convince certification authority)
    \good counterexamples
    \bad verifier may produce wrong results
\end{itemize}

\subsection*{Formal Models and Specifications}
\begin{itemize}
    \item categories for formal specification languages
    \begin{description}
        \item[algebraic languages]
        \item[model-based languages]
        \item[process algebras and calculi]
        \item[logic-based languages]
    \end{description}
    \item
    \begin{description}
        \item[property-oriented] modeling in terms of properties, that system must satisfy
        \item[model-oriented] mathematical model corresponding to particular implementation, closer to final implementation {\tiny (are used in later stages of design)}
    \end{description}
\end{itemize}

\subsubsection{Algebraic Specification Languages}
\begin{itemize}
    \item characterize objects to be specified in terms of algebraic relationships between them
    \item programs are many sorted algebras (collections of data, operations over data)
    \item typically property oriented
\end{itemize}

\subsubsection*{Model-Based Specification Languages}
\begin{itemize}
    \item model-based style of specification, explicit mathematical model of system state and corresponding operations
    \item state consists of set of external variables, preconditions, postconditions
    \item Z, VDM (Vienna Development Method), Larch, Alloy
    \item \emph{Z} based on typed set theory + first-order predicate logic, specification via set of schemas, typically based on a hierarchy of schemas
    \item related to Z \follows B language and method
    \begin{itemize}
        \item a system is modeled as set of interdependent abstract machines
    \end{itemize}
\end{itemize}

\subsubsection*{Process Algebras and Calculi}
\begin{itemize}
    \item set of logical formalisms \follows describe concurrent and distributed systems, explicit model interaction, inter-process communication, synchronization
    \item behavior is described using axiomatic approach and formalized terms (algebraic laws)
    \item \emph{bisumulation equivalence} \follows formal reasoning about equivalence between processes
    \item CSP (Communicating Sequential Processes), CCS (Calculus of Communicating Systems), ACP (Algebra of Communicating Processes), $\pi$-calculus, ambient-calculus, LOTOS
\end{itemize}

\subsubsection*{Logic-Based Languages}
\begin{itemize}
    \item logic in broad sense to describe systems and system properties
    \item theorem provers, temporal (LTL, CTL) and interval logic (e.g. Duration Calculus)
    \item state machines
    \begin{description}
        \item[moore machines] output depends only on state
        \item[mealy machines] output depends on state and input
    \end{description}
    \item \emph{flowcharts} explicitly model program statements and control flow, hoare approach, dijkstras weakest transitions
    \item \emph{petri nets}
    \item \emph{kripke structures}
\end{itemize}

\subsubsection*{State Transition Systems}
\begin{shaded}
    $\mathcal{P}$ is a set of proposition, state transition system is tuple $\langle \mathcal{S}, \mathcal{I}, \mathcal{R}, \mathcal{L}\rangle$
    \begin{itemize}
        \item $\mathcal{S}$ finite set of states
        \item $\mathcal{I} \subseteq \mathcal{S}$ set of initial states
        \item $\mathcal{R} \subseteq \mathcal{S} \times \mathcal{S}$ transition relation
        \item $\mathcal{L}:\mathcal{S}\longrightarrow 2^\mathcal{P}$
    \end{itemize}
\end{shaded}
\begin{itemize}
    \item \emph{trace} start in $\mathcal{I}$, repetedly append states reachable through $\mathcal{R}$
    \item \emph{paths} infinite traces
    \item state $s$ is \emph{reachable}, if there exists a path from $\mathcal{I}$
    \item \emph{NuSMV}
    \begin{itemize}
        \item combination of components via
        \begin{description}
            \item[synchronous composition] (default in NuSMV), components evolve in parallel, perform transitions simultaneously
            \item[asynchronous composition] interleaving model for evolution of component
        \end{description}
    \end{itemize}
\end{itemize}

\subsubsection*{Temporal Logic}
\begin{itemize}
    \item express properties of reactive systems modeled as Kripke structures
    \begin{description}
        \item[safety properties] nothing bad ever happens
        \item[liveness properties] something desirable will eventually happen
    \end{description}
    \item most common: LTL (linear time temporal logic), CTL (computation tree logic)
    \item \emph{PSL} (Property Specification Language), high level language for expressing temporal requirements
    \begin{itemize}
        \item origins in the hardware domain
        \item superset of CTL and LTL
    \end{itemize}
\end{itemize}

\subsection*{Formal Methods for Verification and Validation}
\subsubsection*{Testing and Simulation}
\begin{itemize}
    \item testing and simulation \follows commonly used for verification and validation
    \item model must be executable
    \begin{description}
        \item[statically] execution of model
        \item[dynamically] operation of real system
    \end{description}
    \item \emph{test-oracle} predict what system should do \follows from system requirements
\end{itemize}
\begin{description}
    \item[simulation] use a model
    \begin{itemize}
        \item environmental simulation: simulate environment
        \item good, when testing with environment is not possible (impossible: space flight, dangerous: nuclear power plant)
    \end{itemize}
    \item[forms of testing]
    \begin{description}
        \item[static testing] test without operating the system
        \begin{description}
            \item[walk-throughs, design reviews]  a form of code inspection, typically by peer-review
            \item[fagan inspections] systematic methodology to find defects and omissions in development process
            \item[formal proofs]
            \item[type analysis] statically analyze type information, ensure that type errors don't occur at run-time
            \item[control flow analysis] analyze control structure of program, formalized as graph
            \item[data flow analysis] analyze flow of data of program, analyze operations on data, different forms of static analysis
            \item[symbolic execution] run program using symbolic inputs, compare with expected result
            \item[pointer analysis] analyze how program accesses pointers or heaps \follows memory leaks
            \item[shape analysis] generalization of pointer analysis \follows determine information about heap-allocated data structures, also produces information for debugging
        \end{description}
        \item[dynamic testing] actual execution of system
        \begin{description}
            \item[behavioral testing] intended to evaluate behavior of system form perspective of external user
            \begin{description}
                \item[functional testing] evaluates compliance with functional requirements
                \item[nonfunctional testing] evaluates nonfunctional characteristics (performance, reliability, safety)
            \end{description}
            \item[structural testing] test various routines and different execution paths, requires knowledge of internal implementation \follows white-box testing
            \item[random testing] random choice about test cases
        \end{description}
        \item[black-box testing] tester has no knowledge about internal implementation of system, also \emph{requirement based testing}
        \item[white-box testing] performed by test-engineers, knowledge of internal implementation
    \end{description}
    \item[test vector] set of inputs to be applied
    \item[test coverage]
    \begin{description}
        \item[coverage-based testing] plan test activity according to test coverage
        \item[requirements-based coverage] for how many requirements has been tested
        \item[equivalence partitioning] partition test vectors into equivalence classes \follows similar, qualitative behaviors (black box view)
        \item[boundary analysis] test vectors at extreme boundaries of equivalence class (at either side of boundary)
        \item[state-transition testing] stimulate each transition of system \follows finite state machine
        \item[structure based coverage] based on data-flow or control flow
        \begin{description}
            \item[statement coverage] measures portion of statements to be executed (often minimal requirement for coverage testing)
            \item[branch coverage] measure coverage of paths
            \item[call-graph coverage] each possible invocation tree of subroutines
        \end{description}
    \end{description}
\end{description}

\subsubsection*{Theorem Proving}
\begin{itemize}
    \item system and properties are expressed in common language
    \item based on formal system \follows verify a property $\equiv$ general proof
    \item generate lemmas as intermediate proofs
    \item different theorem provers are based on different inference rules
    \begin{description}
        \item[automated theorem provers] fully automated
        \item[interactive theorem provers] require human guidance
        \item[proof checkers] no proof construction, verify existing proof to be valid
    \end{description}
\end{itemize}

\end{document}
