\documentclass{scrartcl}

\usepackage{Header}

%\usepackage{scalefnt}
%\usepackage{parcolumns}
%\usepackage{tikz-er2}
\usepackage{hyperref}

\begin{document}


\Gruppe{Stephan Heidinger}{TI - Zusammenfassung v0.1}
\Header{Theoretische Grundlagen der Informatik}{Sommersemester 2012}{Stephan Heidinger}{}%leave last variable empty, else there will be aufgabenblatt überschrift

\begin{shaded}
Dieses Dokument wurde unter der Creative Commons - Namensnennung-NichtKommerziell-Weitergabe unter gleichen Bedingungen (\textbf{CC by-nc-sa}) veröffentlicht. Die Bedingungen finden sich unter \href{http://creativecommons.org/licenses/by-nc-sa/3.0/de}{diesem Link}. \\
\centerline{\includegraphics[scale=1]{../cc-by-nc-sa.png} }
\end{shaded}

\textit{Find any errors? Please send them back, I want to keep them!}

\section*{Allgemeines}
\subsection*{Grammatik}
$G=(V,\Sigma,P,S)$ \\
\begin{description}
\item[V -] Variblen
\item[$\Sigma$ -] Terminalalphabet
\item[P -] Regeln/Produktionen
\item[S -] Startvariable
\end{description}

\subsection*{Chromsky-Hirarchie}
\begin{description}
\item[Typ 0:] (Phrasenstrukturgrammatik) - keine Einschränkungen
\item[Typ 1:] (kontextsensitiv) - $(w_1\to w_2) \Rightarrow (\vert w_1 \vert \leq \vert w_2 \vert)$ {\tiny (Wort wird nicht kürzer)}
\item[Typ 2:] (kontextfrei) - $(w_1\to w_2 \Rightarrow (w_1\in V)$ {\tiny $w_1$ ist einzelne Variable}
\item[Typ 3:] (regulär) - $w_2\in\Sigma\cup\Sigma V$ {\tiny "`rechte Seiten"' von Regeln Terminalsymbol oder Terminalsymbole gefolgt von Variablen}
\end{description}
Alle Sprachen der Typen 1,2 und 3 sind \emph{entscheidbar}.

\subsection*{$\varepsilon$-Sonderregelung {\tiny (Zulassen des leeren Wortes $\varepsilon$ in Typ 1,2 oder 3)}}
\begin{itemize}
\item Regel hinzufügen: $S\to\varepsilon$
\item Verhindern von $S$ auf rechter Seite von Regeln: Regel mit "`$\to S$"' ersetzen durch "`$\to S'$"'
\item Zulassen von $A\to\varepsilon$ {\tiny (verändert Sprache nicht)}\\
Algorithmus:
\begin{enumerate}
\item Zerlege $V\to V_1,V_2$, $(A\Rightarrow^*\varepsilon)\in V_1$ und $V_1\cap V_2=\emptyset$.
\item Entferne alle $A\to\varepsilon$, füge für $(B\to xAy)$ $(B\to xy)$ hinzu.
\end{enumerate}
\end{itemize}

\subsection*{Wortproblem {\tiny (Gehört ein Wort zu einer Sprache?)}}
$(\exists \textrm{Algorithmus})[(\textrm{Algo terminiert in endl. Zeit}\wedge(Algo entscheidet (x\in\mathcal{L}(G))\vee(x\not\in\mathcal{L}(G)))]$ \\
$\Rightarrow$ das Wortproblem ist für Typ 1,2 und 3 entscheidbar (aber NP-hart für Typ 1)

\subsection*{Syntaxbäume}
Wurzel: $S$ \\
Für $i=1,2,\ldots,n\; A\to z\in P\Rightarrow\vert z\vert\textrm{ viele Söhne }\to \textrm{"`weitere Kette"'}$\\
\begin{description}
\item[Linksableitung:] Variable am weitesten links wird abgeleitet.
\item[Rechtsableitung:] Variable am weitesten rechts wird abgeleitet.
\item[mehrdeutige Grammatik:] für ein $x$ verschiedene Syntaxbäume möglich
	\begin{itemize}
	\item Mehrdeutigkeit kann oft beseitigt werden.
	\item Ist dies nicht möglich $\Rightarrow$ \emph{inhärent mehrdeutig}
	\end{itemize}
\end{description}

\subsection*{Backus-Naur-Form Bnf (Typ 2 Grammatiken)}
Metaregeln für selbe linke Seite
\[
\left. \begin{array}{ccc}
A & \to    & \beta_1 \\ 
A & \to    & \beta_2 \\ 
  & \vdots &         \\ 
A & \to    & \beta_3
\end{array} \right\rbrace A\to\beta_1\vert\beta_2\vert\ldots\beta_n
\]

\subsection*{erweiterte Backus-Naur-Form Ebnf}
\begin{align*}
A &\to \alpha[\beta]\gamma \Rightarrow \left\lbrace
\begin{array}{ccc}
A & \to & \alpha\gamma \\
A & \to & \alpha\beta\gamma
\end{array}
\right. \\
A &\to \alpha\{\beta\}\gamma \Rightarrow \left\lbrace
\begin{array}{ccc}
A & \to & \alpha\gamma \\
A & \to & \alpha B\gamma \\
B & \to & \beta \\
B & \to & \beta B
\end{array}
\right.
\end{align*}

\section*{Reguläre Sprachen}

\end{document}
