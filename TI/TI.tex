\documentclass{scrartcl}

% \usepackage{Header}
\usepackage{newclude}
% Zum Einbinden in die Zusammenfassungs-files

\usepackage{amsmath,amsthm,amsfonts,amssymb} % Verbesserter Mathesatz
\usepackage{bigfoot} % komplexe Fußnotenapparate(Fußnoten in Fußnoten und andere Späße)%
\usepackage{colortbl}
\usepackage[T1]{fontenc} % normaler erweitere Zeichnesatz
\usepackage{framed, color}  % Ramenpaket für zum Einfügen von schönen Ramen
\usepackage{graphicx}
\usepackage{hyperref} %used for link to creative commons license
\usepackage{listings} %for code-listings (inkl. Tab-Styling)
\usepackage{marginnote}
\usepackage{microtype} % div. Verbesserungen des Schriftsatzes (Grauwert, opt. Randausgleich, Zeilenumbruch)
%\usepackage{multirow}
\usepackage[ngerman]{babel} % Neue Rechtschreibung
\usepackage[sans]{dsfont} %für alternative Mengensymbole
\usepackage{stmaryrd} %u.a. für \lightning
\usepackage{tikz,tikz-er2,tikz-uml} % für Diagramme(Dia!) und Bilder (z.B. *.eps)/für ER-Diagramme/für UML-Diagramme
\usepackage{units} % z.B. fuer \nicefrac{}{}
\usepackage[utf8]{inputenc} % utf8 für den Editor
\usepackage{wasysym} %u.a. für \lightning
\usepackage{xcolor}


\usetikzlibrary{shapes,decorations,arrows,fit,backgrounds} %Zum diversen zeichen%
%\usetikzlibrary{automata} % für den CFlipper, wenn es den soweit ist			%
\usetikzlibrary{positioning} % positionierung 
%\usetikzlibrary{shadows} % fuer schoene schlagschatten

%%%%%%%%%%%%%%%%%%%%%%%%%%%
%  Formatierung der Seite
%%%%%%%%%%%%%%%%%%%%%%%%%%%
\usepackage{fancyhdr}
\pagestyle{fancy}		% für den footer										%
\renewcommand{\headrulewidth}{0pt} % damit oben kein dummer Strich kommt		%
\fancyhead{}
\topmargin -2cm 		% Oberer Rand											%
\textheight 25cm		% Texthöhe												%
\textwidth 16.0 cm		% Textbreite											%
\oddsidemargin -0.1cm 	% Warum?												%
\newcommand{\Gruppe}[2]
{
	\lfoot{#1}
	\rfoot{#2}
}
\colorlet{shadecolor}{gray!25} % Farbe für graue Box definieren
%%%%%%%%%%%%%%%%%%%%%%%%%%%%%%%%%%%%%%%%%%%%%%%%%%%%%%%%%%%%%%%%%%%%%%%%%%%%%%%%%
%Farben die definiert werden zum schreiben und zeichnen							%
%%%%%%%%%%%%%%%%%%%%%%%%%%%%%%%%%%%%%%%%%%%%%%%%%%%%%%%%%%%%%%%%%%%%%%%%%%%%%%%%%
\xdefinecolor{schwarz}{HTML}{000000}
\xdefinecolor{dunkelGruen}{HTML}{007D00}
\xdefinecolor{dunkelBlau}{HTML}{0000A0}
\xdefinecolor{dunkelRot}{HTML}{A00000}
\xdefinecolor{dunkelGelb}{HTML}{FFAA00}
\xdefinecolor{hellesGelb}{HTML}{FFCC00}
\colorlet{dGreen}{dunkelGruen}
\colorlet{dBlue}{dunkelBlau}
\colorlet{dRed}{dunkelRot}
\colorlet{dYellow}{dunkelGelb}

%%%%%%%%%%%%%%%%%%%%%%%%%%%%%%%%%%%%%%%%%%%%%%%%%%
%Farbliche Ausgaben:
%Parameter #1: Text oder Mathematische formel...
%z.B. : \gruen{Hallo Welt Test!}
%%%%%%%%%%%%%%%%%%%%%%%%%%%%%%%%%%%%%%%%%%%%%%%%%%

\newcommand{\yellow}[1]{\textcolor{dYellow}{#1}}
\newcommand{\gray}[1]{\textcolor{gray}{#1}}
\newcommand{\red}[1]{\textcolor{red}{#1}}
\newcommand{\green}[1]{\textcolor{green}{#1}}
\newcommand{\blue}[1]{\textcolor{blue}{#1}}
\newcommand{\dGreen}[1]{\textcolor{dGreen}{#1}}
\newcommand{\dBlue}[1]{\textcolor{dBlue}{#1}}
\newcommand{\dRed}[1]{\textcolor{dRed}{#1}}

%%%%%%%%%%%%%%%%%%%%%%%%%%%%%%%%%%%%%%%%%%%%%%%%
%Konfiguration für das darstellen von Quelltext
%%%%%%%%%%%%%%%%%%%%%%%%%%%%%%%%%%%%%%%%%%%%%%%%
\lstset
{
	language=Java, % oder C++, Pascal, {[77]Fortran}, ...
	numbers=left, % Position der Zeilennummerierung
	firstnumber=auto, % Erste Zeilennummer
	basicstyle=\ttfamily, % Textgröße des Standardtexts
	keywordstyle=\ttfamily\color{dRot}, % Formattierung Schlüsselwörter
	commentstyle=\ttfamily\color{dGruen}, % Formattierung Kommentar
	stringstyle=\ttfamily\color{dBlau}, % Formattierung Strings
	numberstyle=\tiny, % Textgröße der Zeilennummern
	stepnumber=1, % Angezeigte Zeilennummern
	numbersep=5pt, % Abstand zw. Zeilennummern und Code
	aboveskip=15pt, % Abstand oberhalb des Codes
	belowskip=11pt, % Abstand unterhalb des Codes
	captionpos=b, % Position der Überschrift
	xleftmargin=10pt, % Linke Einrückung
	frame=single, % Rahmentyp
	breaklines=true, % Umbruch langer Zeilen
	showstringspaces=false, % Spezielles Zeichen für Leerzeichen
	tabsize=2,
	texcl=true
}

%%%%%%%%
% Kopf
%%%%%%%%
\newcommand{\Header}[3]
{
	{\footnotesize \parindent0em
		{\sc Universität Konstanz}                \hfill #1 \\
		{\sc Fachbereich Informatik \& Informationswissenschaft} \hfill #2 \\
		#3 \hfill \today
	}
}

%%%%%%%%%%%%%%%%%%%%%%%
% load some java code
% \loadJava{file}
%%%%%%%%%%%%%%%%%%%%%%%
\newcommand{\loadJava}[1]
{
	\lstinputlisting[language=Java]{#1.java}
}

%%%%%%%%%%%%%%%%%%%%%%%
% load some cpp code
% \loadCpp{file.cpp}
%%%%%%%%%%%%%%%%%%%%%%%
\newcommand{\loadCpp}[1]
{
	\lstinputlisting[language=C++]{#1}
}

%%%%%%%%%%%%%%%%%%%%%%%%%%%%%
% load some code
% \loadCode{Python}{file.py}
%%%%%%%%%%%%%%%%%%%%%%%%%%%%%
\newcommand{\loadCode}[2]
{
	\lstinputlisting[language=#1]{#2}
}

%%%%%%%%%%%%%%%%%%%%%%%%%%%%%%%%%%%%%%%%%%%%%%%%%%%%%%%%%%%%%%%%%%%%%%%
% some symbols
%%%%%%%%%%%%%%%%%%%%%%%%%%%%%%%%%%%%%%%%%%%%%%%%%%%%%%%%%%%%%%%%%%%%%%%
\newcommand{\correct}{\gruen{\text{\ding{52}}}} %for use in text and math
\newcommand{\wrong}{\rot{\text{\ding{56}}}} %for use in text
\newcommand{\tflash}{$\hellgelb{\lightning}$} %for use in text
\newcommand{\mflash}{\hellgelb{\lightning}} %for use in math
\newcommand{\follows}{$\Rightarrow$} %used so often...

%%%%%%%%%%%%%%%%%%%%%%%%%%%%%%%%%%%%%%
% languages for \loadCode
%ABAP		IDL				Plasm
%ACSL		inform			POV
%Ada		Java			Prolog
%Algol		JVMIS			Promela
%Ant		ksh				Python
%Assembler	Lisp			R
%Awk		Logo			Reduce
%bash		make			Rexx
%Basic		Mathematica1	RSL
%C			Matlab			Ruby
%C++		Mercury			S
%Caml		MetaPost		SAS
%Clean		Miranda			Scilab
%Cobol		Mizar			sh
%Comal		ML				SHELXL
%csh		Modula-2		Simula
%Delphi		MuPAD			SQL
%Eiffel		NASTRAN			tcl
%Elan		Oberon-2		TeX
%erlang		OCL				VBScript
%Euphoria	Octave			Verilog
%Fortran	Oz				VHDL
%GCL		Pascal			VRML
%Gnuplot	Perl			XML
%Haskell	PHP				XSLT
%HTML		PL/I
%%%%%%%%%%%%%%%%%%%%%%%%%%%%%%%%%%%%%%

%\usepackage{scalefnt}
%\usepackage{parcolumns}
%\usepackage{tikz-er2}
\usepackage{hyperref}

\begin{document}


\Gruppe{Stephan Heidinger}{TI - Zusammenfassung v0.1}
\Header{Theoretische Grundlagen der Informatik}{Sommersemester 2012}{Stephan Heidinger}{}%leave last variable empty, else there will be aufgabenblatt überschrift

\begin{shaded}
Dieses Dokument wurde unter der Creative Commons - Namensnennung-NichtKommerziell-Weitergabe unter gleichen Bedingungen (\textbf{CC by-nc-sa}) veröffentlicht. Die Bedingungen finden sich unter \href{http://creativecommons.org/licenses/by-nc-sa/3.0/de}{diesem Link}. \\
\centerline{\includegraphics[scale=1]{../cc-by-nc-sa.png} }
\end{shaded}

\textit{Find any errors? Please send them back, I want to keep them!}

\section*{Allgemeines}
\subsection*{Definitionen}
\begin{description}
    \item[$\bullet$] $\mathds{N}^0 = \{0,1,2,\dots\}$
    \item[$\bullet$] $\mathds{N}^{\backslash\{0\}}=\{1,2,\dots\}$
    \item[Alphabet:] endliche Menge von Zeichen {\tiny $\Sigma=\{a,b\}$}
    \item[Wort] Zeichenfolge {\tiny $w=abab \in\Sigma^*$}
    \item[$\Sigma^*$:] Menge der Wörter
    \item[$\Sigma^+$:] $\Sigma^* \backslash\{\varepsilon\}$ {\tiny $\{\}\not= \varepsilon \not=\{\varepsilon\}$}
    \item[Operator $\circ$:]Konkatenation {\tiny $\varepsilon \circ w = w, w\circ w=ww=w^2$}
    \item[$w^n=\underset{n-mal}{\underbrace{www\dots w}}$] $n \in \mathds{N}^0$
    \item[$|w_1|$:] Wortlänge {\tiny $|w|=5$}
    \item[$|w|_\sigma, \sigma\in\Sigma$] Anzahl $\sigma$ in $\Sigma$ {\tiny $|w|_a=3$}
\end{description}

Seiten $A,B$ Mengen (Sprachen): {\tiny $A=\{\varepsilon, a, bba, bbabb, B=\{a,aa,aaa\}$}
\begin{description}
    \item[$|A|$] Mächtigkeit = Anzahl der Elemente
    \item[Sprache:] Teilmenge von $\Sigma^*$
    \item[Vereinigung] $A\cup B=\{x\in\Sigma^* | x\in A \vee x\in B\}$
    \item[Schnitt] $A\cap B=\{c\in\Sigma^* | x\in A \wedge x\in B\}$
    \item[Komplement] $\overline{A}=\{x\in\Sigma^* | x \not\in A\} = \Sigma^*\backslash A$
    \item[Produkt] $AB=\{xy\in\Sigma^* | x\in A \wedge y\in B\}$
\end{description}

\subsubsection*{De Morgan Regeln}
\begin{align*}
\overline{A \cup B} &= \overline{\overline{A}\cap \overline{B}} \\
\overline{A\cap B} &= \overline{\overline{A} \cup \overline{B}}
\end{align*}

\subsubsection*{Klassen}
Sei $\mathcal{C}$ eine Klasse von Sprachen. $\mathcal{C}$ heisst
\begin{tabular}[t]{|c|}
vereinigungs- \\
schnitt- \\
komplement- \\
produkt-
\end{tabular}
abgeschlossen $\Leftrightarrow A\in\mathcal{C} \wedge B\in\mathcal{C}$
\begin{tabular}[t]{|c|}
$A\cup B \in \mathcal{C}$ \\
$A\cap B \in \mathcal{C}$ \\
$\overline{A} \in \mathcal{C}$ \\
$AB \in \mathcal{C}$
\end{tabular}. \\
$A$ ist abgeschlossen gegen Vereinigung und Komplement $\Leftrightarrow A$ ist abgeschlossen gegen Schnitt und Komplement.
\begin{align*}
    A^0 &= \{\varepsilon\} \\
    A^1 &= A\\
    A^{n+q} &= A^nA\\
    A^i A^j &= A^{i+j}\\
    \left( A^i \right)^j &= A^{i \cdot j}\\
    A^* &= \underset{n\geq0}{\bigcup}A^n &= A^0\cup A^1\cup A^2\cup\dots \\
    A^+ &= \underset{n > 0}{\bigcup}A^n &= \phantom{A^0\cup\;}A^1\cup A^2\cup\dots \\
        &= \underset{n\geq0}{\bigcap}A^n &= A^0\cap A^1\cap A^2\cap\dots \\
    \textrm{Potenzmenge} &= 2^A = \{B | B\subseteq A\} \\
    M_a &= \{x\in\Sigma^* | x=n^a\} \\
    M_2 &= \{1,4,9,16,\dots\} \\
    M_3 &= \{1,8,27,64,\dots\} \\
    M_4 &= \{1,16,81\dots\} \\
    \bigcup^4_{i=2}M_i &= \{1,4,8,9,16,\dots\} \\
    \bigcap^4_{i=2}M_i &= \{1^12,2^12,3^12,\dots\} \\
    M &= \{1,2,3\} \\
    2^M &= \left\{\emptyset, \{1\}, \{2\}, \{3\}, \{1,2\}, \{1,3\}, \{2,3\}, \{1,2,3\} \right\}
\end{align*}

Kreuzprodukt: \\
Seien $A_i$ Mengen,
\begin{align*}
A_1 \times A_2 \times A_3 \times \dots \times A_n &= \{ \left(a_1,a_2,a_3,\dots,a_n\right) | a_i\in A_i, i\in \mathds{N}^n\} \\
\chi^n_i\leq1 A_i &=
\begin{cases}
\emptyset & i<1 \\
A_1 & n=1 \\
\chi^{n-1}_{i=1} A_i \times A_n & n>1
\end{cases}
\end{align*}

\subsubsection*{Relationen}
Relationen, $\tau_{A_i} \subseteq \chi^n_{i=1} A_i$ setzen Elemente von Mengen zueinander in Beziehung. \\
Bsp:
\begin{align*}
    \{a,b,\dots,z\} \tau \{1,2,\dots,26\} &= \{(a,1),(b,2),\dots,(z,26)\} \\
    \mathds{N}^0 \leq \mathds{N}^0 &= \{(a,b) | a+c=b,\; a,b,c\in \mathds{N}^0\}
\end{align*}
Eigenschaften von Relationen auf gleichen Mengen $A$:
\begin{description}
    \item[reflexiv:] $\forall a\in A: a\tau a$
    \item[irreflexiv:]$\forall a\in A: \overline{a\tau a}$
    \item[symmetrisch:]$\forall a,b\in A: a\tau b \Rightarrow b\tau a$
    \item[antisymmetrisch:] $\forall a,b\in A: \left(a\tau b\wedge b\tau a\right) \Rightarrow a=b$ {\tiny (antisymmetrisch $\Rightarrow$ reflexiv)}
    \item[asymmetrisch:] $\forall a,b\in A: a\tau b \Rightarrow \overline{b\tau a}$
    \item[transitiv:] $\forall a,b,c\in A: \left(a\tau b\wedge b\tau c\right) \Rightarrow a\tau c$
    \item[äquivalent:]
    \item[Ordnungsrelationen Typ "`$\leq$"':] reflexiv, transitiv, antisymmetrisch
    \item[Ordnungsrelationen Typ "`$<$"':] irreflexiv, transitiv, antisymmetrisch
    \item[$\tau^+$] transitive Hülle
    \item[$\tau^*$] reflexive, transitive Hülle
\end{description}
\paragraph{Satz:}
$\tau^*$ ist die kleinste, reflexive und transitive Relation (Hülle), die $\tau$ selbst umfasst.
\begin{align*}
geg \; A = \{1,2,3,4,5\} \\
& & R,S &\subseteq A\times A \\
& & R &= \{(1,2), (2,3), (2,5), (3,4), (5,4)\} \\
& & S &= \{(1,3), (3,5), (5,1)\} \\
ges \; R^+, S^+, R^*, S^* \\
& & R^+ &= R \cup   \{(1,3), (1,4), (2,4), (1,5)\} \\
& & R^* &= R^+ \cup \{(1,1),(2,2),(3,3),(5,5) \} \\
& & S^+ &= S^+ \cup \{(1,5),(1,1),(3,1),(3,3),(4,5),(5,3) \} \\
& & S^* &= S^* \cup \{(2,2),(4,4) \} \\
\end{align*}

\paragraph{Bsp}
\begin{description}
    \item[a] "`ist blutsverwandt"' auf Menge der Personen
    \item[b] "`$x$ ist Teiler von $y$"', d.h. $\exists z$ mit $x\cdot z=y\; x,y,z\in \mathds{Z}$
    \item[c] $R\subseteq \mathds{Z} \times \mathds{Z}$ mit $R=\{(x,y) | xy > 0\}$
    \item[d] "`$w$ ist Präfix von $v$"', d.h. $\exists u$ mit $v=wu$,\; $u,v,w\in\Sigma^*,\;\Sigma$ sei Menge
\end{description}

\begin{tabular}{c|c|c|c|c|c|c|c|c|c|}
\hline
& refl. & irrefl. & symm. & antiymm. & asymm. & trans. & äquiv. & "`<"' & "`$\leq$"' \\
\hline
a & \correct & \wrong & \correct & \wrong & \wrong & \wrong & \wrong & \wrong & \wrong \\
\hline
b & \correct & \wrong & \wrong & \wrong & \wrong & \correct & \wrong & \wrong & \wrong \\
\hline
c & \wrong   & \wrong & \correct & \wrong & \wrong & \correct & \wrong & \wrong & \wrong \\
\hline
d & \correct & \wrong & \wrong & \correct & \wrong & \correct & \wrong & \correct & \wrong \\
\hline
\end{tabular}

\begin{description}
    \item[Äquivalenzklasse:] Sei $R \subseteq A\times A$ eine ÄR, dann heissen für $a\in A$ die Mengen $[a]_R:=\{b\in A | aRb\}$ Äquivalenzklassen von R.
    \item[Partitionierung:] Menge aller ÄK $A/_R:=\{[a]_R | a\in A\}$
    \item[Index] Anzahl der ÄK $|A/_R|$ \\
    Bsp $R=\{(x,y) | x \mod 2 = y \mod 2\}$ partitioniert $\mathds{N}^0$ in 2 ÄK:
    \begin{itemize}
        \item $[0]_a = \{0,2,3,\dots\}$
        \item $[1]_a = \{1,3,5,\dots\}$
    \end{itemize}
\end{description}

\subsection*{Grammatik}
Eine Grammatik ist ein 4-Tupel $G=(V,\Sigma,P,S)$ \\[.5cm]
\begin{minipage}[t]{4.5cm}
\begin{description}
\item[V -] Variblen
\item[$\Sigma$ -] Terminalalphabet
\item[P -] Regeln/Produktionen
\item[S -] Startvariable
\end{description}
\end{minipage}
\begin{minipage}[t]{10cm}
    \begin{align*}
        |V|         & < \infty \\
        |\Sigma|    & < \infty \\
        |P|         & < \infty
    \end{align*}
\end{minipage}

\subsection*{Chomsky-Hirarchie}
\begin{description}
\item[Typ 0:] (Phrasenstrukturgrammatik) - keine Einschränkungen
\item[Typ 1:] (kontextsensitiv) - $(w_1\to w_2) \Rightarrow (\vert w_1 \vert \leq \vert w_2 \vert)$ {\tiny (Wort wird nicht kürzer)}
\item[Typ 2:] (kontextfrei) - $(w_1\to w_2 \Rightarrow (w_1\in V)$ {\tiny $w_1$ ist einzelne Variable}
\item[Typ 3:] (regulär) - $w_2\in\Sigma\cup\Sigma V$ {\tiny "`rechte Seiten"' von Regeln Terminalsymbol oder Terminalsymbole gefolgt von Variablen}
\end{description}
Alle Sprachen der Typen 1,2 und 3 sind \emph{entscheidbar}.

\subsection*{$\varepsilon$-Sonderregelung {\tiny (Zulassen des leeren Wortes $\varepsilon$ in Typ 1,2 oder 3)}}
\begin{itemize}
\item Regel hinzufügen: $S\to\varepsilon$
\item Verhindern von $S$ auf rechter Seite von Regeln: Regel mit "`$\to S$"' ersetzen durch "`$\to S'$"'
\item Zulassen von $A\to\varepsilon$ {\tiny (verändert Sprache nicht)}\\
Algorithmus:
\begin{enumerate}
\item Zerlege $V\to V_1,V_2$, $(A\Rightarrow^*\varepsilon)\in V_1$ und $V_1\cap V_2=\emptyset$.
\item Entferne alle $A\to\varepsilon$, füge für $(B\to xAy)$ $(B\to xy)$ hinzu.
\end{enumerate}
\end{itemize}

\subsection*{Wortproblem {\tiny (Gehört ein Wort zu einer Sprache?)}}
$(\exists \textrm{Algorithmus})[(\textrm{Algo terminiert in endl. Zeit}\wedge(Algo entscheidet (x\in\mathcal{L}(G))\vee(x\not\in\mathcal{L}(G)))]$ \\
$\Rightarrow$ das Wortproblem ist für Typ 1,2 und 3 entscheidbar (aber NP-hart für Typ 1)

\subsection*{Syntaxbäume}
Wurzel: $S$ \\
Für $i=1,2,\ldots,n\; A\to z\in P\Rightarrow\vert z\vert\textrm{ viele Söhne }\to \textrm{"`weitere Kette"'}$\\
\begin{description}
\item[Linksableitung:] Variable am weitesten links wird abgeleitet.
\item[Rechtsableitung:] Variable am weitesten rechts wird abgeleitet.
\item[mehrdeutige Grammatik:] für ein $x$ verschiedene Syntaxbäume möglich
	\begin{itemize}
	\item Mehrdeutigkeit kann oft beseitigt werden.
	\item Ist dies nicht möglich $\Rightarrow$ \emph{inhärent mehrdeutig}
	\end{itemize}
\end{description}

\subsection*{Backus-Naur-Form Bnf (Typ 2 Grammatiken)}
Metaregeln für selbe linke Seite
\[
\left. \begin{array}{ccc}
A & \to    & \beta_1 \\
A & \to    & \beta_2 \\
  & \vdots &         \\
A & \to    & \beta_3
\end{array} \right\rbrace A\to\beta_1\vert\beta_2\vert\ldots\beta_n
\]

\subsection*{erweiterte Backus-Naur-Form Ebnf}
\begin{align*}
A &\to \alpha[\beta]\gamma \Rightarrow \left\lbrace
\begin{array}{ccc}
A & \to & \alpha\gamma \\
A & \to & \alpha\beta\gamma
\end{array}
\right. \\
A &\to \alpha\{\beta\}\gamma \Rightarrow \left\lbrace
\begin{array}{ccc}
A & \to & \alpha\gamma \\
A & \to & \alpha B\gamma \\
B & \to & \beta \\
B & \to & \beta B
\end{array}
\right.
\end{align*}

\section*{Reguläre Sprachen}
\subsection*{Endliche (deterministische) Automaten DFA}
\begin{minipage}{6cm}
    $M = \left(Z,\Sigma,\delta,z_0,E\right)$
    \begin{description}
        \item[$\Sigma$] Eingabealphabet
        \item[$z_0$] Startzustand
        \item[$z\cap\Sigma = \emptyset$]
    \end{description}
\end{minipage}
\begin{minipage}{5.5cm}
\begin{description}
    \item[$Z$] Menge der Zustände
    \item[$E\subseteq Z$] Menge der Endzustände
    \item[$\delta: z\times E\to z$] Überführungsfunktion
    \item[$|z|< \infty,\; |\Sigma |<\infty$]
\end{description}
\end{minipage}\\
$\to$ Zustandsgraphen

\subsection*{akzeptierte Sprache}
Die von $M$ akzeptierte Sprache ist:
\begin{align*}
    T(M) &= \{x\in\Sigma^* \vert \hat\delta(z,e)\} \\
    & \textrm{wobei} & \hat\delta(z,\varepsilon)&=z \\
    & & \hat\delta(z,ax) &= \hat\delta\left(\delta(z,a),x\right)
\end{align*}

\begin{shaded}
    \ \\ Jede durch Endliche Automaten erkennbare Sprache ist Regulär (Typ 3).\\
    Jede Reguläre Sprache ist durch einen Endlichen Autoamten erkennbar.
\end{shaded}

\subsection*{Nichtdeterministische Automaten} % (fold)
Ein \emph{nichtdeterministischer, endlicher Automat} (\textsc{Nfa}) wird spezifiziert durch ein 5-Tupel: \\
\begin{minipage}{6cm}
    $M = \left(Z,\Sigma,\delta,S,E\right)$
    \begin{description}
        \item[$\Sigma$] Eingabealphabet
        \item[$S\subseteq Z$] Menge der Startzustände
        \item[$z\cap\Sigma = \emptyset$]
    \end{description}
\end{minipage}
\begin{minipage}{7cm}
\begin{description}
    \item[$Z$] Menge der Zustände
    \item[$E\subseteq Z$] Menge der Endzustände
    \item[$\delta: Z\times E\to \mathcal{P}(z)$] Überführungsfunktion
    \item[$|z|< \infty,\; |\Sigma |<\infty$]
\end{description}
\end{minipage}\\

\begin{shaded}
    Jede durch einen \textsc{Nfa} akzeptierbare Sprache ist auch durch einen \textsc{Dfa} akzeptierbar.
\end{shaded}

\begin{shaded}
    Für jede Reguläre Grammatik $G$ gibt es einen \textsc{Nfa} $M$ mit $L(G)=T(M)$.
\end{shaded}

\subsection*{regulärer Ausdruck} % (fold)
\begin{description}
    \item[$\emptyset$] ist regulärer Ausdruck.
    \item[$\varepsilon$] ist regulärer Audruck.
    \item[$a\in\Sigma$] ist regulärer Audruck.
    \item[$\alpha\beta, \left(\alpha\vert\beta\right), \left(\alpha\right)^* $] sind reguläre Ausdrücke, wenn $\alpha,\beta$ reguläre Ausrdücke sind.
\end{description}

\begin{minipage}[t]{0.48\textwidth}
    \begin{align*}
        \gamma &:= \emptyset & \Rightarrow L(\gamma) &= \emptyset \\
        \gamma &:=\varepsilon & \Rightarrow L(\gamma) &= \{\varepsilon\} \\
        \gamma &:=\alpha & \Rightarrow L(\gamma) &= \{\alpha\} \\
        \gamma &:=\alpha\beta & \Rightarrow L(\gamma) &= L\{\alpha\}L\{\beta\}
    \end{align*}
\end{minipage}
\hfill \vrule \hfill
\begin{minipage}[t]{0.48\textwidth}
    \begin{align*}
        \gamma &:= \left(\alpha\vert\beta\right) & \Rightarrow L\left(\gamma\right)&=L(\alpha)\cup L(\beta) \\
        \gamma &:= \left(\alpha\right)^* & \Rightarrow L(\gamma) &= L(\gamma)^* \\
        \gamma &:= \alpha\alpha = \alpha^2
    \end{align*}
\end{minipage}

\begin{shaded}
    Die Menge der durch Reguläre Ausdrücke beschreibbaren Sprachen ist genau die Menge der Regulären Sprachen.
\end{shaded}

\subsection*{Pumping-Lemma {\tiny Schleifenlemma, Iterationslemma, Lemma von Bar-Hillel, $uvw-$Theorem}} % (fold)
\begin{shaded}
    Sei $L$ eine Reguläre Sprache. Dann gibt es eine Zahl $n$ so, dass sich alle Wörter $x\in L$ mit $|x|\geq n$ zerlegen lasen in $x=uwv$, so dass folgende Eigenschaften erfüllt sind:
    \begin{itemize}
        \item $|v|\geq1$
        \item $|uv|\leq n$
        \item $\forall i\in\{0,1,\dots\}$ gilt: $uv^iw\in L$
    \end{itemize}
\end{shaded}
\begin{description}
    \item[$\to$] Zum Erkennen von nicht regulären Sprachen (geht nicht bei allen).
    \item[$\to$] $v^i$-Schleifen
\end{description}
Anwendung: Annehmen, $L$ sei regulär \follows Wenn Widerspruch, $L$ nicht regulär.

\subsection{Äquivalenzrelationen und Minimalautomaten}
$xR_Ly$ gdw für alle $z\in\Sigma^*: xz\in L$
\begin{shaded}
    Eine Sprache $L$ ist genau dann Regulär, wenn der Index von $R_L$ endlich ist:
    \[L \textrm{ regulär } \Leftrightarrow Index(R_L) < \infty \]
    {\tiny $Index$: Anzahl erzeugbarer Äquivalenzklassen}
\end{shaded}

Äquivalenzklassenautomat $\equiv$ Minimalautomat (nur DFA!) \\
Minimalautomat $\to$ Minimalautomat mit min Zustandszahl

\subsubsection*{Algorithmus Minimalautomat $\mathcal{O}(n^2)$}
\textbf{Eingabe:} \textsc{Dfa}, alle Zustände erreichbar.
\textbf{Ausgabe:} zu verschmelzende Zustände
\begin{enumerate}
    \item Tabelle mit Zustandspaaren $\{z,z'\}, z\not= z'$
    \item Markiere Paare $\{z,z'\}$ mit $z\in E$ und $z'\not\in E$ und umgekehrt
    \item unmarkierte Felder: Teste, ob $\underset{1\textrm{ hop!}}{\underbrace{\{\delta(z,a),\delta(z',a)\}}}$ markiert ist, wenn ja, markiere $\{z,z'\}$
    \item Punkt $3$ widerholen, bis keine Änderung mehr in der Tabelle passiert.
    \item Verschmelze unmarkierte Paare.
\end{enumerate}

\subsubsection*{Abschlusseigenschaften}
\begin{shaded}
    Reguläre Sprachen sind abgeschlossen unter:
    \begin{itemize}
        \item Vereinigung
        \item Schnitt
        \item Komplement
        \item Produkt
        \item Stern
    \end{itemize}
\end{shaded}

\subsubsection*{Entscheidbarkeit}
\begin{description}
    \item[Wortproblem:] Linearer Aufwand bei bekanntem \textsc{Dfa}
    \item[Leerheitsproblem:] Entscheidbar (\textsc{Dfa}, kein Weg $S\to E$)
    \item[Endlichkeitsproblem:] $T(M) = \infty \Leftrightarrow S\to\circlearrowright\to E$
    \item[Schnittproblem:] $\to$ Leerheitsproblem
    \item[Äquivalenzproblem] \textsc{Dfa} $\to$ Minimalautomat $\to$ Isomorphie checken $\to \mathcal{O}(n)$
    \[ L_1 = L_2 \Leftrightarrow \left(L_1 \cap \overline{L_2}\right) \cup \left(L_2 \cap L_1\right) = \emptyset \]
    $\to$ Leerheitsproblem $\to$ NP-hart
\end{description}

\subsection*{Kontextfreie Sprachen}
{\tiny korrekt geklammerte Ausdrücke}
\subsubsection*{Normalformen}
\begin{description}
    \item[Chromsky-Normalform:] (\textsc{Cnf}) Alle Regeln haben die Form $A\to BC$ oder $A\to a$
    \begin{shaded}
        Zu jeder Kontextfreien Grammatik $G$ mit $\varepsilon\not\in L(G)$ gibt es eine Chromsky-Normalform-Grammatik $G'$ mit $L(G) = L(G')$.
    \end{shaded}
    \item[Greibach-Normalform:] (\textsc{Gnf}) Alle Regeln haben die Form $A\to aB_1B_2\dots B_k\; (K\geq0)$ \\ {\tiny Erweiterung der Regulären Sprachen, hier nur $k=0$ und $k=1$}
    \begin{shaded}
        Zu jeder Kontextfreien Grammatik $G$ mit $\varepsilon\not\in L(G)$ gibt es eine Greibach-Normalform-Grammatik $G'$ mit $L(G) = L(G')$.
    \end{shaded}
\end{description}

\subsection*{Pumpinglemma}
\begin{shaded}
    Sei $L$ eine kontextfreie Sprache. Dann gibt es eine Zahl $n\in \mathds{N}$, so adss sich alle Wörter $z\in L$ mit $|z|\geq n$ zerlegen lassen in $z=uvwxy$ mit folgenden Eigenschaften:
    \begin{itemize}
        \item $|vx|\geq 1$
        \item $|vwx|\leq n$
        \item $\forall i\geq0$ gilt $uv^iwx^iy\in L$
    \end{itemize}
    Jede kontextfreie Sprache über einem einelementigen Alphabet ist bereits regulär.
\end{shaded}

\subsection*{Abschlusseigenschaften}
\begin{shaded}
Die Kontextfreien Sprachen sind\dots \\
\begin{minipage}[t]{0.48\textwidth}
\dots abgeschlossen unter
\begin{itemize}
    \item Vereinigung
    \item Produkt
    \item Stern
\end{itemize}
\end{minipage}
\hfill \vrule \hfill
\begin{minipage}[t]{0.48\textwidth}
\dots nicht abgeschlossen unter
\begin{itemize}
    \item Schnitt
    \item Komplement
\end{itemize}
\end{minipage}
\end{shaded}

\subsection*{Der CYK-Algorithmus}
\begin{algorithm}[H]
    \caption{CYK}
    \SetAlgoLined
    \LinesNumbered
    \PrintSemicolon
    \KwData{$x=a_1a_2\dots a_n$}
    %\KwResult{}
    initialisation\;
    \For{$i:=1$ \KwTo $n$ {\tiny ($j=1$)}}{
        $T[i,1]:=\{A\in V | A\to a_i \in P\}$
    }
     \For{$j:=2$ \KwTo n {\tiny ($j>1$)}}{
         \For{$i:=1$ \KwTo $n+1-j$}{
             $T[i,j]:=\emptyset$
             \For{$k:=1$ \KwTo $j-1$}{
                $T[i,j] := T[i,j]\cup \{A\in V | A\to BC\in P\wedge B\in T[i,k]\wedge C\in T[i+k,j-k]\}$
             }
         }
     }
     \eIf{$S\in T[1,n]$}{
     "`$x$ liegt in $L(G)$"'
     }{
     "`$x$ liegt nicht in $L(G)$"'
     }
\end{algorithm}

\subsection*{Kellerautomaten (PDA)}
\begin{shaded}
    Ein \emph{(nichtdeterministischer) Kellerautomat} (\emph{pushdown automaton}) wird angegeben durch ein 6-Tupel: $M=\left(Z,\Sigma,\Gamma,\delta,z_0,\#\right) $ \\
    \begin{minipage}[t]{0.48\textwidth}
        \begin{itemize}
            \item $Z$ - endliche Zustandsmenge
            \item $\Sigma$ - Eingabealphabet
            \item $\Gamma$ - Kelleralphabet
        \end{itemize}
    \end{minipage}
    \hfill \vrule \hfill
    \begin{minipage}[t]{0.48\textwidth}
        \begin{itemize}
            \item $ \delta:Z\times(\Sigma\cup\{\varepsilon\})\times\Gamma\to\mathcal{P}_e (Z\times\Gamma^* $ - die Überfühungsfunktion
            \item $z_0\in Z$ Startzustand
            \item $\#\in\Gamma$ unterstes Kellerzeichen
        \end{itemize}
    \end{minipage}
\end{shaded}

Um Schreibarbeit zu sparen, schreibt man statt $(z',x)\in\delta(z,a,A)$ einfach $zaA\to z'x$.

\begin{shaded}
    Eine Sprache $L$ ist kontextfrei genau dann, wenn $L$ von einem nichtdeterministischen Kellerautomaten erkannt wird.
\end{shaded}

\subsection*{Deterministisch kontextfreie Sprachen}

\end{document}
